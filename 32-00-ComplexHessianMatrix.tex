\documentclass[12pt]{article}
\usepackage{pmmeta}
\pmcanonicalname{ComplexHessianMatrix}
\pmcreated{2013-03-22 14:31:16}
\pmmodified{2013-03-22 14:31:16}
\pmowner{jirka}{4157}
\pmmodifier{jirka}{4157}
\pmtitle{complex Hessian matrix}
\pmrecord{7}{36062}
\pmprivacy{1}
\pmauthor{jirka}{4157}
\pmtype{Definition}
\pmcomment{trigger rebuild}
\pmclassification{msc}{32-00}
\pmrelated{HessianMatrix}

\endmetadata

% this is the default PlanetMath preamble.  as your knowledge
% of TeX increases, you will probably want to edit this, but
% it should be fine as is for beginners.

% almost certainly you want these
\usepackage{amssymb}
\usepackage{amsmath}
\usepackage{amsfonts}

% used for TeXing text within eps files
%\usepackage{psfrag}
% need this for including graphics (\includegraphics)
%\usepackage{graphicx}
% for neatly defining theorems and propositions
%\usepackage{amsthm}
% making logically defined graphics
%%%\usepackage{xypic}

% there are many more packages, add them here as you need them

% define commands here
\begin{document}
Suppose that $f \colon {\mathbb{C}}^n \to \mathbb{C}$ be twice differentiable
and let
\begin{equation*}
\frac{\partial}{\partial z_k} :=
\frac{1}{2}\left(
  \frac{\partial}{\partial x_k} - i \frac{\partial}{\partial y_k}
\right)
\quad \text{ and } \quad
\frac{\partial}{\partial \bar{z}_k} :=
\frac{1}{2}\left(
  \frac{\partial}{\partial x_k} + i \frac{\partial}{\partial y_k}
\right) .
\end{equation*}

Then the {\em \PMlinkescapetext{complex Hessian}} is the matrix
\begin{equation*}
\begin{bmatrix}
\frac{\partial^2 f}{\partial z_1 \partial \bar{z}_1} &
  \frac{\partial^2 f}{\partial z_1 \partial \bar{z}_2} &
    \ldots &
      \frac{\partial^2 f}{\partial z_1 \partial \bar{z}_n}
\\
\frac{\partial^2 f}{\partial z_2 \partial \bar{z}_1} &
  \frac{\partial^2 f}{\partial z_2 \partial \bar{z}_2} &
    \ldots &
      \frac{\partial^2 f}{\partial z_2 \partial \bar{z}_n}
\\
\vdots &
  \vdots &
    \ddots &
      \vdots
\\
\frac{\partial^2 f}{\partial z_n \partial \bar{z}_1} &
  \frac{\partial^2 f}{\partial z_n \partial \bar{z}_2} &
    \ldots &
      \frac{\partial^2 f}{\partial z_n \partial \bar{z}_n}
\end{bmatrix}
.
\end{equation*}

When applied to tangent vectors of the zero set of $f$,
it is called the Levi form and used to define a Levi
pseudoconvex point of a boundary of a domain.  Note that the \PMlinkescapetext{complex Hessian}
matrix is not the same as the \PMlinkescapetext{normal} (real) Hessian.  A twice continuously
differentiable real valued
function with a
positive semidefinite real Hessian matrix at every point is convex, but a function with
positive semidefinite \PMlinkescapetext{complex Hessian} matrix at every point is
plurisubharmonic (since it's
continuous it's also called a pseudoconvex function).

\begin{thebibliography}{9}
\bibitem{Krantz:several}
Steven~G.\@ Krantz.
{\em \PMlinkescapetext{Function Theory of Several Complex Variables}},
AMS Chelsea Publishing, Providence, Rhode Island, 1992.
\end{thebibliography}
%%%%%
%%%%%
\end{document}
