\documentclass[12pt]{article}
\usepackage{pmmeta}
\pmcanonicalname{ComplexLine}
\pmcreated{2013-03-22 14:29:05}
\pmmodified{2013-03-22 14:29:05}
\pmowner{jirka}{4157}
\pmmodifier{jirka}{4157}
\pmtitle{complex line}
\pmrecord{5}{36018}
\pmprivacy{1}
\pmauthor{jirka}{4157}
\pmtype{Definition}
\pmcomment{trigger rebuild}
\pmclassification{msc}{32-00}
\pmrelated{AffineTransformation}
\pmdefines{complex affine space}

\endmetadata

% this is the default PlanetMath preamble.  as your knowledge
% of TeX increases, you will probably want to edit this, but
% it should be fine as is for beginners.

% almost certainly you want these
\usepackage{amssymb}
\usepackage{amsmath}
\usepackage{amsfonts}

% used for TeXing text within eps files
%\usepackage{psfrag}
% need this for including graphics (\includegraphics)
%\usepackage{graphicx}
% for neatly defining theorems and propositions
\usepackage{amsthm}
% making logically defined graphics
%%%\usepackage{xypic}

% there are many more packages, add them here as you need them

% define commands here
\theoremstyle{theorem}
\newtheorem*{thm}{Theorem}
\newtheorem*{lemma}{Lemma}
\newtheorem*{conj}{Conjecture}
\newtheorem*{cor}{Corollary}
\theoremstyle{definition}
\newtheorem*{defn}{Definition}
\begin{document}
\begin{defn}
Let $a, b \in {\mathbb{C}}^n$.  The set
$\ell := \{a + b z \mid z \in {\mathbb{C}} \}$ is called the {\em complex line}.
\end{defn}

A complex line is a holomorphic complex affine imbedding of ${\mathbb{C}}$
into ${\mathbb{C}}^n$ so that if $f$ is holomorphic, then
$z \mapsto f(a + b z)$ is also holomorphic.  That is the complex structures of $\ell$ and ${\mathbb{C}}^n$ are compatible.  Hence not every two dimensional real affine space is a complex line.

\begin{defn}
Let $a, b_1, \ldots, b_k \in {\mathbb{C}}^n$ such that
$b_1, \ldots, b_k$ are linearly independent 
over ${\mathbb{C}}$,
then.  The set
\begin{equation*}
\{a + \sum_{j=1}^k b_k z_k \mid z_1,\ldots,z_k \in {\mathbb{C}} \}
\end{equation*}
is called the {\em complex affine space}.
\end{defn}

\begin{thebibliography}{9}
\bibitem{Krantz:several}
Steven~G.\@ Krantz.
{\em \PMlinkescapetext{Function Theory of Several Complex Variables}},
AMS Chelsea Publishing, Providence, Rhode Island, 1992.
\end{thebibliography}
%%%%%
%%%%%
\end{document}
