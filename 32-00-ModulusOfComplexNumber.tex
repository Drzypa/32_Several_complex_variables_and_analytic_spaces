\documentclass[12pt]{article}
\usepackage{pmmeta}
\pmcanonicalname{ModulusOfComplexNumber}
\pmcreated{2013-03-22 13:36:39}
\pmmodified{2013-03-22 13:36:39}
\pmowner{matte}{1858}
\pmmodifier{matte}{1858}
\pmtitle{modulus of complex number}
\pmrecord{17}{34242}
\pmprivacy{1}
\pmauthor{matte}{1858}
\pmtype{Definition}
\pmcomment{trigger rebuild}
\pmclassification{msc}{32-00}
\pmclassification{msc}{30-00}
\pmclassification{msc}{12D99}
\pmsynonym{complex modulus}{ModulusOfComplexNumber}
\pmsynonym{modulus}{ModulusOfComplexNumber}
\pmsynonym{absolute value of complex number}{ModulusOfComplexNumber}
\pmsynonym{absolute value}{ModulusOfComplexNumber}
\pmsynonym{modulus of a complex number}{ModulusOfComplexNumber}
\pmrelated{AbsoluteValue}
\pmrelated{Subadditive}
\pmrelated{SignumFunction}
\pmrelated{ComplexConjugate}
\pmrelated{PotentialOfHollowBall}
\pmrelated{ConvergenceOfRiemannZetaSeries}
\pmrelated{RealPartSeriesAndImaginaryPartSeries}
\pmrelated{ArgumentOfProductAndSum}
\pmrelated{ArgumentOfProductAndQuotient}
\pmrelated{EqualityOfComplexNumbers}

% this is the default PlanetMath preamble.  as your knowledge
% of TeX increases, you will probably want to edit this, but
% it should be fine as is for beginners.

% almost certainly you want these
\usepackage{amssymb}
\usepackage{amsmath}
\usepackage{amsfonts}

% used for TeXing text within eps files
%\usepackage{psfrag}
% need this for including graphics (\includegraphics)
%\usepackage{graphicx}
% for neatly defining theorems and propositions
%\usepackage{amsthm}
% making logically defined graphics
%%%\usepackage{xypic}

% there are many more packages, add them here as you need them

% define commands here

\newcommand{\sR}[0]{\mathbb{R}}
\newcommand{\sC}[0]{\mathbb{C}}
\newcommand{\sN}[0]{\mathbb{N}}
\newcommand{\sZ}[0]{\mathbb{Z}}
\begin{document}
\newcommand{\ccj}[1]{\overline{#1}}
{\bf Definition} 
Let $z$ be a complex number, and let 
$\ccj{z}$ be the complex conjugate of $z$. 
Then the \emph{modulus}, or \emph{absolute value}, of $z$ is defined as 
\[
  |z| := \sqrt{z\ccj{z}}.
\]
There is also the notation
$$\mod{z}$$
for the modulus of $z$.

If we write $z$ in polar form as\, $z = re^{i\phi}$\, with\, $r\ge 0,\; \phi\in[0,\,2\pi)$,\,
then\, $|z| = r$.  It follows that the modulus is a positive real number or zero.  
Alternatively, if $a$ is the real part of $z$, and $b$ the imaginary part, then 
\begin{eqnarray}
\label{eq100}
|z| &=& \sqrt{a^2+b^2},
\end{eqnarray}
which is simply the Euclidean norm of the point \,$(a,\,b)\in \sR^2$. 
It follows that the modulus satisfies the triangle inequality
\[
   |z_1+z_2| \le |z_1|+|z_2|, 
\]
also
$$|\Re{z}| \le |z|,\quad |\Im{z}| \le |z|,\quad |z| \le |\Re{z}|+|\Im{z}|.$$

Modulus is \PMlinkescapetext{multiplicative}:
$$|z_1z_2| = |z_1|\cdot|z_2|, \quad 
\left|\frac{z_1}{z_2}\right| = \frac{|z_1|}{|z_2|}$$



Since $\sR\subset\sC$, the definition of modulus includes the real numbers.  Explicitly, if we write\, $x\in\sR$\, in polar form,\, $x = re^{i\phi}$,\, $r > 0$,\, $\phi\in[0,2\pi)$,\, then\, $\phi = 0$\, or\, $\phi=\pi$, so\, $e^{i\phi}=\pm 1$.  Thus,
\[
  |x| = \sqrt{x^2} =
  \begin{cases}
    x  & x>0 \\
    0  & x=0 \\
    -x & x<0
  \end{cases}
.\]
%%%%%
%%%%%
\end{document}
