\documentclass[12pt]{article}
\usepackage{pmmeta}
\pmcanonicalname{ComplexLogarithm}
\pmcreated{2013-03-22 14:43:11}
\pmmodified{2013-03-22 14:43:11}
\pmowner{pahio}{2872}
\pmmodifier{pahio}{2872}
\pmtitle{complex logarithm}
\pmrecord{10}{36342}
\pmprivacy{1}
\pmauthor{pahio}{2872}
\pmtype{Definition}
\pmcomment{trigger rebuild}
\pmclassification{msc}{32A05}
\pmclassification{msc}{30D20}
\pmsynonym{natural logarithm}{ComplexLogarithm}
\pmrelated{Logarithm}
\pmrelated{NaturalLogarithm2}
\pmrelated{ValuesOfComplexCosine}
\pmrelated{EqualityOfComplexNumbers}
\pmrelated{SomeValuesCharacterisingI}
\pmrelated{UsingResidueTheoremNearBranchPoint}

% this is the default PlanetMath preamble.  as your knowledge
% of TeX increases, you will probably want to edit this, but
% it should be fine as is for beginners.

% almost certainly you want these
\usepackage{amssymb}
\usepackage{amsmath}
\usepackage{amsfonts}

% used for TeXing text within eps files
%\usepackage{psfrag}
% need this for including graphics (\includegraphics)
%\usepackage{graphicx}
% for neatly defining theorems and propositions
%\usepackage{amsthm}
% making logically defined graphics
%%%\usepackage{xypic}

% there are many more packages, add them here as you need them

% define commands here
\begin{document}
The \PMlinkescapetext{{\em logarithm} of a complex number} $z$ is defined as every complex number $w$ which satisfies the equation
\begin{align}
                              e^w = z.
\end{align}
This is is denoted by 
                      $$\log{z} := w.$$

The solution of (1) is obtained by using the form \,$e^w = re^{i\varphi}$\,, where\, $r = |z|$\, and \,$\varphi = \arg{z}$;\, the result is
            $$w = \log{z} = \ln{|z|}+i\arg{z}.$$
Here, the $\ln|z|$ means the usual Napierian or \PMlinkname{natural logarithm}{NaturalLogarithm2} (`{\em logarithmus naturalis}') of the real number $|z|$.\, If we fix the phase angle $\varphi$ of $|z|$ so that \,$0 \leqq \varphi < 2\pi$, we can write
    $$\log{z} = \ln{r}+i\varphi+n\cdot 2\pi i\quad(n = 0,\,\pm1,\,\pm2,\,...).$$

The complex logarithm $\log{z}$ is defined for all \,$z \neq 0$\, and it is infinitely multivalued $-$ e.g.\, $\log{(-1)} = (2n+1)\pi i$\, where $n$ is an arbitrary integer.\, The values with\, $n = 0$\, are called the \PMlinkescapetext{{\em principal values}} of the \PMlinkescapetext{logarithm}; if $z$ is real, the \PMlinkescapetext{principal} value of $\log{z}$ coincides with $\ln{z}$.
%%%%%
%%%%%
\end{document}
