\documentclass[12pt]{article}
\usepackage{pmmeta}
\pmcanonicalname{DerivativeOfExponentialFunction}
\pmcreated{2013-03-22 17:01:39}
\pmmodified{2013-03-22 17:01:39}
\pmowner{rspuzio}{6075}
\pmmodifier{rspuzio}{6075}
\pmtitle{derivative of exponential function}
\pmrecord{15}{39313}
\pmprivacy{1}
\pmauthor{rspuzio}{6075}
\pmtype{Theorem}
\pmcomment{trigger rebuild}
\pmclassification{msc}{32A05}
\pmrelated{ExponentialFunction}
\pmrelated{ComplexExponentialFunction}
\pmrelated{DerivativeOfTheNaturalLogarithmFunction}

% this is the default PlanetMath preamble.  as your knowledge
% of TeX increases, you will probably want to edit this, but
% it should be fine as is for beginners.

% almost certainly you want these
\usepackage{amssymb}
\usepackage{amsmath}
\usepackage{amsfonts}

% used for TeXing text within eps files
%\usepackage{psfrag}
% need this for including graphics (\includegraphics)
%\usepackage{graphicx}
% for neatly defining theorems and propositions
\usepackage{amsthm}
% making logically defined graphics
%%%\usepackage{xypic}

% there are many more packages, add them here as you need them

% define commands here
\newtheorem{thm}{Theorem}
\begin{document}
In this entry, we shall compute the derivative of the exponential
function from its definition as a limit of powers.

\begin{thm}
If $0 \le x < 1$, then
\[
1 + x \le \exp x \le {1 \over 1-x}
\]
\end{thm}

\begin{proof}
By the inequalities for differences of powers, we have
\[
x \le 
\left( 1 + {x \over n} \right)^n - 1 \le 
{x \over 1 - \left( {n-1 \over n} \right) x} .
\]
Since $n-1 < n$, and $x > 0$, we have $0 < (n-1 / n) x < x$.  Because
$x < 1$, this implies $1 - (n-1 / n) x > 1 - x$, so
\[
{x \over 1 - \left( {n-1 \over n} \right) x} <
{x \over 1 - x}.
\]
Hence 
\[
1 + x \le \left( 1 + {x \over n} \right)^n \le {1 \over 1 - x}.
\]
Taking the limit as $n \to \infty$, we obtain our result.
\end{proof}

\begin{thm}
\[
\lim_{x \to 0} {\exp (x) - 1 \over x} = 1
\]
\end{thm}

\begin{proof}
Assume $0 < x < 1$.  By our bound, we have
\[
1 \le
{\exp (x) - 1 \over x} \le
{1 \over 1 - x} .
\]
Suppose that $-1 < x < 0$.  Then, since $\exp (x) = 1 / \exp (-x)$, we have
\[
{\exp (x) - 1 \over x} =
{1 \over \exp (-x)} \cdot {1 - \exp (-x) \over x} .
\]
From the inequality above, we have
\[
1 \le
{1 - \exp (-x) \over x} \le
{1 \over 1+x} .
\]
Hence
\[
{1 \over \exp (-x)} \le
{\exp (x) - 1 \over x} \le
{1 \over (1+x) \exp (-x)}.
\]
By theorem 1, we have $1 - x \le \exp (-x) \le 1 / (1 + x)$, so
\[
1+x \le
{1 - \exp (-x) \over x} \le
{1 \over (1 + x)(1-x)} =
{1 \over 1 - x^2} .
\]
By the squeeze rule, we conclude that 
\[
\lim_{x \to 0} {1 - \exp (-x) \over x} = 1 
\]
whether we approach the limit from the left or the right.
\end{proof}

\begin{thm}
\[
{d \over dx} \exp (x) = \exp (x)
\]
\end{thm}

\begin{proof}
By definition,
\[
{d \over dx} \exp (x) = 
\lim_{y \to x} {\exp (y) - \exp (x) \over y - x} .
\]
By the addition theorem for the exponential, we have
\[
{\exp (y) - \exp (x) \over y - x} =
\exp (x) \cdot {\exp (y-x) - 1 \over y - x} ,
\]
so
\[
\lim_{y \to x} {\exp (y) - \exp (x) \over y - x} =
\exp (x) \lim_{y \to x}
{\exp (y-x) - 1 \over y - x} =
\exp (x) \lim_{y \to 0} {\exp y - 1 \over y} .
\]
By theorem 2, the limit on the right-hand side equals $1$, so we have
\[
\lim_{y \to x} {\exp (y) - \exp (x) \over y - x} = \exp (x) .
\]
\end{proof}
%%%%%
%%%%%
\end{document}
