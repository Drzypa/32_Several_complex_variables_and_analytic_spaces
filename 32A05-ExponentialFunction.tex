\documentclass[12pt]{article}
\usepackage{pmmeta}
\pmcanonicalname{ExponentialFunction}
\pmcreated{2013-03-22 12:26:25}
\pmmodified{2013-03-22 12:26:25}
\pmowner{CWoo}{3771}
\pmmodifier{CWoo}{3771}
\pmtitle{exponential function}
\pmrecord{16}{32541}
\pmprivacy{1}
\pmauthor{CWoo}{3771}
\pmtype{Definition}
\pmcomment{trigger rebuild}
\pmclassification{msc}{32A05}
\pmsynonym{logarithm function}{ExponentialFunction}
\pmrelated{FourExponentialsConjecture}
\pmrelated{ComplexExponentialFunction}
\pmrelated{ExampleOfTaylorPolynomialsForTheExponentialFunction}
\pmrelated{ProofOfEquivalenceOfFormulasForExp}
\pmrelated{GroupsOfRealNumbers}
\pmrelated{FunctionXx}
\pmrelated{ExampleOfJumpDiscontinuity}
\pmrelated{ExponentialFunctionDefinedAsLimitOfPowers}
\pmrelated{Deriva}

% this is the default PlanetMath preamble.  as your knowledge
% of TeX increases, you will probably want to edit this, but
% it should be fine as is for beginners.

% almost certainly you want these
\usepackage{amssymb}
\usepackage{amsmath}
\usepackage{amsfonts}
\usepackage{amsthm}
\usepackage{pstricks}
\usepackage{pst-plot}

% used for TeXing text within eps files
%\usepackage{psfrag}
% need this for including graphics (\includegraphics)
%\usepackage{graphicx}
% for neatly defining theorems and propositions
%\usepackage{amsthm}
% making logically defined graphics
%%%\usepackage{xypic}

% there are many more packages, add them here as you need them

% define commands here

\newcommand{\mc}{\mathcal}
\newcommand{\mb}{\mathbb}
\newcommand{\mf}{\mathfrak}
\newcommand{\ol}{\overline}
\newcommand{\ra}{\rightarrow}
\newcommand{\la}{\leftarrow}
\newcommand{\La}{\Leftarrow}
\newcommand{\Ra}{\Rightarrow}
\newcommand{\nor}{\vartriangleleft}
\newcommand{\Gal}{\text{Gal}}
\newcommand{\GL}{\text{GL}}
\newcommand{\Z}{\mb{Z}}
\newcommand{\R}{\mb{R}}
\newcommand{\Q}{\mb{Q}}
\newcommand{\C}{\mb{C}}
\newcommand{\<}{\langle}
\renewcommand{\>}{\rangle}
\begin{document}
We begin by defining the exponential function $\exp:\mb{R}\ra\mb{R}^+$ for all real values of $x$ by the power series

$$\exp(x) = \sum_{k = 0}^{\infty}\frac{x^k}{k!}$$

The graph of the exponential function is as follows:

\begin{center}
\psset{yunit=8pt}
\begin{pspicture}(-4,-5)(4,25)
\psaxes[Dx=1,Dy=2]{->}(0,0)(-3.5,-1.5)(3.5,24)
\rput(0.3,24){$y$}
\rput(3.4,1.4){$x$}
\psplot[linecolor=blue]{-3.3}{3.2}{2.7 x exp}
\end{pspicture}
\end{center}

It has a few elementary properties, which can be easily shown.

\begin{itemize}
\item The radius of convergence is infinite
\item $\exp(0) = 1$
\item It is infinitely differentiable, and the derivative is the exponential function itself
\item $\exp(x) \ge 1 + x$ so it is positive and unbounded on the non-negative reals
\end{itemize}

Now consider the function $f:\mathbb{R}\rightarrow\mathbb{R}$ with

$$f(x) = \exp(x)\exp(y-x)$$

so, by the product rule and property 3

$$f'(x) = 0.$$

Since the only continuous functions that have a derivative of 0 are constant functions, and since $f(0)=\exp(y)$, we get

$$\exp(x)\exp(y-x) = \exp(y) \quad \forall y,x \in \mathbb{R}.$$

With a suitable change of variables, we have

$$\exp(x+y) = \exp(x)\exp(y),$$
$$\exp(x)\exp(-x) = 1.$$

Consider just the non-negative reals. Since it is unbounded, by the intermediate value theorem, it can take any value on the interval $[1,\infty)$. We have that the derivative is strictly positive so by the mean-value theorem, $\exp(x)$ is strictly increasing. This gives surjectivity and injectivity i.e. it is a bijection from $[0,\infty) \rightarrow [1,\infty)$.

Now $\exp(-x) = \frac{1}{\exp(x)}$, so it is also a bijection from $(-\infty,0) \rightarrow (0, 1)$. Therefore we can say that $\exp(x)$ is a bijection onto $\mathbb{R}^+.$

We can now naturally define the logarithm function, as the inverse of the exponential function. It is usually denoted by $\operatorname{ln}(x)$, and it maps $\mathbb{R}^+$ to $\mathbb{R}.$

Similarly, the natural log base, $e$ may be defined by $e = \exp(1).$

Since the exponential function obeys the rules normally associated with powers, it is often denoted by $e^x$. In fact it is now possible to define powers in terms of the exponential function by

$$a^x = e^{x\operatorname{ln}(a)} \quad a > 0.$$

Note the domain may be extended to the complex plane with all the same properties as before, except the bijectivity and ordering properties.

Comparison with the power series expansions for sine and cosine yields the following identity, with the famous corollary attributed to Euler:

$$e^{ix} = \cos(x) + i\sin(x)$$
$$e^{i\pi}+1=0,$$

the last of which often being considered one of the most beautiful identities in all of mathematics, relating the important numbers $e$, $\pi$, $i$, $1$, and $0$ all in one equation.
%%%%%
%%%%%
\end{document}
