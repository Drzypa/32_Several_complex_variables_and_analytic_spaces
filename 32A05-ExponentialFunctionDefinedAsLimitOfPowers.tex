\documentclass[12pt]{article}
\usepackage{pmmeta}
\pmcanonicalname{ExponentialFunctionDefinedAsLimitOfPowers}
\pmcreated{2013-03-22 17:01:35}
\pmmodified{2013-03-22 17:01:35}
\pmowner{rspuzio}{6075}
\pmmodifier{rspuzio}{6075}
\pmtitle{exponential function defined as limit of powers}
\pmrecord{27}{39312}
\pmprivacy{1}
\pmauthor{rspuzio}{6075}
\pmtype{Definition}
\pmcomment{trigger rebuild}
\pmclassification{msc}{32A05}
\pmrelated{ExponentialFunction}
\pmrelated{ComplexExponentialFunction}
\pmrelated{ExponentialFunctionNeverVanishes}

% this is the default PlanetMath preamble.  as your knowledge
% of TeX increases, you will probably want to edit this, but
% it should be fine as is for beginners.

% almost certainly you want these
\usepackage{amssymb}
\usepackage{amsmath}
\usepackage{amsfonts}

% used for TeXing text within eps files
%\usepackage{psfrag}
% need this for including graphics (\includegraphics)
%\usepackage{graphicx}
% for neatly defining theorems and propositions
\usepackage{amsthm}
% making logically defined graphics
%%%\usepackage{xypic}

% there are many more packages, add them here as you need them

% define commands here
\newtheorem{dfn}{Definition}
\newtheorem{thm}{Theorem}
\begin{document}
It is possible to define the exponential function and the natural logarithm 
in terms of a limit of powers.  In this entry, we shall present these
definitions after some background information and demonstrate the basic
properties of these functions from these definitions.

Two basic results which are needed to make this development possible
are the following:

\begin{thm}
Let $x$ be a real number and let $n$ be an integer such that
$n > 0$ and $n + x > 0$.  Then
\[
\left( {n + x \over n} \right)^n < 
\left( {n + 1 + x \over n + 1} \right)^{n+1}.
\]
\end{thm}

\begin{thm}
Suppose that $\{s_n\}_{n=1}^\infty$ is a sequence such that
$\lim_{n \to \infty} n s_n = 0$.  Then $\lim_{n \to \infty} 
(1 + s_n)^n = 1$,
\end{thm}

For proofs, see the attachments.  From them, we first conclude
that a sequence converges.

\begin{thm}
Let $x$ be any real number.  Then the sequence
\[
\left\{ \left( {n + x \over n} \right)^n \right\}_{n=1}^\infty
\]
is convergent.
\end{thm}

The foregoing results show that the limit in the following
definition converges, and hence defines a bona fide function.

\begin{dfn}
Let $x$ be a real number.  Then we define
\[
\exp (x) = \lim_{n \to \infty}
\left( {n + x \over n} \right)^n .
\]
\end{dfn}

We may now derive some of the chief properties of this function.
starting with the addition formula.

\begin{thm}
For any two real numbers $x$ and $y$, we have $\exp (x+y) = \exp(x) \exp (y)$.
\end{thm}

\begin{proof}
Since
\[
 {n (n + x + y) \over (n + x) (n + y)} =
 1 - {xy \over (n + x) (n + y)}
\]
and
\[
 \lim_{n \to \infty} {n \over (n + x) (n + y)} = 0 ,
\]
theorem 2 above implies that
\[
 \lim_{n \to \infty}
 \left(
  {n (n + x + y) \over (n + x) (n + y)}
 \right)^n = 1 .
\]
Since it permissible to multiply convergent sequences termwise, we have
\begin{align*}
 \exp (x) \exp (y) &=
 \lim_{n \to \infty} \left( {n + x \over n} \right)^n
                     \left( {n + y \over n} \right)^n
                     \left(
                      {n (n + x + y) \over (n + x) (n + y)}
                     \right)^n \\ &=
 \lim_{n \to \infty} \left( {n + x + y \over n} \right)^n = 
 \exp (x+y)
\end{align*}
\end{proof}

\begin{thm}
The function $\exp$ is strictly increasing.
\end{thm}

\begin{proof}
Suppose that $x$ is a strictly positive real number.  By theorem 1 and 
the definition of the exponential as a limit, we have $1 + x < \exp (x)$,
so we conclude that $0 < x$ implies $1 < \exp (x)$.

Now, suppose that $x$ and $y$ are two real numbers with $x > y$.  Since
$x - y > 0$, we have $\exp (x-y) > 1$.  Using theorem 4, we have $\exp (x) 
= \exp (x-y) \exp (y) > \exp (y)$, so the function is strictly increasing.
\end{proof}

\begin{thm}
 The function $\exp$ is continuous.
\end{thm}

\begin{proof}
Suppose that $0 < x < 1$.  By theorem 1 and the definition of the exponential 
as a limit, we have $1 - x < \exp (-x)$ and $1 + x < \exp (x)$.  By theorem 4,
$\exp (x) \exp (-x) = \exp (0) = 0$.  Hence, we have the bounds $1 + x < \exp (x)
< 1/(1-x)$ and $1-x < \exp (-x) < 1/(1+x)$.  From the former bound, we conclude
that $\lim_{x \to 0-} \exp (x) = 1$ and, from the latter, that $\lim_{x \to 0+} 
\exp (x) = 1$, so $\lim_{x \to 0} \exp (x) = 1$.

Suppose that $y$ is any real number.  By theorem 4, $\exp (x + y) = \exp (x)
\exp (y)$.  Hence, $\lim_{x \to 0} \exp (x + y) = \exp (y) \lim_{x \to 0}
\exp (x) = \exp (y)$.  In other words, for all real $y$, we have $\lim_{ x
\to y} \exp (x) = \exp (y)$, so the exponential function is continuous.
\end{proof}

\begin{thm}
 The function $\exp$ is one-to-one and maps onto the positive real axis.
\end{thm}

\begin{proof}
The one-to-one property follows readily from monotonicity --- if $\exp (x) =
\exp (y)$, then we must have $x = y$, because otherwise, either $x < y$ or
$x > y$, which would imply $\exp (x) < \exp (y)$ or $\exp (x) > \exp (y)$,
respectively.  Next, suppose that $x$ is a real number greater than $1$.  
By theorem 1 and the definition of the exponential as a limit, we have 
$1 + x < \exp (x)$.  Thus, $1 < x < \exp (x)$; since $\exp$ is continuous,
the intermediate value theorem asserts that there must exist a real
number $y$ between $0$ and $x$ such that $\exp (y) = x$.  If, instead,
$0 < x < 1$, then $1/x > 1$ so we have a real number $y$ such that $\exp (y)
= 1/x$.  By theorem 4, we then have $\exp (-y) = x$.  So, given any real 
number $x > 0$, there exists a real number $y$ such that $\exp (y) = x$, 
hence the function maps onto the positive real axis.
\end{proof}

\begin{thm}
The function $\exp$ is convex.
\end{thm}

\begin{proof}
Since the function is already known to be continuous, it suffices to show
that $\exp ((x + y)/2) \le (\exp(x) + \exp (y))/2$ for all real numbers $x$ 
and $y$.  Changing variables, this is equivalent to showing that $2 \exp
(a + b) \le \exp (a) + \exp (a + 2b)$ for all real numbers $a$ and $b$.
By theorem 4, we have
\begin{align}
 \exp (a + b) &= \exp (a) \exp (b) \\
 \exp (a + 2b) &= \exp (a) \exp (b)^2 .
\end{align}
Using the inequality $2x \le 1 + x^2$ with $x = \exp (b)$ and multiplying
by $\exp (a)$, we conclude that $2 \exp (a + b) \le \exp (a) + \exp (a + 2b)$,
hence the exponential function is convex.
\end{proof}

Defining the constant $e$ as $\exp (1)$, we find that the exponential
function gives powers of this number.

\begin{thm}
For every real number $x$, we have $\exp (x) = e^x$.
\end{thm}

\begin{proof}
Applying an induction argument to theorem 4, it can be shown that $\exp (nx) =
\exp(x)^n$ for every real number $x$ and every integer $n$.  Hence, given a
rational number $m/n$, we have $\exp (m/n)^n = \exp (m) = \exp(1)^m = e^m$.
Thus, $exp(m/n) = e^{m/n}$ so we see that $\exp (x) = e^x$ when $x$ is a
rational number.  By continuity, it follows that $\exp (x) = e^x$ for every
real number $x$.
\end{proof}
%%%%%
%%%%%
\end{document}
