\documentclass[12pt]{article}
\usepackage{pmmeta}
\pmcanonicalname{ExponentialFunctionNeverVanishes}
\pmcreated{2014-11-22 21:23:32}
\pmmodified{2014-11-22 21:23:32}
\pmowner{pahio}{2872}
\pmmodifier{pahio}{2872}
\pmtitle{exponential function never vanishes}
\pmrecord{8}{41924}
\pmprivacy{1}
\pmauthor{pahio}{2872}
\pmtype{Theorem}
\pmcomment{trigger rebuild}
\pmclassification{msc}{32A05}
\pmclassification{msc}{30D20}
\pmsynonym{real exponential function is positive}{ExponentialFunctionNeverVanishes}
\pmrelated{ExponentialFunction}
\pmrelated{ExponentialFunctionDefinedAsLimitOfPowers}
\pmrelated{PeriodicityOfExponentialFunction}

\endmetadata

% this is the default PlanetMath preamble.  as your knowledge
% of TeX increases, you will probably want to edit this, but
% it should be fine as is for beginners.

% almost certainly you want these
\usepackage{amssymb}
\usepackage{amsmath}
\usepackage{amsfonts}

% used for TeXing text within eps files
%\usepackage{psfrag}
% need this for including graphics (\includegraphics)
%\usepackage{graphicx}
% for neatly defining theorems and propositions
 \usepackage{amsthm}
% making logically defined graphics
%%%\usepackage{xypic}

% there are many more packages, add them here as you need them

% define commands here

\theoremstyle{definition}
\newtheorem*{thmplain}{Theorem}

\begin{document}
In the entry \PMlinkname{exponential function}{ExponentialFunction} one defines for real variable $x$ the real exponential function \,$\exp{x}$, i.e. $e^x$, as the sum of power series:
$$e^x \;=\; \sum_{k=0}^\infty\frac{x^k}{k!}$$
The series form implies immediately that the real exponential 
function attains only positive values when\, $x \geqq 0$.\, Also 
for\, $-1 \leqq x < 0$\, the positiveness is easy to see by 
grouping the series terms pairwise.

In \PMlinkescapetext{order} to study the sign of $e^x$ for 
arbitrary real $x$, we may multiply the series of $e^x$ and 
$e^{-x}$ using \PMlinkname{Abel's multiplication rule for series}{AbelsMultiplicationRuleForSeries}.\, We obtain
$$e^xe^{-x} \;=\; \sum_{n=0}^\infty\frac{x^n}{n!}\!\sum_{k=0}^\infty(-1)^k\frac{x^k}{k!} 
\;=\; \sum_{n=0}^\infty\!\sum_{j=0}^n(-1)^j\frac{x^n}{j!(n\!-\!j)!} 
\;=\; \sum_{n=0}^\infty\frac{x^n}{n!}\!\sum_{j=0}^n\!{n\choose j}(-1)^j \;=\; \sum_{n=0}^\infty\frac{x^n}{n!}\cdot0^n.$$
The last sum equals 1.\, So, if\, $-x > 0$,\, then\, $e^{-x} > 0$,\, whence $e^x$ must be positive.\\

Let us now consider arbitrary complex value\, $z = x\!+\!iy$\, where $x$ and $y$ are real.\, Using the addition formula of complex exponential function and the Euler relation, we can write
$$e^z \;=\; e^{x+iy} \;=\; e^xe^{iy} \;=\; e^x(\cos{y}+i\sin{y}).$$
From this we see that the absolute value of $e^z$ is $e^x$, which we above have proved to be positive.\, Accordingly, we may write the

\textbf{Theorem.}\, The complex exponential function never vanishes.



%%%%%
%%%%%
\end{document}
