\documentclass[12pt]{article}
\usepackage{pmmeta}
\pmcanonicalname{GeneratingFunctionForTheReciprocalAlternatingCentralBinomialCoefficients}
\pmcreated{2013-03-22 18:58:12}
\pmmodified{2013-03-22 18:58:12}
\pmowner{juanman}{12619}
\pmmodifier{juanman}{12619}
\pmtitle{generating function for  the reciprocal alternating central binomial coefficients}
\pmrecord{14}{41830}
\pmprivacy{1}
\pmauthor{juanman}{12619}
\pmtype{Example}
\pmcomment{trigger rebuild}
\pmclassification{msc}{32A05}
\pmclassification{msc}{11B65}
\pmclassification{msc}{05A19}
\pmclassification{msc}{05A15}
\pmclassification{msc}{05A10}
%\pmkeywords{sum of reciprocals}
%\pmkeywords{alternanting series}
%\pmkeywords{sequence of inverses of integers}

\endmetadata

% this is the default PlanetMath preamble.  as your knowledge
% of TeX increases, you will probably want to edit this, but
% it should be fine as is for beginners.

% almost certainly you want these
\usepackage{amssymb}
\usepackage{amsmath}
\usepackage{amsfonts}

% used for TeXing text within eps files
%\usepackage{psfrag}
% need this for including graphics (\includegraphics)
%\usepackage{graphicx}
% for neatly defining theorems and propositions
%\usepackage{amsthm}
% making logically defined graphics
%%%\usepackage{xypic}

% there are many more packages, add them here as you need them

% define commands here

\begin{document}
It is also not very well known this relation:

$$\frac{
4\,\left(\,{\sqrt{x+4}}-{\sqrt{x}}\,{\rm{arcsinh}}(\frac{{\sqrt{x}}}{2})\right)
}
{
\sqrt{(x+4)^3}
}
=1-\frac{x}{2}+\frac{x^2}{6}-\frac{x^3}{20}+\frac{x^4}{70}-\frac{x^5}{252}+\frac{x^6}{924}-...$$
where one clearly appreciate that the function on LHS generates the sequence $(-1)^n{2n\choose n}^{-1}$.

To obtain the relation above one should use some kind of software because for the function is ``terrible'' to calculate derivatives of any order. It is a little challenge to give a recursive formula that gives the inverses of these alternating central binomial numbers, when evaluated at $x=0$ at those derivatives.
%%%%%
%%%%%
\end{document}
