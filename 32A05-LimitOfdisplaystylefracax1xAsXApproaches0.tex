\documentclass[12pt]{article}
\usepackage{pmmeta}
\pmcanonicalname{LimitOfdisplaystylefracax1xAsXApproaches0}
\pmcreated{2013-03-22 17:40:21}
\pmmodified{2013-03-22 17:40:21}
\pmowner{Wkbj79}{1863}
\pmmodifier{Wkbj79}{1863}
\pmtitle{limit of $\displaystyle \frac{a^x-1}{x}$ as $x$ approaches 0}
\pmrecord{5}{40111}
\pmprivacy{1}
\pmauthor{Wkbj79}{1863}
\pmtype{Corollary}
\pmcomment{trigger rebuild}
\pmclassification{msc}{32A05}

\endmetadata

\usepackage{amssymb}
\usepackage{amsmath}
\usepackage{amsfonts}
\usepackage{pstricks}
\usepackage{psfrag}
\usepackage{graphicx}
\usepackage{amsthm}
%%\usepackage{xypic}

\newtheorem*{cor*}{Corollary}
\begin{document}
\begin{cor*}
For $a>0$, we have
\[
\lim_{x\to 0}\frac{a^x-1}{x}=\ln a.
\]
\end{cor*}

\begin{proof}
Recall that $a^x=e^{x\ln a}$.  Thus,
\begin{align*}
\lim_{x\to 0}\frac{a^x-1}{x} & =\lim_{x\to 0}\frac{e^{x\ln a}-1}{x} \\
                             & =\lim_{x\to 0}\frac{(e^{x\ln a}-1)\ln a}{x\ln a} \\
                             & =(\ln a)\lim_{x\to 0}\frac{e^{x\ln a}-1}{x\ln a}.
\end{align*}
Let $t=x\ln a$.  Then $t\to 0$ as $x\to 0$.  Therefore,
\begin{align*}
\lim_{x\to 0}\frac{a^x-1}{x} & =(\ln a)\lim_{t\to 0}\frac{e^t-1}{t} \\
                             & =(\ln a)1 \\
                             & =\ln a.  \qedhere
\end{align*}
\end{proof}

The formula from the corollary is useful for proving that $\displaystyle \frac{d}{dx}a^x=a^x\ln a$.  On the other hand, once this fact is known, the corollary is easily proven via \PMlinkname{l'H\^{o}pital's rule}{LHpitalsRule}:

\begin{align*}
\lim_{x\to 0}\frac{a^x-1}{x} & =\lim_{x\to 0}\frac{a^x\ln a}{1} \\
                             & =a^0\ln a \\
                             & =\ln a.
\end{align*}
%%%%%
%%%%%
\end{document}
