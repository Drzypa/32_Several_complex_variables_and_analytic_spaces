\documentclass[12pt]{article}
\usepackage{pmmeta}
\pmcanonicalname{PeriodicityOfExponentialFunction}
\pmcreated{2014-02-20 14:29:59}
\pmmodified{2014-02-20 14:29:59}
\pmowner{pahio}{2872}
\pmmodifier{pahio}{2872}
\pmtitle{periodicity of exponential function}
\pmrecord{16}{37107}
\pmprivacy{1}
\pmauthor{pahio}{2872}
\pmtype{Theorem}
\pmcomment{trigger rebuild}
\pmclassification{msc}{32A05}
\pmclassification{msc}{30D20}
\pmsynonym{period of exponential function}{PeriodicityOfExponentialFunction}
\pmrelated{PeriodicFunctions}
\pmrelated{AnalyticContinuationOfRiemannZetaUsingIntegral}
\pmrelated{ExamplesOfPeriodicFunctions}
\pmrelated{ExponentialFunctionNeverVanishes}
\pmdefines{one-periodic}

% this is the default PlanetMath preamble.  as your knowledge
% of TeX increases, you will probably want to edit this, but
% it should be fine as is for beginners.

% almost certainly you want these
\usepackage{amssymb}
\usepackage{amsmath}
\usepackage{amsfonts}

% used for TeXing text within eps files
%\usepackage{psfrag}
% need this for including graphics (\includegraphics)
%\usepackage{graphicx}
% for neatly defining theorems and propositions
 \usepackage{amsthm}
% making logically defined graphics
%%%\usepackage{xypic}

% there are many more packages, add them here as you need them

% define commands here

\theoremstyle{definition}
\newtheorem*{thmplain}{Theorem}
\begin{document}
\begin{thmplain}
\, The only periods of the complex exponential function \,$z\mapsto e^z$\, are the multiples of $2\pi i$.\, Thus the function is {\em one-periodic}.
\end{thmplain}

{\em Proof.}\, Let $\omega$ be any period of the exponential function, i.e.\, $e^{z+\omega} = e^ze^\omega = e^z$\, for all\, $z\in\mathbb{C}$.\, Because $e^z$ is always $\neq 0$, we have
\begin{align}
e^\omega \;=\; 1.
\end{align}
If we set\, $\omega =: a\!+\!ib$\, with $a$ and $b$ reals, (1) gets the form
\begin{align}
e^a\cos{b}+ie^a\sin{b} \;=\; 1,
\end{align}
which implies (see equality of complex numbers)
$$e^a\cos{b} \;=\; 1,\qquad e^a\sin{b} \;=\; 0.$$
As these equations are squared and added, we obtain\, $e^{2a} = 1$\, which \PMlinkescapetext{means}, since $a$ is real, that\, $a = 0$.\, Thus the preceding equations get the form 
$$\cos{b} \;=\; 1,\qquad \sin{b} \;=\; 0.$$
These result that\, $b = n\!\cdot\!2\pi$\, and therefore
$$\omega \;=\; n\!\cdot\! 2\pi i \qquad (n \;=\; 0,\,\pm 1,\,\pm 2,\,\pm 3,\,\ldots)$$
Q.E.D.

\begin{thebibliography}{8}
\bibitem{lindelof}{\sc Ernst Lindel\"of}: {\em Johdatus funktioteoriaan} (`Introduction to function theory').\, Mercatorin kirjapaino, Helsinki (1936).
\end{thebibliography}
%%%%%
%%%%%
\end{document}
