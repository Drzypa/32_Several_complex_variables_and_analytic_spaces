\documentclass[12pt]{article}
\usepackage{pmmeta}
\pmcanonicalname{ProofOfGrowthOfExponentialFunction}
\pmcreated{2013-03-22 15:48:36}
\pmmodified{2013-03-22 15:48:36}
\pmowner{rspuzio}{6075}
\pmmodifier{rspuzio}{6075}
\pmtitle{proof of growth of exponential function}
\pmrecord{11}{37774}
\pmprivacy{1}
\pmauthor{rspuzio}{6075}
\pmtype{Proof}
\pmcomment{trigger rebuild}
\pmclassification{msc}{32A05}

\endmetadata

% this is the default PlanetMath preamble.  as your knowledge
% of TeX increases, you will probably want to edit this, but
% it should be fine as is for beginners.

% almost certainly you want these
\usepackage{amssymb}
\usepackage{amsmath}
\usepackage{amsfonts}

% used for TeXing text within eps files
%\usepackage{psfrag}
% need this for including graphics (\includegraphics)
%\usepackage{graphicx}
% for neatly defining theorems and propositions
%\usepackage{amsthm}
% making logically defined graphics
%%%\usepackage{xypic}

% there are many more packages, add them here as you need them

% define commands here
\begin{document}
In this proof, we first restrict to when $x$ and $a$ are integers
and only later lift this restricton.

Let $a > 0$ be an integer, let $b > 1$ be real, and let $x$ be an
integer.

Consider the following inequality
 \[ \left( 1 + {1 \over x} \right)^a \le 1 + {a \over x} \left( 1 + {1 \over x}
 \right)^{a-1}  \]
If $x \ge 2$, then we have
 \[ \left( 1 + {1 \over x} \right)^a \le 1  + {a \over
 x} \left( {3 \over 2} \right)^{a-1} . \]
Define $X$ to be the greater of $2$ and $\lceil a
(3/2)^{a-1} / (1 - \sqrt{b}) \rceil$; when $x > X$, we have 
 \[ \left( 1 + {1 \over x} \right)^a \le \sqrt{b}. \] 

Rewrite $x^a / b^x$ as follows when $x > X$:
 \[ {x^a \over b^x} = {X^a \over b^X} \prod_{n=X}^x \left( 1 + {1 \over n} 
 \right)^a {1 \over b} \] 
By the inequality established above, each term in the product will be
bounded by $1 / \sqrt{b}$, hence
 \[  {x^a \over b^x} \le {X^a \over b^X} {1 \over (\sqrt{b})^{x - X}}
 \]
Since $b > 1$, it is also the case that $\sqrt{b} > 1$, hence we have
the inequality
 \[ (\sqrt{b})^n \ge 1 + n (\sqrt{b} - 1) \]
Combining the last two inequalities yields the following:
 \[ {x^a \over b^x} \le {X^a \over b^X} \le {1 \over 1 + (x - X)
 (\sqrt{b} - 1)} \]
From this, it follows that $\lim_{x \to \infty} x^a / b^x = 0$ when
$a$ and $x$ are integers.

Now we lift the restriction that $a$ be an integer.  Since the power
function is increasing, $x^a / b^x \le x^{\lceil a \rceil} /
b^x$, so we have  $\lim_{x \to \infty} x^a / b^x = 0$ for real values
of $a$ as well.

To lift the restriction on $x$, let us write $x = x_1 + x_2$ where
$x_1$ is an integer and $0 \le x_2 < 1$.  Then we have \[ {x^a \over
b^x} = {x_1^a \over b^{x_1}} \left( {x_1 + x_2 \over x_1} \right)^a
b^{-x_2} \] If $x > 2$, then $(x_1 + x_2) / x_2 < 1.5$.  Since $x_2
\ge 0, b^{-x_2} \le 1$.  Hence, for all real $x > 2$, we have \[ {x^a
\over b^x} \le 1.5^a {x_1^a \over b^{x_1}}\] From this inequality, it
follows that $\lim_{x \to \infty} x^a / b^x = 0$ for real values of
$x$ as well.
%%%%%
%%%%%
\end{document}
