\documentclass[12pt]{article}
\usepackage{pmmeta}
\pmcanonicalname{CauchyIntegralFormulaInSeveralVariables}
\pmcreated{2013-03-22 15:33:46}
\pmmodified{2013-03-22 15:33:46}
\pmowner{jirka}{4157}
\pmmodifier{jirka}{4157}
\pmtitle{Cauchy integral formula in several variables}
\pmrecord{7}{37466}
\pmprivacy{1}
\pmauthor{jirka}{4157}
\pmtype{Theorem}
\pmcomment{trigger rebuild}
\pmclassification{msc}{32A07}
\pmclassification{msc}{32A10}

\endmetadata

% this is the default PlanetMath preamble.  as your knowledge
% of TeX increases, you will probably want to edit this, but
% it should be fine as is for beginners.

% almost certainly you want these
\usepackage{amssymb}
\usepackage{amsmath}
\usepackage{amsfonts}

% used for TeXing text within eps files
%\usepackage{psfrag}
% need this for including graphics (\includegraphics)
%\usepackage{graphicx}
% for neatly defining theorems and propositions
\usepackage{amsthm}
% making logically defined graphics
%%%\usepackage{xypic}

% there are many more packages, add them here as you need them

% define commands here
\theoremstyle{theorem}
\newtheorem*{thm}{Theorem}
\newtheorem*{lemma}{Lemma}
\newtheorem*{conj}{Conjecture}
\newtheorem*{cor}{Corollary}
\newtheorem*{example}{Example}
\newtheorem*{prop}{Proposition}
\theoremstyle{definition}
\newtheorem*{defn}{Definition}
\theoremstyle{remark}
\newtheorem*{rmk}{Remark}
\begin{document}
Let $D = D_1 \times \ldots \times D_n \subset {\mathbb{C}}^n$ be a polydisc.

\begin{thm}
Let $f$ be a function continuous in $\bar{D}$ (the closure of $D$).  Then
$f$ is
\PMlinkname{holomorphic}{HolomorphicFunctionsOfSeveralVariables} in $D$
if and only if
for all $z = (z_1,\ldots,z_n) \in D$ we have
\begin{equation*}
f(z_1,\ldots,z_n)
=
\frac{1}{{(2 \pi i)}^n}
\int_{\partial D_1}
\cdots
\int_{\partial D_n}
\frac{f(\zeta_1,\ldots,\zeta_n)}
{(\zeta_1 - z_1)
\ldots
(\zeta_n - z_n)}
d\zeta_1 \ldots d\zeta_n .
\end{equation*}
\end{thm}

As in the case of one variable this theorem can be in fact used as a
definition of holomorphicity.  Note that when $n > 1$ then we are no longer
integrating over the \PMlinkescapetext{entire} boundary of the polydisc but over the
distinguished boundary, that is over $\partial D_1 \times \ldots \times
\partial D_n$.

\begin{thebibliography}{9}
\bibitem{Hormander:several}
Lars H\"ormander.
{\em \PMlinkescapetext{An Introduction to Complex Analysis in Several
Variables}},
North-Holland Publishing Company, New York, New York, 1973.
\bibitem{Krantz:several}
Steven~G.\@ Krantz.
{\em \PMlinkescapetext{Function Theory of Several Complex Variables}},
AMS Chelsea Publishing, Providence, Rhode Island, 1992.
\end{thebibliography}
%%%%%
%%%%%
\end{document}
