\documentclass[12pt]{article}
\usepackage{pmmeta}
\pmcanonicalname{LogarithmicallyConvexSet}
\pmcreated{2013-03-22 14:29:32}
\pmmodified{2013-03-22 14:29:32}
\pmowner{jirka}{4157}
\pmmodifier{jirka}{4157}
\pmtitle{logarithmically convex set}
\pmrecord{5}{36027}
\pmprivacy{1}
\pmauthor{jirka}{4157}
\pmtype{Definition}
\pmcomment{trigger rebuild}
\pmclassification{msc}{32A07}

\endmetadata

% this is the default PlanetMath preamble.  as your knowledge
% of TeX increases, you will probably want to edit this, but
% it should be fine as is for beginners.

% almost certainly you want these
\usepackage{amssymb}
\usepackage{amsmath}
\usepackage{amsfonts}

% used for TeXing text within eps files
%\usepackage{psfrag}
% need this for including graphics (\includegraphics)
%\usepackage{graphicx}
% for neatly defining theorems and propositions
\usepackage{amsthm}
% making logically defined graphics
%%%\usepackage{xypic}

% there are many more packages, add them here as you need them

% define commands here
\theoremstyle{theorem}
\newtheorem*{thm}{Theorem}
\newtheorem*{lemma}{Lemma}
\newtheorem*{conj}{Conjecture}
\newtheorem*{cor}{Corollary}
\newtheorem*{example}{Example}
\theoremstyle{definition}
\newtheorem*{defn}{Definition}
\begin{document}
Suppose $G \subset {\mathbb{C}}^n$, then
we define
\begin{equation*}
\log \lVert G \rVert :=
 \{ (\log \lvert z_1 \rvert ,\ldots,
 \log \lvert z_n \rvert) \in {\mathbb{R}}^n
\mid
 (z_1,\ldots,z_n) \in G \} .
\end{equation*}

\begin{defn}
We say $G \subset {\mathbb{C}}^n$ is a {\em logarithmically convex set}
if $\log \lVert G \rVert \subset {\mathbb{R}}^n$ is a convex set.
\end{defn}

\begin{thebibliography}{9}
\bibitem{Krantz:several}
Steven~G.\@ Krantz.
{\em \PMlinkescapetext{Function Theory of Several Complex Variables}},
AMS Chelsea Publishing, Providence, Rhode Island, 1992.
\end{thebibliography}
%%%%%
%%%%%
\end{document}
