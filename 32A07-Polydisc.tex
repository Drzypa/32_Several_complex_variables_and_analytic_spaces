\documentclass[12pt]{article}
\usepackage{pmmeta}
\pmcanonicalname{Polydisc}
\pmcreated{2013-03-22 14:29:41}
\pmmodified{2013-03-22 14:29:41}
\pmowner{jirka}{4157}
\pmmodifier{jirka}{4157}
\pmtitle{polydisc}
\pmrecord{9}{36030}
\pmprivacy{1}
\pmauthor{jirka}{4157}
\pmtype{Definition}
\pmcomment{trigger rebuild}
\pmclassification{msc}{32A07}
\pmclassification{msc}{32-00}
\pmsynonym{open polydisc}{Polydisc}
\pmdefines{bidisc}
\pmdefines{distinguished boundary}

% this is the default PlanetMath preamble.  as your knowledge
% of TeX increases, you will probably want to edit this, but
% it should be fine as is for beginners.

% almost certainly you want these
\usepackage{amssymb}
\usepackage{amsmath}
\usepackage{amsfonts}

% used for TeXing text within eps files
%\usepackage{psfrag}
% need this for including graphics (\includegraphics)
%\usepackage{graphicx}
% for neatly defining theorems and propositions
\usepackage{amsthm}
% making logically defined graphics
%%%\usepackage{xypic}

% there are many more packages, add them here as you need them

% define commands here
\theoremstyle{theorem}
\newtheorem*{thm}{Theorem}
\newtheorem*{lemma}{Lemma}
\newtheorem*{conj}{Conjecture}
\newtheorem*{cor}{Corollary}
\newtheorem*{example}{Example}
\newtheorem*{prop}{Proposition}
\theoremstyle{definition}
\newtheorem*{defn}{Definition}
\begin{document}
\begin{defn}
We denote the set
\begin{equation*}
D^n(z,r) := \{ w \in {\mathbb{C}}^n \mid \lvert z_k - w_k \rvert < r
\text{ for all } k = 1,\ldots,n \}
\end{equation*}
an {\em open polydisc}.  We can also have {\em polydiscs} of the form
\begin{equation*}
D^1(z_1,r_1) \times \ldots \times D^1(z_n,r_n) .
\end{equation*}
The set $\partial D^1(z_1,r_1) \times \ldots \times \partial D^1(z_n,r_n)$ is called the {\em distinguished boundary} of the polydisc.
\end{defn}

Be careful not to confuse this with the open ball in ${\mathbb{C}}^n$
as that is defined as
\begin{equation*}
B(z,r) := \{ w \in {\mathbb{C}}^n \mid \lvert z - w \rvert < r \} .
\end{equation*}
When $n > 1$ then open balls and open polydiscs are not biholomorphically
equivalent (there is no 1-1 biholomorphic mapping between the two).

It is common to write $\bar{D}^n(z,r)$ for the closure of the polydisc.
Be careful with this notation however as some texts outside of
complex analysis use $D(x,r)$ and
the \PMlinkescapetext{term} ``disc'' to represent a closed ball in two real dimensions.

Also note that when $n=2$ the \PMlinkescapetext{term} {\em bidisc} is sometimes used.

\begin{thebibliography}{9}
\bibitem{Hormander:several}
Lars H\"ormander.
{\em \PMlinkescapetext{An Introduction to Complex Analysis in Several
Variables}},
North-Holland Publishing Company, New York, New York, 1973.
\bibitem{Krantz:several}
Steven~G.\@ Krantz.
{\em \PMlinkescapetext{Function Theory of Several Complex Variables}},
AMS Chelsea Publishing, Providence, Rhode Island, 1992.
\end{thebibliography}
%%%%%
%%%%%
\end{document}
