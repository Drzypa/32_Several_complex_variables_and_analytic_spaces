\documentclass[12pt]{article}
\usepackage{pmmeta}
\pmcanonicalname{ReinhardtDomain}
\pmcreated{2013-03-22 14:29:37}
\pmmodified{2013-03-22 14:29:37}
\pmowner{jirka}{4157}
\pmmodifier{jirka}{4157}
\pmtitle{Reinhardt domain}
\pmrecord{7}{36029}
\pmprivacy{1}
\pmauthor{jirka}{4157}
\pmtype{Definition}
\pmcomment{trigger rebuild}
\pmclassification{msc}{32A07}

\endmetadata

% this is the default PlanetMath preamble.  as your knowledge
% of TeX increases, you will probably want to edit this, but
% it should be fine as is for beginners.

% almost certainly you want these
\usepackage{amssymb}
\usepackage{amsmath}
\usepackage{amsfonts}

% used for TeXing text within eps files
%\usepackage{psfrag}
% need this for including graphics (\includegraphics)
%\usepackage{graphicx}
% for neatly defining theorems and propositions
\usepackage{amsthm}
% making logically defined graphics
%%%\usepackage{xypic}

% there are many more packages, add them here as you need them

% define commands here
\theoremstyle{theorem}
\newtheorem*{thm}{Theorem}
\newtheorem*{lemma}{Lemma}
\newtheorem*{conj}{Conjecture}
\newtheorem*{cor}{Corollary}
\newtheorem*{example}{Example}
\newtheorem*{prop}{Proposition}
\theoremstyle{definition}
\newtheorem*{defn}{Definition}
\begin{document}
\begin{defn}
We call an open set $G \subset {\mathbb{C}}^n$ a {\em Reinhardt domain}
if $(z_1,\ldots,z_n) \in G$ implies that 
$(e^{i\theta_1}z_1,\ldots,e^{i\theta_n}z_n) \in G$ for all real
$\theta_1,\ldots,\theta_n$.
\end{defn}

The reason for studying these kinds of domains is that
\PMlinkname{logarithmically convex}{LogarithmicallyConvexSet}
Reinhardt domain are the domains of convergence of power series in
several complex variables.  Note that in one complex variable, a 
\PMlinkescapetext{logarithmically convex}
Reinhardt domain is just a disc.

Note that the intersection of
\PMlinkescapetext{logarithmically convex}
Reinhardt domains is still a
\PMlinkescapetext{logarithmically convex}
Reinhardt domain, so for every Reinhardt domain, there is a smallest
\PMlinkescapetext{logarithmically convex}
Reinhardt domain which contains it.

\begin{thm}
Suppose that $G$ is a Reinhardt domain which contains 0 and
that $\tilde{G}$ is the smallest 
\PMlinkescapetext{logarithmically convex}
Reinhardt domain such that $G \subset \tilde{G}$.  Then
any function holomorphic on $G$ has a holomorphic \PMlinkescapetext{extension}
to $\tilde{G}$.
\end{thm}

It actually turns out that a 
\PMlinkescapetext{logarithmically convex}
Reinhardt domain is a domain of convergence.

\PMlinkescapetext{Simple} examples of 
\PMlinkescapetext{logarithmically convex}
Reinhardt domains in ${\mathbb{C}}^n$ are polydiscs such as
$\underbrace{{\mathbb{D}} \times \cdots \times {\mathbb{D}}}_n$
where ${\mathbb{D}} \subset {\mathbb{C}}$ is the unit disc.

\begin{thebibliography}{9}
\bibitem{Hormander:several}
Lars H\"ormander.
{\em \PMlinkescapetext{An Introduction to Complex Analysis in Several
Variables}},
North-Holland Publishing Company, New York, New York, 1973.
\bibitem{Krantz:several}
Steven~G.\@ Krantz.
{\em \PMlinkescapetext{Function Theory of Several Complex Variables}},
AMS Chelsea Publishing, Providence, Rhode Island, 1992.
\end{thebibliography}
%%%%%
%%%%%
\end{document}
