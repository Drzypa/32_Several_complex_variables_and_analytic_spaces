\documentclass[12pt]{article}
\usepackage{pmmeta}
\pmcanonicalname{TubeDomain}
\pmcreated{2013-03-22 14:32:44}
\pmmodified{2013-03-22 14:32:44}
\pmowner{jirka}{4157}
\pmmodifier{jirka}{4157}
\pmtitle{tube domain}
\pmrecord{6}{36093}
\pmprivacy{1}
\pmauthor{jirka}{4157}
\pmtype{Definition}
\pmcomment{trigger rebuild}
\pmclassification{msc}{32A07}

% this is the default PlanetMath preamble.  as your knowledge
% of TeX increases, you will probably want to edit this, but
% it should be fine as is for beginners.

% almost certainly you want these
\usepackage{amssymb}
\usepackage{amsmath}
\usepackage{amsfonts}

% used for TeXing text within eps files
%\usepackage{psfrag}
% need this for including graphics (\includegraphics)
%\usepackage{graphicx}
% for neatly defining theorems and propositions
\usepackage{amsthm}
% making logically defined graphics
%%%\usepackage{xypic}

% there are many more packages, add them here as you need them

% define commands here
\theoremstyle{theorem}
\newtheorem*{thm}{Theorem}
\newtheorem*{lemma}{Lemma}
\newtheorem*{conj}{Conjecture}
\newtheorem*{cor}{Corollary}
\newtheorem*{example}{Example}
\newtheorem*{prop}{Proposition}
\theoremstyle{definition}
\newtheorem*{defn}{Definition}
\theoremstyle{remark}
\newtheorem*{rmk}{Remark}
\begin{document}
\begin{defn}
Suppose $S \subset {\mathbb{R}}^n$ is an open set (usually connected),
then we define 
\begin{equation*}
T_S := \{ z \in {\mathbb{C}}^n \mid \operatorname{Re} z \in S \} .
\end{equation*}
We call $T_S$ the {\em tube domain} associated to $S$.
\end{defn}

Basically the idea is that these domains give you complete freedom in the
imaginary coordinates.  An interesting fact about these domains is that
$T_S$ is pseudoconvex if and only if $T_S$ is convex in the geometric sense,
and this is true if and only if $S$ itself is convex.

\begin{thebibliography}{9}
\bibitem{Krantz:several}
Steven~G.\@ Krantz.
{\em \PMlinkescapetext{Function Theory of Several Complex Variables}},
AMS Chelsea Publishing, Providence, Rhode Island, 1992.
\end{thebibliography}
%%%%%
%%%%%
\end{document}
