\documentclass[12pt]{article}
\usepackage{pmmeta}
\pmcanonicalname{HolomorphicFunctionsOfSeveralVariables}
\pmcreated{2013-03-22 15:33:42}
\pmmodified{2013-03-22 15:33:42}
\pmowner{jirka}{4157}
\pmmodifier{jirka}{4157}
\pmtitle{holomorphic functions of several variables}
\pmrecord{7}{37465}
\pmprivacy{1}
\pmauthor{jirka}{4157}
\pmtype{Definition}
\pmcomment{trigger rebuild}
\pmclassification{msc}{32A10}

% this is the default PlanetMath preamble.  as your knowledge
% of TeX increases, you will probably want to edit this, but
% it should be fine as is for beginners.

% almost certainly you want these
\usepackage{amssymb}
\usepackage{amsmath}
\usepackage{amsfonts}

% used for TeXing text within eps files
%\usepackage{psfrag}
% need this for including graphics (\includegraphics)
%\usepackage{graphicx}
% for neatly defining theorems and propositions
\usepackage{amsthm}
% making logically defined graphics
%%%\usepackage{xypic}

% there are many more packages, add them here as you need them

% define commands here
\theoremstyle{theorem}
\newtheorem*{thm}{Theorem}
\newtheorem*{lemma}{Lemma}
\newtheorem*{conj}{Conjecture}
\newtheorem*{cor}{Corollary}
\newtheorem*{example}{Example}
\newtheorem*{prop}{Proposition}
\theoremstyle{definition}
\newtheorem*{defn}{Definition}
\theoremstyle{remark}
\newtheorem*{rmk}{Remark}
\begin{document}
\begin{defn}
Let $\Omega \subset {\mathbb{C}}^n$ be a domain and let $f \colon \Omega \to
{\mathbb{C}}$ be a function.  $f$ is called {\em \PMlinkescapetext{holomorphic}} if it is
\PMlinkname{holomorphic}{Holomorphic} in each variable separately as a function of one variable.
\end{defn}

That means that the function $z_k \mapsto f(z_1,\ldots,z_k,\ldots,z_n)$
is holomorphic as a function of one variable.  It is not at all obvious that
such a function is even continuous and we must apply the Hartogs's theorem on separate
analyticity which is not a trivial result.

Historically and some authors today still continue to do so, the definition
of being holomorphic in several variables did include the continuity or at
least local boundedness requirement.

Of course we can also characterize holomorphic functions by their power series.

\begin{prop}
$f$ is holomorphic in $\Omega$ if and only if near each point $\zeta \in \Omega$ there is a neighbourhood
$U$ and a power series in several variables
\begin{equation*}
\sum_{\alpha} a_\alpha (z-\zeta)^\alpha ,
\end{equation*}
where $\alpha$ ranges over all the
multi-indices, $a_\alpha \in {\mathbb{C}}$ and such that the series converges to $f(z)$ for $z \in U$.
\end{prop}

Another way to characterize holomorphic functions is by the use of the
Cauchy-Riemann equations, which can be given in a very \PMlinkescapetext{simple} form by the
\PMlinkname{$\bar{\partial}$-operator}{BarpartialOperator}.

\begin{prop}
$f$ is holomorphic if and only if $\bar{\partial} f = 0$.
\end{prop}

Despite the similarities,
one should be careful about carelessly generalizing results about functions
of one variable to functions of several variables as the theory is quite
different.  See the topic entry on several complex variables for more
\PMlinkescapetext{information}.

\begin{thebibliography}{9}
\bibitem{Hormander:several}
Lars H\"ormander.
{\em \PMlinkescapetext{An Introduction to Complex Analysis in Several
Variables}},
North-Holland Publishing Company, New York, New York, 1973.
\bibitem{Krantz:several}
Steven~G.\@ Krantz.
{\em \PMlinkescapetext{Function Theory of Several Complex Variables}},
AMS Chelsea Publishing, Providence, Rhode Island, 1992.
\end{thebibliography}
%%%%%
%%%%%
\end{document}
