\documentclass[12pt]{article}
\usepackage{pmmeta}
\pmcanonicalname{SpaceOfAnalyticFunctions}
\pmcreated{2013-03-22 14:24:48}
\pmmodified{2013-03-22 14:24:48}
\pmowner{jirka}{4157}
\pmmodifier{jirka}{4157}
\pmtitle{space of analytic functions}
\pmrecord{7}{35918}
\pmprivacy{1}
\pmauthor{jirka}{4157}
\pmtype{Definition}
\pmcomment{trigger rebuild}
\pmclassification{msc}{32A10}
\pmclassification{msc}{30D20}
\pmsynonym{space of holomorphic functions}{SpaceOfAnalyticFunctions}
\pmrelated{Holomorphic}
\pmrelated{Meromorphic}
\pmrelated{MontelsTheorem}
\pmdefines{space of meromorphic functions}

% this is the default PlanetMath preamble.  as your knowledge
% of TeX increases, you will probably want to edit this, but
% it should be fine as is for beginners.

% almost certainly you want these
\usepackage{amssymb}
\usepackage{amsmath}
\usepackage{amsfonts}

% used for TeXing text within eps files
%\usepackage{psfrag}
% need this for including graphics (\includegraphics)
%\usepackage{graphicx}
% for neatly defining theorems and propositions
\usepackage{amsthm}
% making logically defined graphics
%%%\usepackage{xypic}

% there are many more packages, add them here as you need them

% define commands here
\theoremstyle{theorem}
\newtheorem*{thm}{Theorem}
\newtheorem*{lemma}{Lemma}
\newtheorem*{conj}{Conjecture}
\newtheorem*{cor}{Corollary}
\newtheorem*{example}{Example}
\newtheorem*{prop}{Proposition}
\theoremstyle{definition}
\newtheorem*{defn}{Definition}
\theoremstyle{remark}
\newtheorem*{rmk}{Remark}
\begin{document}
For what follows suppose that $G \subset {\mathbb{C}}$ is a region.  We wish to take the set of all holomorphic functions on $G$, denoted by ${\mathcal{O}}(G)$, and make it into a metric space.  We will define a metric such that convergence in this metric is the same as uniform convergence on compact subsets of $G$.  We will call this the {\em space of analytic functions} on $G$.

It is known that there
exists a sequence of compact subsets $K_n \subset G$ such that $K_n \subset
K_{n+1}^\circ$ (interior of $K_{n+1}$), such that $\bigcup K_n^\circ = G$ and
such that if $K$ is any compact subset of $G$, then $K \subset K_n$ for some $n$.
Now define the quantity $\rho_n(f,g)$ for $f,g \in {\mathcal{O}}(G)$ as
\begin{equation*}
\rho_n(f,g) := \sup_{z \in K_n} \{ \lvert f(z) - g(z) \rvert \} .
\end{equation*}
We define the metric on ${\mathcal{O}}(G)$ as
\begin{equation*}
d(f,g) := \sum_{n=1}^\infty \left(\frac{1}{2}\right)^n
\frac{\rho_n(f,g)}{1+\rho_n(f,g)} .
\end{equation*}
This can be shown to be a metric.  Furthermore, it can be shown that the topology generated by this metric is independent of the choice of $K_n$, even though
the actual values of the metric do depend on the particular $K_n$ we have chosen.
Finally, it can be shown that convergence in $d$ is the same as uniform convergence on compact subsets.  It is known that if you have a sequence of
analytic functions on $G$ that converge uniformly on compact subsets, then the limit is in fact analytic in $G$, and thus ${\mathcal{O}}(G)$ is a complete space. 

Similarly, we can treat the functions that are meromorphic on $G$, and define
$M(G)$ to be the {\em space of meromorphic functions} on $G$.  We assume that the
functions take the value $\infty$ at their poles, so that they are defined at
every point of $G$.  That is, they take their values in the Riemann sphere, or the extended complex plane.  We just need to replace
the definition of $\rho_n(f,g)$ with
\begin{equation*}
\rho_n(f,g) := \sup_{z \in K_n} \{ \sigma (f(z), g(z)) \} ,
\end{equation*}
where $\sigma$ is either the spherical metric on the Riemann sphere, or alternatively the metric induced by embedding the Riemann sphere in ${\mathbb{R}}^3$.  Both of those metrics produce the same topology, and that is all that we care about.  The rest of the definition is the same as that of ${\mathcal{O}}(G)$.  There is, however, one small glitch here.  $M(G)$ is not a complete metric space.  It is possible that functions in $M(G)$ go off to infinity pointwise, but this is the worst that can happen.  For example, the sequence $f_n(z) = n$ is a sequence of meromorphic functions on $G$, and
this sequence is Cauchy in $M(G)$, but the limit would be $f(z) = \infty$ and
that is not a function in $M(G)$.

\begin{rmk}
Note that ${\mathcal{O}}(G)$ is sometimes denoted by $H(G)$ in literature.  Also note that $A(G)$ is usually reserved for functions which are
analytic on $G$ and continuous on $\bar{G}$ (closure of $G$).
\end{rmk}

\begin{rmk}
We can similarly define the space of continuous functions, and treat ${\mathcal{O}}(G)$ and $M(G)$ as subspaces of that.  That is, ${\mathcal{O}}(G)$
would be a subspace of $C(G,{\mathbb{C}})$ and $M(G)$ would be a subspace
of $C(G,\hat{\mathbb{C}})$.
\end{rmk}

\begin{thebibliography}{9}
\bibitem{Conway:complexI}
John~B. Conway.
{\em \PMlinkescapetext{Functions of One Complex Variable I}}.
Springer-Verlag, New York, New York, 1978.
\end{thebibliography}
%%%%%
%%%%%
\end{document}
