\documentclass[12pt]{article}
\usepackage{pmmeta}
\pmcanonicalname{Multifunction}
\pmcreated{2013-03-22 17:42:08}
\pmmodified{2013-03-22 17:42:08}
\pmowner{jirka}{4157}
\pmmodifier{jirka}{4157}
\pmtitle{multifunction}
\pmrecord{4}{40144}
\pmprivacy{1}
\pmauthor{jirka}{4157}
\pmtype{Definition}
\pmcomment{trigger rebuild}
\pmclassification{msc}{32A12}
\pmsynonym{m-function}{Multifunction}
\pmrelated{SymmetricPower}
\pmrelated{WeierstrassPolynomial}
\pmrelated{MultivaluedFunction}
\pmdefines{multigraph}
\pmdefines{multiple valued function}

\endmetadata

% this is the default PlanetMath preamble.  as your knowledge
% of TeX increases, you will probably want to edit this, but
% it should be fine as is for beginners.

% almost certainly you want these
\usepackage{amssymb}
\usepackage{amsmath}
\usepackage{amsfonts}

% used for TeXing text within eps files
%\usepackage{psfrag}
% need this for including graphics (\includegraphics)
%\usepackage{graphicx}
% for neatly defining theorems and propositions
\usepackage{amsthm}
% making logically defined graphics
%%%\usepackage{xypic}

% there are many more packages, add them here as you need them

% define commands here
\theoremstyle{theorem}
\newtheorem*{thm}{Theorem}
\newtheorem*{lemma}{Lemma}
\newtheorem*{conj}{Conjecture}
\newtheorem*{cor}{Corollary}
\newtheorem*{example}{Example}
\newtheorem*{prop}{Proposition}
\theoremstyle{definition}
\newtheorem*{defn}{Definition}
\theoremstyle{remark}
\newtheorem*{rmk}{Remark}

\begin{document}
It is common practice among complex analysts to speak of {\em multiple valued functions} in contexts of ``functions'' such as $\sqrt{z}.$  This somewhat informal notion can be made very precise when the ``function'' has finitely many values (as the $\sqrt{z}$ does).

Let $X$ and $Y$ be sets and denote by
$Y^m_{sym}$ the $m^{\text{th}}$ symmetric power of $Y.$

\begin{defn}
A function $f \colon X \to Y^m_{sym}$ is called a {\em multifunction},
or an $m$-function from $X$ to $Y$, where $m$ is the multiplicity.
\end{defn}

We can think of the value of $f$ at any point as a set of $m$ (or fewer) elements.
Let $Y$ be a topological space (resp. ${\mathbb{C}}$)
A multifunction is said to be continuous (resp. holomorphic) if all the elementary symmetric polynomials of
the elements of $f$ are continuous (resp. holomorphic).  Equivalently, $f$ is continuous (resp. holomorphic)
if it is continuous (resp. holomorphic) as functions to $Y^m_{sym} \cong Y^m$
(resp. ${\mathbb{C}}^m_{sym} \cong {\mathbb{C}}^m$).

With this definition $\sqrt{z}$ is a holomorphic multifunction (or a 2-function), into ${\mathbb{C}}^2_{sym} .$

Define the {\em multigraph} of $f$ to be the set:
\begin{equation*}
\{ (x,y) \mid X \times Y \mid y \in f(x) \} .
\end{equation*}

The multigraph of $\sqrt{z}$ is the corresponding Riemann surface imbedded in ${\mathbb{C}}^2.$  In general, with the  aid of the Weierstrass preparation theorem we can realize any codimension 1 analytic set in ${\mathbb{C}}^n$ as a multigraph over ${\mathbb{C}}^{n-1} .$
The roots of any Weierstrass polynomial (or in general of any monic polynomial with holomorphic coefficients) are a holomorphic multifunction.


\begin{thebibliography}{9}
\bibitem{Whitney:varieties}
Hassler Whitney.
{\em \PMlinkescapetext{Complex Analytic Varieties}}.
Addison-Wesley, Philippines, 1972.
\end{thebibliography}
%%%%%
%%%%%
\end{document}
