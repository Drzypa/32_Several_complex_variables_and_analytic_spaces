\documentclass[12pt]{article}
\usepackage{pmmeta}
\pmcanonicalname{SymmetricPower}
\pmcreated{2013-03-22 17:42:05}
\pmmodified{2013-03-22 17:42:05}
\pmowner{jirka}{4157}
\pmmodifier{jirka}{4157}
\pmtitle{symmetric power}
\pmrecord{5}{40143}
\pmprivacy{1}
\pmauthor{jirka}{4157}
\pmtype{Definition}
\pmcomment{trigger rebuild}
\pmclassification{msc}{32A12}
\pmclassification{msc}{05E05}
\pmrelated{Multifunction}

% this is the default PlanetMath preamble.  as your knowledge
% of TeX increases, you will probably want to edit this, but
% it should be fine as is for beginners.

% almost certainly you want these
\usepackage{amssymb}
\usepackage{amsmath}
\usepackage{amsfonts}

% used for TeXing text within eps files
%\usepackage{psfrag}
% need this for including graphics (\includegraphics)
%\usepackage{graphicx}
% for neatly defining theorems and propositions
\usepackage{amsthm}
% making logically defined graphics
%%%\usepackage{xypic}

% there are many more packages, add them here as you need them

% define commands here
\theoremstyle{theorem}
\newtheorem*{thm}{Theorem}
\newtheorem*{lemma}{Lemma}
\newtheorem*{conj}{Conjecture}
\newtheorem*{cor}{Corollary}
\newtheorem*{example}{Example}
\newtheorem*{prop}{Proposition}
\theoremstyle{definition}
\newtheorem*{defn}{Definition}
\theoremstyle{remark}
\newtheorem*{rmk}{Remark}

\begin{document}
Let $X$ be a set and let 
\begin{equation*}
X^m := \underbrace{X \times \cdots \times X}_{m-\text{times}} .
\end{equation*}
Denote an element of $X^m$ by $x = (x_1,\ldots,x_m).$
 Define an equivalence relation
by $x \sim x'$ if and only if there exists a 
permutation $\sigma$ of $(1,\ldots,m),$ such that
$x_i = x'_{\sigma{i}}$.

\begin{defn}
The $m^{\text{th}}$ symmetric power of $X$ is
the set $X^m_{sym} := X^m / \sim.$  That is, the set of equivalence classes of $X^m$ under the
relation $\sim.$
\end{defn}

Let $\pi$ be the natural projection of $X^m$ onto $X^m_{sym}$.

\begin{prop}
$f \colon X^m \to Y$ is a symmetric function if and only if there exists a function
$g \colon X^m_{sym} \to Y$ such that $f = g \circ \pi.$
\end{prop}

From now on let $R$ be an integral domain.  Let $\tau' \colon X^m \to X^m$ be the map
$\tau'(x) := (\tau_1(x),\ldots,\tau_m(x)),$ where $\tau_k$ is the $k^\text{th}$ elementary symmetric
polynomial.  By the above lemma, we have a function $\tau \colon X^m_{sym} \to X^m$, where
$\tau' = \tau \circ \pi .$

\begin{prop}
$\tau$ is one to one.  If $R$ is algebraically closed, then $\tau$ is onto.
\end{prop}

% FIXME: topology and other structure.

A very useful case is when $R = \mathbb{C}.$  In this case, when we put on the natural complex manifold structure
onto ${\mathbb{C}}^m_{sym},$ the map $\tau$ is a biholomorphism of ${\mathbb{C}}^m_{sym}$ and
${\mathbb{C}}^m .$

\begin{thebibliography}{9}
\bibitem{Whitney:varieties}
Hassler Whitney.
{\em \PMlinkescapetext{Complex Analytic Varieties}}.
Addison-Wesley, Philippines, 1972.
\end{thebibliography}
%%%%%
%%%%%
\end{document}
