\documentclass[12pt]{article}
\usepackage{pmmeta}
\pmcanonicalname{WeierstrassPolynomial}
\pmcreated{2013-03-22 15:04:25}
\pmmodified{2013-03-22 15:04:25}
\pmowner{jirka}{4157}
\pmmodifier{jirka}{4157}
\pmtitle{Weierstrass polynomial}
\pmrecord{7}{36795}
\pmprivacy{1}
\pmauthor{jirka}{4157}
\pmtype{Definition}
\pmcomment{trigger rebuild}
\pmclassification{msc}{32A17}
\pmclassification{msc}{32B05}
\pmsynonym{W-polynomial}{WeierstrassPolynomial}
\pmrelated{Multifunction}
\pmrelated{WeierstrassPreparationTheorem}

\endmetadata

% this is the default PlanetMath preamble.  as your knowledge
% of TeX increases, you will probably want to edit this, but
% it should be fine as is for beginners.

% almost certainly you want these
\usepackage{amssymb}
\usepackage{amsmath}
\usepackage{amsfonts}

% used for TeXing text within eps files
%\usepackage{psfrag}
% need this for including graphics (\includegraphics)
%\usepackage{graphicx}
% for neatly defining theorems and propositions
\usepackage{amsthm}
% making logically defined graphics
%%%\usepackage{xypic}

% there are many more packages, add them here as you need them

% define commands here
\theoremstyle{theorem}
\newtheorem*{thm}{Theorem}
\newtheorem*{lemma}{Lemma}
\newtheorem*{conj}{Conjecture}
\newtheorem*{cor}{Corollary}
\newtheorem*{example}{Example}
\newtheorem*{prop}{Proposition}
\theoremstyle{definition}
\newtheorem*{defn}{Definition}
\theoremstyle{remark}
\newtheorem*{rmk}{Remark}
\begin{document}
\begin{defn}
A function $W\colon {\mathbb{C}}^n \to {\mathbb{C}}$ of the form
\begin{equation*}
W(z_1,\ldots,z_n) = z_n^m + \sum_{j=1}^{m-1}a_j(z_1,\ldots,z_{n-1})z_n^j ,
\end{equation*}
where the $a_j$ are holomorphic functions in a neighbourhood of the origin, which vanish at the origin,
is called a {\em Weierstrass polynomial}.
\end{defn}

Any codimension 1 complex analytic subvariety of ${\mathbb{C}}^n$ can be written as the zero set
of a Weierstrass polynomial using the Weierstrass preparation theorem.  This in general cannot be done for higher codimension.

\begin{thebibliography}{9}
\bibitem{Hormander:several}
Lars H\"ormander.
{\em \PMlinkescapetext{An Introduction to Complex Analysis in Several
Variables}},
North-Holland Publishing Company, New York, New York, 1973.
\bibitem{Krantz:several}
Steven~G.\@ Krantz.
{\em \PMlinkescapetext{Function Theory of Several Complex Variables}},
AMS Chelsea Publishing, Providence, Rhode Island, 1992.
\end{thebibliography}
%%%%%
%%%%%
\end{document}
