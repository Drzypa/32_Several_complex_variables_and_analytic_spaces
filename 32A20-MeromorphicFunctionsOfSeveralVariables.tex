\documentclass[12pt]{article}
\usepackage{pmmeta}
\pmcanonicalname{MeromorphicFunctionsOfSeveralVariables}
\pmcreated{2013-03-22 16:01:10}
\pmmodified{2013-03-22 16:01:10}
\pmowner{jirka}{4157}
\pmmodifier{jirka}{4157}
\pmtitle{meromorphic functions of several variables}
\pmrecord{4}{38058}
\pmprivacy{1}
\pmauthor{jirka}{4157}
\pmtype{Definition}
\pmcomment{trigger rebuild}
\pmclassification{msc}{32A20}
\pmdefines{indeterminancy set}

\endmetadata

% this is the default PlanetMath preamble.  as your knowledge
% of TeX increases, you will probably want to edit this, but
% it should be fine as is for beginners.

% almost certainly you want these
\usepackage{amssymb}
\usepackage{amsmath}
\usepackage{amsfonts}

% used for TeXing text within eps files
%\usepackage{psfrag}
% need this for including graphics (\includegraphics)
%\usepackage{graphicx}
% for neatly defining theorems and propositions
\usepackage{amsthm}
% making logically defined graphics
%%%\usepackage{xypic}

% there are many more packages, add them here as you need them

% define commands here
\theoremstyle{theorem}
\newtheorem*{thm}{Theorem}
\newtheorem*{lemma}{Lemma}
\newtheorem*{conj}{Conjecture}
\newtheorem*{cor}{Corollary}
\newtheorem*{example}{Example}
\newtheorem*{prop}{Proposition}
\theoremstyle{definition}
\newtheorem*{defn}{Definition}
\theoremstyle{remark}
\newtheorem*{rmk}{Remark}

\begin{document}
\begin{defn}
Let $\Omega \subset {\mathbb{C}}^n$ be a domain and let $h \colon \Omega \to
{\mathbb{C}}$ be a function. $h$ is called {\em \PMlinkescapetext{meromorphic}} if for each $p \in \Omega$ there exists a neighbourhood $U \subset \Omega$ ($p \in U$) and two 
\PMlinkname{holomorphic}{HolomorphicFunctionsOfSeveralVariables}
functions $f, g$ defined in $U$ where $g$ is not identically zero, such that
$h = f/g$ outside the set where $g = 0$.
\end{defn}

Note that $h$ is really defined only outside of a complex analytic subvariety.  Unlike in one variable, we cannot simply define $h$ to be equal to $\infty$ at the poles and expect $h$ to be a continuous mapping to some larger space (the Riemann sphere in the case of one variable).  The simplest counterexample in ${\mathbb{C}}^2$ is $(z,w) \mapsto z/w$, which does not have a unique limit at the origin.  The set of points where there is no unique limit, is called the {\em indeterminancy set}.  That is, the set of points where if $h = f/g$, and $f$ and $g$ have no common factors, then the indeterminancy set of $h$ is the set where $f = g = 0$.

\begin{thebibliography}{9}
\bibitem{Hormander:several}
Lars H\"ormander.
{\em \PMlinkescapetext{An Introduction to Complex Analysis in Several
Variables}},
North-Holland Publishing Company, New York, New York, 1973.
\bibitem{Krantz:several}
Steven~G.\@ Krantz.
{\em \PMlinkescapetext{Function Theory of Several Complex Variables}},
AMS Chelsea Publishing, Providence, Rhode Island, 1992.
\end{thebibliography}

%%%%%
%%%%%
\end{document}
