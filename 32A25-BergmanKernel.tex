\documentclass[12pt]{article}
\usepackage{pmmeta}
\pmcanonicalname{BergmanKernel}
\pmcreated{2013-03-22 15:04:45}
\pmmodified{2013-03-22 15:04:45}
\pmowner{jirka}{4157}
\pmmodifier{jirka}{4157}
\pmtitle{Bergman kernel}
\pmrecord{7}{36802}
\pmprivacy{1}
\pmauthor{jirka}{4157}
\pmtype{Definition}
\pmcomment{trigger rebuild}
\pmclassification{msc}{32A25}
\pmrelated{BergmanSpace}
\pmrelated{BergmanMetric}

\endmetadata

% this is the default PlanetMath preamble.  as your knowledge
% of TeX increases, you will probably want to edit this, but
% it should be fine as is for beginners.

% almost certainly you want these
\usepackage{amssymb}
\usepackage{amsmath}
\usepackage{amsfonts}

% used for TeXing text within eps files
%\usepackage{psfrag}
% need this for including graphics (\includegraphics)
%\usepackage{graphicx}
% for neatly defining theorems and propositions
\usepackage{amsthm}
% making logically defined graphics
%%%\usepackage{xypic}

% there are many more packages, add them here as you need them

% define commands here
\theoremstyle{theorem}
\newtheorem*{thm}{Theorem}
\newtheorem*{lemma}{Lemma}
\newtheorem*{conj}{Conjecture}
\newtheorem*{cor}{Corollary}
\theoremstyle{definition}
\newtheorem*{defn}{Definition}
\begin{document}
Let $G \subset {\mathbb{C}}^n$ be a \PMlinkname{domain}{Domain2}.  And let $A^2(G)$ be the Bergman space.  For a fixed $z \in G$, the functional $f \mapsto f(z)$ is a bounded
linear functional.  By the Riesz representation theorem (as $A^2(G)$ is a Hilbert space) there exists an element of $A^2(G)$ that represents it, and
let's call that element $k_z \in A^2(G)$.  That is we have that
$f(z) = \langle f, k_z \rangle$.  So we can define the {\em Bergman kernel}.

\begin{defn}
The function 
\begin{equation*}
K(z,w) := \overline{k_z(w)}
\end{equation*}
is called the {\em Bergman kernel}.
\end{defn}

By definition of the inner product in $A^2(G)$ we then have that
for $f \in A^2(G)$
\begin{equation*}
f(z) =
\int_G f(w) K(z,w) dV(w) ,
\end{equation*}
where $dV$ is the volume measure.

As the $A^2(G)$ space is a subspace of $L^2(G,dV)$ which is a separable Hilbert space then $A^2(G)$ also has a countable orthonormal basis, say $\{ \varphi_j \}_{j=1}^\infty$.

\begin{thm}
We can compute the Bergman kernel as
\begin{equation*}
K(z,w) = \sum_{j=1}^\infty \varphi_j(z)\overline{\varphi_j(w)} ,
\end{equation*}
where the sum converges uniformly on compact subsets of $G \times G$.
\end{thm}

Note that integration against the Bergman kernel is just the orthogonal
projection from $L^2(G,dV)$ to $A^2(G)$.  So not only is this kernel reproducing for holomorphic functions, but it will produce a holomorphic function when we just feed in any $L^2(G,dV)$ function.

\begin{thebibliography}{9}
\bibitem{Krantz:several}
Steven~G.\@ Krantz.
{\em \PMlinkescapetext{Function Theory of Several Complex Variables}},
AMS Chelsea Publishing, Providence, Rhode Island, 1992.
\end{thebibliography}
%%%%%
%%%%%
\end{document}
