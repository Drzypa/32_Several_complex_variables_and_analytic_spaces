\documentclass[12pt]{article}
\usepackage{pmmeta}
\pmcanonicalname{BergmanSpace}
\pmcreated{2013-03-22 15:04:43}
\pmmodified{2013-03-22 15:04:43}
\pmowner{jirka}{4157}
\pmmodifier{jirka}{4157}
\pmtitle{Bergman space}
\pmrecord{10}{36801}
\pmprivacy{1}
\pmauthor{jirka}{4157}
\pmtype{Definition}
\pmcomment{trigger rebuild}
\pmclassification{msc}{32A36}
\pmrelated{BergmanKernel}

% this is the default PlanetMath preamble.  as your knowledge
% of TeX increases, you will probably want to edit this, but
% it should be fine as is for beginners.

% almost certainly you want these
\usepackage{amssymb}
\usepackage{amsmath}
\usepackage{amsfonts}

% used for TeXing text within eps files
%\usepackage{psfrag}
% need this for including graphics (\includegraphics)
%\usepackage{graphicx}
% for neatly defining theorems and propositions
\usepackage{amsthm}
% making logically defined graphics
%%%\usepackage{xypic}

% there are many more packages, add them here as you need them

% define commands here
\theoremstyle{theorem}
\newtheorem*{thm}{Theorem}
\newtheorem*{lemma}{Lemma}
\newtheorem*{conj}{Conjecture}
\newtheorem*{cor}{Corollary}
\theoremstyle{definition}
\newtheorem*{defn}{Definition}
\begin{document}
Let $G \subset {\mathbb{C}}^n$ be a domain and let
$dV$ denote the Euclidean volume measure on $G$.

\begin{defn}
Let
\begin{equation*}
A^2(G) :=
\Big\{ f \text{ holomorpic in } G ~\Big|~
\sqrt{ \int_G \lvert f(z) \rvert^2 dV(z) } < \infty \Big\} .
\end{equation*}
$A^2(G)$ is called the {\em Bergman space} on $G$.  The norm on this space
is defined as
\begin{equation*}
\lVert f \rVert :=
\sqrt{ \int_G \lvert f(z) \rvert^2 dV(z) } .
\end{equation*}
Further we define an inner product on $A^2(G)$ as
\begin{equation*}
\langle f , g \rangle :=
\int_G f(z) \overline{g(z)} dV(z) .
\end{equation*}
\end{defn}

The inner product as defined above really is an inner product and further
it can be shown that $A^2(G)$ is complete since convergence in the above norm implies normal convergence (uniform convergence on
compact subsets).  The space $A^2(G)$ is therefore a Hilbert space.
Sometimes this space is also denoted by $L_a^2(G)$.

\begin{thebibliography}{9}
\bibitem{DAngelo}
D'Angelo, John~P.
{\em \PMlinkescapetext{Several complex variables and the geometry of real
hypersurfaces}},
CRC Press, 1993.
\bibitem{Krantz:several}
Steven~G.\@ Krantz.
{\em \PMlinkescapetext{Function Theory of Several Complex Variables}},
AMS Chelsea Publishing, Providence, Rhode Island, 1992.
\end{thebibliography}
%%%%%
%%%%%
\end{document}
