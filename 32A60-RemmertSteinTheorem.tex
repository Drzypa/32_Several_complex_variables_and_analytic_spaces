\documentclass[12pt]{article}
\usepackage{pmmeta}
\pmcanonicalname{RemmertSteinTheorem}
\pmcreated{2013-03-22 15:04:55}
\pmmodified{2013-03-22 15:04:55}
\pmowner{jirka}{4157}
\pmmodifier{jirka}{4157}
\pmtitle{Remmert-Stein theorem}
\pmrecord{9}{36805}
\pmprivacy{1}
\pmauthor{jirka}{4157}
\pmtype{Theorem}
\pmcomment{trigger rebuild}
\pmclassification{msc}{32A60}
\pmclassification{msc}{32C25}
\pmsynonym{Remmert-Stein extension theorem}{RemmertSteinTheorem}
\pmrelated{ChowsTheorem}

% this is the default PlanetMath preamble.  as your knowledge
% of TeX increases, you will probably want to edit this, but
% it should be fine as is for beginners.

% almost certainly you want these
\usepackage{amssymb}
\usepackage{amsmath}
\usepackage{amsfonts}

% used for TeXing text within eps files
%\usepackage{psfrag}
% need this for including graphics (\includegraphics)
%\usepackage{graphicx}
% for neatly defining theorems and propositions
\usepackage{amsthm}
% making logically defined graphics
%%%\usepackage{xypic}

% there are many more packages, add them here as you need them

% define commands here
\theoremstyle{theorem}
\newtheorem*{thm}{Theorem}
\newtheorem*{lemma}{Lemma}
\newtheorem*{conj}{Conjecture}
\newtheorem*{cor}{Corollary}
\newtheorem*{example}{Example}
\newtheorem*{prop}{Proposition}
\theoremstyle{definition}
\newtheorem*{defn}{Definition}
\begin{document}
For a complex analytic subvariety $V$ and $p \in V$ a regular point, let $\dim_p V$ denote the complex dimension of $V$ near the point $p.$

\begin{thm}[Remmert-Stein]
Let $U \subset {\mathbb{C}}^n$ be a \PMlinkname{domain}{Domain2} and let $S$ be a complex analytic subvariety of $U$ of
dimension $m < n.$  Let $V$ be a complex analytic subvariety of $U \backslash S$ such that $\dim_p V > m$ for all
regular points $p \in V.$  Then the closure of $V$ in $U$ is an analytic variety in $U.$
\end{thm}

The condition that $\dim_p V > m$ for all regular $p$ is the same as saying that all the irreducible
components of $V$ are of dimension strictly greater than $m.$  To show that the restriction on the dimension
of $S$ is ``sharp,''
consider the following example where the dimension of $V$ equals the dimension of $S$.
Let $(z,w) \in {\mathbb C}^2$ be our coordinates and let $V$ be defined by $w = e^{1/z}$ in ${\mathbb C}^2 \setminus S,$ where $S$ is defined by $z = 0.$  The closure of $V$ in ${\mathbb C}^2$ cannot possibly be
analytic.  To see this look for example at $W = \overline{V} \cap \{ w = 1 \}.$
If $\overline{V}$ is analytic then $W$ ought to be a zero dimensional
complex analytic set and thus a set of isolated points, but it has a limit point $(0,1)$ by Picard's theorem. 

Finally note that there are various generalizations of this theorem where the set $S$ need not be a variety,
as long as it is of small enough dimension.  Alternatively, if $V$ is of finite volume, we can weaken the
restrictions on $S$ even further.

\begin{thebibliography}{9}
\bibitem{holtoman}
Klaus Fritzsche, Hans Grauert.
{\em \PMlinkescapetext{From Holomorphic Functions to Complex Manifolds}},
Springer-Verlag, New York, New York, 2002.
\bibitem{Whitney:varieties}
Hassler Whitney.
{\em \PMlinkescapetext{Complex Analytic Varieties}}.
Addison-Wesley, Philippines, 1972.
\end{thebibliography}
%%%%%
%%%%%
\end{document}
