\documentclass[12pt]{article}
\usepackage{pmmeta}
\pmcanonicalname{WeierstrassPreparationTheorem}
\pmcreated{2013-03-22 15:04:28}
\pmmodified{2013-03-22 15:04:28}
\pmowner{jirka}{4157}
\pmmodifier{jirka}{4157}
\pmtitle{Weierstrass preparation theorem}
\pmrecord{6}{36796}
\pmprivacy{1}
\pmauthor{jirka}{4157}
\pmtype{Theorem}
\pmcomment{trigger rebuild}
\pmclassification{msc}{32B05}
\pmsynonym{Weierstrass division theorem}{WeierstrassPreparationTheorem}
\pmrelated{WeierstrassPolynomial}

\endmetadata

% this is the default PlanetMath preamble.  as your knowledge
% of TeX increases, you will probably want to edit this, but
% it should be fine as is for beginners.

% almost certainly you want these
\usepackage{amssymb}
\usepackage{amsmath}
\usepackage{amsfonts}

% used for TeXing text within eps files
%\usepackage{psfrag}
% need this for including graphics (\includegraphics)
%\usepackage{graphicx}
% for neatly defining theorems and propositions
\usepackage{amsthm}
% making logically defined graphics
%%%\usepackage{xypic}

% there are many more packages, add them here as you need them

% define commands here
\theoremstyle{theorem}
\newtheorem*{thm}{Theorem}
\newtheorem*{lemma}{Lemma}
\newtheorem*{conj}{Conjecture}
\newtheorem*{cor}{Corollary}
\newtheorem*{example}{Example}
\newtheorem*{prop}{Proposition}
\theoremstyle{definition}
\newtheorem*{defn}{Definition}
\theoremstyle{remark}
\newtheorem*{rmk}{Remark}
\begin{document}
The following theorem is known as the Weierstrass preparation theorem, though sometimes that name is reserved for the corollary and this theorem is then known as the Weierstrass division theorem.

In the following we use the standard notation for coordinates in ${\mathbb{C}}^n$ that $z = (z_1,\ldots,z_n) = (z',z_n)$.  That is $z'$ is the first $n-1$ coordinates.

\begin{thm}
Let $f\colon {\mathbb{C}}^n \to {\mathbb{C}}$ be a function analytic in a neighbourhood $U$ of the origin such that
$\frac{f(0,z_n)}{z_n^m}$ extends to be analytic at the origin and is not zero
at the origin for some positive integer $m$ (in other words, as a function of $z_n$, the function has a zero of order $m$ at the origin).  Then there exists
a polydisc $D \subset U$ such that every function $g$ holomorphic and bounded in $D$ can be written as
\begin{equation*}
g =  qf + r ,
\end{equation*}
where $q$ is an analytic function and $r$ is a polynomial in the $z_n$ variable
of degree less then $m$ with the coefficients being holomorphic functions in $z'$.  Further there exists a constant $C$ independent of $g$ such that
\begin{equation*}
\sup_{z \in D} \lvert q(z) \rvert \leq C \sup_{z \in D} \lvert g(z) \rvert .
\end{equation*}
The representation $g = qf +r$ is unique.
Finally the coefficients of the power series expansions of $q$ and $r$ are
finite linear combinations of the coefficients of the power series of $g$.
\end{thm}

Note that $r$ is not necessarily a Weierstrass polynomial.

\begin{cor}
Let $f$ be as above, then there is a unique representation of $f$ as
$f = h W$, where $h$ is analytic in a neighbourhood of the origin and $h(0) \not= 0$ and $W$ being a Weierstrass polynomial.
\end{cor}

It should be noted that the condition that $\frac{f(0,z_n)}{z_n^m}$ extends to be analytic, which is equivalent to
saying that $f(0,z_n) \not\equiv 0$, is not an essential restriction.
In fact $f(0,z_n) \equiv 0$, then there exists a linear change of coordinates, arbitrarily close to the identity, 
such that the condition of the theorem is satisfied in the new set of coordinates.

\begin{thebibliography}{9}
\bibitem{Hormander:several}
Lars H\"ormander.
{\em \PMlinkescapetext{An Introduction to Complex Analysis in Several
Variables}},
North-Holland Publishing Company, New York, New York, 1973.
\bibitem{Krantz:several}
Steven~G.\@ Krantz.
{\em \PMlinkescapetext{Function Theory of Several Complex Variables}},
AMS Chelsea Publishing, Providence, Rhode Island, 1992.
\end{thebibliography}
%%%%%
%%%%%
\end{document}
