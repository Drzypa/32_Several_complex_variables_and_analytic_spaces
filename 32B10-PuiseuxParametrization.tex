\documentclass[12pt]{article}
\usepackage{pmmeta}
\pmcanonicalname{PuiseuxParametrization}
\pmcreated{2013-03-22 15:20:32}
\pmmodified{2013-03-22 15:20:32}
\pmowner{jirka}{4157}
\pmmodifier{jirka}{4157}
\pmtitle{Puiseux parametrization}
\pmrecord{6}{37162}
\pmprivacy{1}
\pmauthor{jirka}{4157}
\pmtype{Theorem}
\pmcomment{trigger rebuild}
\pmclassification{msc}{32B10}
\pmsynonym{Puiseux series parametrization}{PuiseuxParametrization}
\pmsynonym{Puiseux normalization}{PuiseuxParametrization}
\pmsynonym{Puiseux series normalization}{PuiseuxParametrization}
\pmsynonym{Puiseux parameterization}{PuiseuxParametrization}
\pmsynonym{Puiseux series parameterization}{PuiseuxParametrization}
\pmrelated{PuiseuxSeries}

% this is the default PlanetMath preamble.  as your knowledge
% of TeX increases, you will probably want to edit this, but
% it should be fine as is for beginners.

% almost certainly you want these
\usepackage{amssymb}
\usepackage{amsmath}
\usepackage{amsfonts}

% used for TeXing text within eps files
%\usepackage{psfrag}
% need this for including graphics (\includegraphics)
%\usepackage{graphicx}
% for neatly defining theorems and propositions
\usepackage{amsthm}
% making logically defined graphics
%%%\usepackage{xypic}

% there are many more packages, add them here as you need them

% define commands here
\theoremstyle{theorem}
\newtheorem*{thm}{Theorem}
\newtheorem*{lemma}{Lemma}
\newtheorem*{conj}{Conjecture}
\newtheorem*{cor}{Corollary}
\newtheorem*{example}{Example}
\newtheorem*{prop}{Proposition}
\theoremstyle{definition}
\newtheorem*{defn}{Definition}
\theoremstyle{remark}
\newtheorem*{rmk}{Remark}
\begin{document}
\begin{thm}
Suppose that $V \subset U \subset {\mathbb{C}}^2$ is an irreducible complex analytic subset of (complex) dimension 1 where $U$ is a domain.  Suppose that $0 \in V$.  Then there exists an analytic (holomorphic) map $f \colon {\mathbb{D}} \to V$, where ${\mathbb{D}}$ is the unit disc,
such that $f(0) = 0$ and $f({\mathbb{D}}) = N$ where $N \subset V$ is a neighbourhood of $0$ in $V$, $f$ is one to one, and further $f |_{{\mathbb{D}}\backslash \{0\}}$ is a biholomorphism onto $N \backslash \{0\}$.
In fact there exist suitable local coordinates $(z,w)$ in ${\mathbb{C}}^2$ such that $f$ is then given by $\xi \mapsto (z,w)$ where $z = \xi^k$,
$w = \sum_{n=m}^\infty a_n \xi^n$ where $m > k$.
\end{thm} 

This is sometimes written as
\begin{equation*}
w = \sum_{n=m}^\infty a_n z^{n/k}
\end{equation*}
and hence the name {\em Puiseux series parametrization}.  If you do however write it like this, it must be properly interpreted, as the Puiseux series is in general not single valued.

A similar result for arbitrary complex analytic sets with singularities of codimension 1 in higher dimensional spaces under further conditions on the singular set was obtained by Stutz, see Chirka \cite{Chirka:CAS} page 98.

\begin{thebibliography}{9}
\bibitem{Chirka:CAS}
E.\@ M.\@ Chirka.
{\em \PMlinkescapetext{Complex Analytic Sets}}.
Kluwer Academic Publishers, Dordrecht, The Netherlands, 1989.
\bibitem{Dimca:singu}
Alexandru Dimca.
{\em \PMlinkescapetext{Topics on Real and Complex Singularities}}.
Vieweg, Braunschweig, Germany, 1987.
\end{thebibliography}
%%%%%
%%%%%
\end{document}
