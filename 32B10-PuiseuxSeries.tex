\documentclass[12pt]{article}
\usepackage{pmmeta}
\pmcanonicalname{PuiseuxSeries}
\pmcreated{2013-03-22 15:20:28}
\pmmodified{2013-03-22 15:20:28}
\pmowner{jirka}{4157}
\pmmodifier{jirka}{4157}
\pmtitle{Puiseux series}
\pmrecord{5}{37161}
\pmprivacy{1}
\pmauthor{jirka}{4157}
\pmtype{Definition}
\pmcomment{trigger rebuild}
\pmclassification{msc}{32B10}
\pmsynonym{fractional power series}{PuiseuxSeries}
\pmrelated{PuiseuxParametrization}
\pmrelated{GeneralPower}

\endmetadata

% this is the default PlanetMath preamble.  as your knowledge
% of TeX increases, you will probably want to edit this, but
% it should be fine as is for beginners.

% almost certainly you want these
\usepackage{amssymb}
\usepackage{amsmath}
\usepackage{amsfonts}

% used for TeXing text within eps files
%\usepackage{psfrag}
% need this for including graphics (\includegraphics)
%\usepackage{graphicx}
% for neatly defining theorems and propositions
\usepackage{amsthm}
% making logically defined graphics
%%%\usepackage{xypic}

% there are many more packages, add them here as you need them

% define commands here
\theoremstyle{theorem}
\newtheorem*{thm}{Theorem}
\newtheorem*{lemma}{Lemma}
\newtheorem*{conj}{Conjecture}
\newtheorem*{cor}{Corollary}
\newtheorem*{example}{Example}
\newtheorem*{prop}{Proposition}
\theoremstyle{definition}
\newtheorem*{defn}{Definition}
\theoremstyle{remark}
\newtheorem*{rmk}{Remark}
\begin{document}
A formal series of the form
\begin{equation*}
\sum_{n=m}^\infty a_n z^{n/k}
\end{equation*}
where $m$ and $k$ are integers such that $k \geq 1$ is
is called a {\em Puiseux series} or a {\em fractional power series}.  Note that if $k > 1$, then $z^{n/k}$ could be multivalued.  One example of the use of such a power series is the Puiseux parametrization of one-dimensional complex analytic varieties.

\begin{thebibliography}{9}
\bibitem{Chirka:CAS}
E.\@ M.\@ Chirka.
{\em \PMlinkescapetext{Complex Analytic Sets}}.
Kluwer Academic Publishers, Dordrecht, The Netherlands, 1989.
\bibitem{Dimca:singu}
Alexandru Dimca.
{\em \PMlinkescapetext{Topics on Real and Complex Singularities}}.
Vieweg, Braunschweig, Germany, 1987.
\end{thebibliography}
%%%%%
%%%%%
\end{document}
