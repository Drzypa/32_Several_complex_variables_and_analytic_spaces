\documentclass[12pt]{article}
\usepackage{pmmeta}
\pmcanonicalname{SubanalyticSet}
\pmcreated{2013-03-22 16:46:16}
\pmmodified{2013-03-22 16:46:16}
\pmowner{jirka}{4157}
\pmmodifier{jirka}{4157}
\pmtitle{subanalytic set}
\pmrecord{6}{38999}
\pmprivacy{1}
\pmauthor{jirka}{4157}
\pmtype{Definition}
\pmcomment{trigger rebuild}
\pmclassification{msc}{32B20}
\pmclassification{msc}{14P15}
\pmrelated{TarskiSeidenbergTheorem}
\pmrelated{SemialgebraicSet}
\pmdefines{subanalytic}
\pmdefines{semianalytic set}
\pmdefines{semianalytic}
\pmdefines{semianalytic function}
\pmdefines{subanalytic function}
\pmdefines{semianalytic mapping}
\pmdefines{subanalytic mapping}
\pmdefines{dimension of a subanalytic set}

% this is the default PlanetMath preamble.  as your knowledge
% of TeX increases, you will probably want to edit this, but
% it should be fine as is for beginners.

% almost certainly you want these
\usepackage{amssymb}
\usepackage{amsmath}
\usepackage{amsfonts}

% used for TeXing text within eps files
%\usepackage{psfrag}
% need this for including graphics (\includegraphics)
%\usepackage{graphicx}
% for neatly defining theorems and propositions
\usepackage{amsthm}
% making logically defined graphics
%%%\usepackage{xypic}

% there are many more packages, add them here as you need them

% define commands here
\theoremstyle{theorem}
\newtheorem*{thm}{Theorem}
\newtheorem*{lemma}{Lemma}
\newtheorem*{conj}{Conjecture}
\newtheorem*{cor}{Corollary}
\newtheorem*{example}{Example}
\newtheorem*{prop}{Proposition}
\theoremstyle{definition}
\newtheorem*{defn}{Definition}
\theoremstyle{remark}
\newtheorem*{rmk}{Remark}

\begin{document}
Let $U \subset {\mathbb{R}}^n$.
Suppose $\mathcal{A}(U)$ is any ring of real valued functions on
$U$.
Define $\mathcal{S}(\mathcal{A}(U))$ to be the smallest
set of subsets of $U$, which contain the sets
$\{ x\in U \mid f(x) > 0 \}$ for all $f \in \mathcal{A}(U)$,
and is closed under finite union, finite intersection and complement.

\begin{defn}
A set $V \subset {\mathbb{R}}^n$ is {\em semianalytic}
if and only if for each $x \in {\mathbb{R}}^n$, there exists a neighbourhood
$U$ of $x$, such that $V \cap U \in \mathcal{S}(\mathcal{O}(U))$, where $\mathcal{O}(U)$
denotes the real-analytic real valued functions.
\end{defn}

Unlike for semialgebraic sets, there is no Tarski-Seidenberg theorem for semianalytic sets, and projections of semianalytic sets are in general not semianalytic.

\begin{defn}
We say
$V \subset {\mathbb{R}}^n$ is a {\em subanalytic} set if for
each $x \in {\mathbb{R}}^n$, there exists a relatively compact semianalytic set
$X \subset {\mathbb{R}}^{n+m}$ and a neighbourhood $U$ of $x$, such that
$V \cap U$ is the projection of $X$ onto the first $n$ coordinates.
\end{defn}

In particular all semianalytic sets are subanalytic.
On an open dense set
subanalytic sets are submanifolds and hence we
can define dimension.  Hence at a point $p$, where a set $A$ is a submanifold,
the dimension $\dim_p A$ is the dimension of the submanifold.  The {\em dimension} of the subanalytic set is the maximum $\dim_p A$ for all
$p$ where $A$ is a submanifold.
Semianalytic sets are contained in a real-analytic subvariety of the same dimension.  However, subanalytic sets are not in general
contained in any subvariety of the same dimension.  We do have however the
following.

\begin{thm}
A subanalytic set $A$ can be written as a locally finite union of
submanifolds.
\end{thm}

The set of subanalytic sets is still not completely closed under projections however.  Note that
a real-analytic subvariety that is not relatively compact can have a
projection which is not a locally finite union of submanifolds, and hence
is not subanalytic.

\begin{defn}
Let $U \subset {\mathbb{R}}^n$.  A mapping $f \colon U \to {\mathbb{R}}^m$ is said to be subanalytic (resp.\@ semianalytic)
if the graph of $f$ (i.e.\@ the set $\{ (x,y) \in U \times {\mathbb{R}}^m ~\mid~
x, y=f(x) \}$) is subanalytic (resp.\@ semianalytic)
\end{defn}










\begin{thebibliography}{9}
\bibitem{BM:semisub}
Edward Bierstone and Pierre~D. Milman, \emph{Semianalytic and subanalytic
  sets}, Inst. Hautes \'Etudes Sci. Publ. Math. (1988), no.~67, 5--42.
  \PMlinkexternal{MR 89k:32011}{http://www.ams.org/mathscinet-getitem?mr=89k:32011}
\end{thebibliography}

%%%%%
%%%%%
\end{document}
