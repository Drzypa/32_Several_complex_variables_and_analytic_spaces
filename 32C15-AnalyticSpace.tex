\documentclass[12pt]{article}
\usepackage{pmmeta}
\pmcanonicalname{AnalyticSpace}
\pmcreated{2013-03-22 17:41:43}
\pmmodified{2013-03-22 17:41:43}
\pmowner{jirka}{4157}
\pmmodifier{jirka}{4157}
\pmtitle{analytic space}
\pmrecord{4}{40136}
\pmprivacy{1}
\pmauthor{jirka}{4157}
\pmtype{Definition}
\pmcomment{trigger rebuild}
\pmclassification{msc}{32C15}
\pmsynonym{complex analytic space}{AnalyticSpace}
\pmrelated{LocallyCompactGroupoids}

\endmetadata

% this is the default PlanetMath preamble.  as your knowledge
% of TeX increases, you will probably want to edit this, but
% it should be fine as is for beginners.

% almost certainly you want these
\usepackage{amssymb}
\usepackage{amsmath}
\usepackage{amsfonts}

% used for TeXing text within eps files
%\usepackage{psfrag}
% need this for including graphics (\includegraphics)
%\usepackage{graphicx}
% for neatly defining theorems and propositions
\usepackage{amsthm}
% making logically defined graphics
%%%\usepackage{xypic}

% there are many more packages, add them here as you need them

% define commands here
\theoremstyle{theorem}
\newtheorem*{thm}{Theorem}
\newtheorem*{lemma}{Lemma}
\newtheorem*{conj}{Conjecture}
\newtheorem*{cor}{Corollary}
\newtheorem*{example}{Example}
\newtheorem*{prop}{Proposition}
\theoremstyle{definition}
\newtheorem*{defn}{Definition}
\theoremstyle{remark}
\newtheorem*{rmk}{Remark}

\begin{document}
A Hausdorff topological space $X$ is said to be an {\em analytic space} if:
\begin{enumerate}
\item There exists a countable number of open sets $V_j$ covering $X.$
\item For each $V_j$ there exists a homeomorphism $\varphi_j \colon Y_j \to V_j ,$
 where $Y_j$ is a local complex analytic subvariety in some ${\mathbb{C}}^n .$
\item If $V_j$ and $V_k$ overlap, then $\varphi_j^{-1} \circ \varphi_k$ is a biholomorphism.
\end{enumerate}

Usually one attaches to $X$ a set of coordinate systems $\mathcal{G}$, which is a set (now uncountable)
of triples $(V_\iota,\varphi_\iota,Y_\iota)$ as above, such that whenever $V$ is an open set, $Y$
a local complex analytic subvariety, and a homeomorphism $\varphi \colon Y \to V$, such that
$\varphi_\iota^{-1} \circ \varphi$ is a biholomorphism for some  $(V_\iota,\varphi_\iota,Y_\iota) \in \mathcal{G}$
then $(V,\varphi,Y) \in \mathcal{G} .$  Basically $\mathcal{G}$ is the set of all possible coordinate systems
for $X$.

We can also define the singular set of an analytic space.  A point $p$ is
\PMlinkescapetext{{\em simple}, {\em regular} or {\em nonsingular}}
if there exists (at least one) a coordinate system $(V_\iota,\varphi_\iota,Y_\iota) \in \mathcal{G}$ with $p \in V_\iota$
and $Y_\iota$ a complex manifold.  All other points are the singular points.

Any local complex analytic subvariety is an analytic space, so this is a natural generalization of the concept of a subvariety.

%FIXME: this needs more discussion I think

\begin{thebibliography}{9}
\bibitem{Whitney:varieties}
Hassler Whitney.
{\em \PMlinkescapetext{Complex Analytic Varieties}}.
Addison-Wesley, Philippines, 1972.
\end{thebibliography}
%%%%%
%%%%%
\end{document}
