\documentclass[12pt]{article}
\usepackage{pmmeta}
\pmcanonicalname{RiemannSphere}
\pmcreated{2013-03-22 13:45:55}
\pmmodified{2013-03-22 13:45:55}
\pmowner{CWoo}{3771}
\pmmodifier{CWoo}{3771}
\pmtitle{Riemann sphere}
\pmrecord{19}{34469}
\pmprivacy{1}
\pmauthor{CWoo}{3771}
\pmtype{Definition}
\pmcomment{trigger rebuild}
\pmclassification{msc}{32C15}
%\pmkeywords{compactification}
\pmrelated{StereographicProjection}
\pmrelated{Complex}
\pmrelated{ClosedComplexPlane}
\pmrelated{CircumferentialAngle}
\pmrelated{MercatorProjection}
\pmdefines{geographic coordinates}
\pmdefines{longitude}
\pmdefines{latitude}

\endmetadata

% this is the default PlanetMath preamble.  as your knowledge
% of TeX increases, you will probably want to edit this, but
% it should be fine as is for beginners.

% almost certainly you want these
\usepackage{amssymb}
\usepackage{amsmath}
\usepackage{amsfonts}
\usepackage{amsthm}

\usepackage{mathrsfs}
\usepackage{pstricks}
\usepackage{pst-plot}

% used for TeXing text within eps files
%\usepackage{psfrag}
% need this for including graphics (\includegraphics)
%\usepackage{graphicx}
% for neatly defining theorems and propositions
%
% making logically defined graphics
%%%\usepackage{xypic}

% there are many more packages, add them here as you need them

% define commands here

\newcommand{\sR}[0]{\mathbb{R}}
\newcommand{\sC}[0]{\mathbb{C}}
\newcommand{\sN}[0]{\mathbb{N}}
\newcommand{\sZ}[0]{\mathbb{Z}}

 \usepackage{bbm}
 \newcommand{\Z}{\mathbbmss{Z}}
 \newcommand{\C}{\mathbbmss{C}}
 \newcommand{\F}{\mathbbmss{F}}
 \newcommand{\R}{\mathbbmss{R}}
 \newcommand{\Q}{\mathbbmss{Q}}
\newcommand*{\norm}[1]{\lVert #1 \rVert}
\newcommand*{\abs}[1]{| #1 |}

\newtheorem{thm}{Theorem}
\newtheorem{defn}{Definition}
\newtheorem{prop}{Proposition}
\newtheorem{lemma}{Lemma}
\newtheorem{cor}{Corollary}
\begin{document}
The Riemann sphere, denoted $\hat{\mathbb{C}}$, is the one-point compactification of the complex plane $\mathbb{C}$, obtained by identifying the limits of all infinitely extending rays from the origin as one single ``point at infinity.''  Heuristically, $\hat{\mathbb{C}}$ can be viewed as a 2-sphere with the top point corresponding to the point at infinity, and the bottom point corresponding the origin.  An atlas for the Riemann sphere is given by two charts:  
\begin{align*}
\hat{\mathbb{C}}\backslash\{\infty\}\rightarrow\mathbb{C}:z\mapsto z
\end{align*}
and
\begin{align*}
\hat{\mathbb{C}}\backslash\{0\}\rightarrow\mathbb{C}:z\mapsto \frac{1}{z}
\end{align*}
Any rational function on $\hat{\mathbb{C}}$ has a unique smooth extension to a map $\hat{p}:\hat{\mathbb{C}}\rightarrow\hat{\mathbb{C}}$.\\

Concretely, the bijective correspondence of the points of the closed complex plane and the Riemann sphere is implemented by the stereographic projection.\, Think a sphere of radius $R$ being above the complex plane and having it as tangent plane with the origin as the point of tangency.\, Call this point the South Pole and the opposite point $N$ of the  sphere the North Pole.\, For an arbitrary point $P$ of the complex plane, set the line through it and $N$.\, The line intersects the sphere in another point $P'$.\, The mapping
\begin{align}
P \mapsto P'
\end{align}
is a bijection between the closed complex plane and the sphere.\, Especially, the origin is mapped onto the South Pole and $\infty$ onto the North Pole.

\begin{center}
\begin{pspicture}(-6,-4)(7.5,3)
\psdot(0,0)
\psdot[linecolor=blue](0,1.98)
\rput(0,2.3){$N$}
\rput(4.3,-3.1){$P$}
\rput(1.2,1.1){$P'$}
\pscircle[linecolor=blue](0,0){2.04}
\psplot[linecolor=blue]{-0.6}{0}{3.33 0.36 x x mul sub sqrt mul}
\psplot[linecolor=blue]{-0.6}{0}{-3.33 0.36 x x mul sub sqrt mul}
\psplot[linestyle=dotted]{0}{0.6}{3.33 0.36 x x mul sub sqrt mul}
\psplot[linestyle=dotted]{0}{0.6}{-3.33 0.36 x x mul sub sqrt mul}
\psplot[linecolor=blue]{-2}{2}{-0.3 4 x x mul sub sqrt mul}
\psplot[linestyle=dotted]{-2}{2}{0.3 4 x x mul sub sqrt mul}
\psline[linewidth=0.06](-5.5,-3)(4,-3)
\psline[linewidth=0.05](4,-3)(6.5,-1)
\psline(-5.5,-3)(-3,-1)
\psline(-3,-1)(-1.73,-1)
\psline(1.73,-1)(2.3,-1)
\psline(2.5,-1)(6.5,-1)
\psline(-1.2,-3)(0,-2)
\psline[linestyle=dashed](0,2)(0.89,0.89)
\psline[linecolor=red](0.89,0.89)(4,-3)
\psdot[linecolor=red](4,-3)
\psdot[linecolor=blue](0.89,0.89)
\rput(-6,-4){.}
\rput(7.4,3){.}
\end{pspicture}
\end{center}


If we equip the sphere with {\em geographic coordinates}, the {\em longitude} $\lambda$ ($-\pi < \lambda \leqq \pi$) and the {\em latitude} $\varphi$ ($-\frac{\pi}{2} \leqq \varphi \leqq \frac{\pi}{2}$) and fix that the points of the positive real axis are mapped onto the zero meridian\, $\lambda = 0$,\, then the polar coordinates (argument and modulus) $\theta$ and $r$ of $P$ in the mapping (1) are \PMlinkescapetext{connected} with the geographic coordinates of $P'$ by the equations
   $$\theta \;\equiv\; \lambda \!\pmod{2\pi}, \quad r \;=\; 2R\tan\left(\frac{\varphi}{2}+\frac{\pi}{4}\right),$$
as is easily checked.\, One can also state that the distance $h$ of $P'$ from the plane is given by
              $$h \;=\; \frac{2Rr^2}{4R^2\!+\!r^2}.$$

%%%%%
%%%%%
\end{document}
