\documentclass[12pt]{article}
\usepackage{pmmeta}
\pmcanonicalname{WeaklyHolomorphic}
\pmcreated{2013-03-22 17:41:46}
\pmmodified{2013-03-22 17:41:46}
\pmowner{jirka}{4157}
\pmmodifier{jirka}{4157}
\pmtitle{weakly holomorphic}
\pmrecord{4}{40137}
\pmprivacy{1}
\pmauthor{jirka}{4157}
\pmtype{Definition}
\pmcomment{trigger rebuild}
\pmclassification{msc}{32C15}
\pmclassification{msc}{32C20}
\pmsynonym{w-holomoprhic}{WeaklyHolomorphic}
\pmrelated{NormalComplexAnalyticVariety}

\endmetadata

% this is the default PlanetMath preamble.  as your knowledge
% of TeX increases, you will probably want to edit this, but
% it should be fine as is for beginners.

% almost certainly you want these
\usepackage{amssymb}
\usepackage{amsmath}
\usepackage{amsfonts}

% used for TeXing text within eps files
%\usepackage{psfrag}
% need this for including graphics (\includegraphics)
%\usepackage{graphicx}
% for neatly defining theorems and propositions
\usepackage{amsthm}
% making logically defined graphics
%%%\usepackage{xypic}

% there are many more packages, add them here as you need them

% define commands here
\theoremstyle{theorem}
\newtheorem*{thm}{Theorem}
\newtheorem*{lemma}{Lemma}
\newtheorem*{conj}{Conjecture}
\newtheorem*{cor}{Corollary}
\newtheorem*{example}{Example}
\newtheorem*{prop}{Proposition}
\theoremstyle{definition}
\newtheorem*{defn}{Definition}
\theoremstyle{remark}
\newtheorem*{rmk}{Remark}

\begin{document}
Let $V$ be a local complex analytic variety.
A function $f \colon U \subset V \to \mathbb{C}$ (where $U$ is open in $V$)
is said to be {\em weakly holomorphic} through $U$
if there exists a nowhere dense complex analytic subvariety $W \subset V$
and $W$ contains the singular points of $V$ and $V \setminus W \subset U$,
and such that $f$ is holomorphic on $V \setminus W$ and
$f$ is locally bounded on $V$.

It is not hard to show that we can then just take $W$ to be the set of singular
points of $V$ and have $U = V \setminus W$ as we can extend $f$ to all the
nonsingular points of $V$.

Usually we denote by ${\mathcal{O}}^w(V)$ the ring of weakly holomorphic functions
through $V$.  Since any neighbourhood of a point $p$ in $V$ is a local analytic subvariety,
we can define germs of weakly holomorphic functions at $p$ in the obvious way.  We usually
denote by ${\mathcal{O}}_p^w(V)$ the ring of germs at $p$ of weakly holomorphic
functions.

\begin{thebibliography}{9}
\bibitem{Whitney:varieties}
Hassler Whitney.
{\em \PMlinkescapetext{Complex Analytic Varieties}}.
Addison-Wesley, Philippines, 1972.
\end{thebibliography}
%%%%%
%%%%%
\end{document}
