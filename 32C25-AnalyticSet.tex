\documentclass[12pt]{article}
\usepackage{pmmeta}
\pmcanonicalname{AnalyticSet}
\pmcreated{2013-03-22 14:59:28}
\pmmodified{2013-03-22 14:59:28}
\pmowner{jirka}{4157}
\pmmodifier{jirka}{4157}
\pmtitle{analytic set}
\pmrecord{10}{36696}
\pmprivacy{1}
\pmauthor{jirka}{4157}
\pmtype{Definition}
\pmcomment{trigger rebuild}
\pmclassification{msc}{32C25}
\pmclassification{msc}{32A60}
\pmsynonym{analytic variety}{AnalyticSet}
\pmsynonym{complex analytic variety}{AnalyticSet}
\pmrelated{IrreducibleComponent2}
\pmdefines{regular point}
\pmdefines{simple point}
\pmdefines{top simple point}
\pmdefines{singular point}
\pmdefines{locally analytic}
\pmdefines{dimension of a variety}
\pmdefines{subvariety of a complex analytic variety}
\pmdefines{complex analytic subvariety}

% this is the default PlanetMath preamble.  as your knowledge
% of TeX increases, you will probably want to edit this, but
% it should be fine as is for beginners.

% almost certainly you want these
\usepackage{amssymb}
\usepackage{amsmath}
\usepackage{amsfonts}

% used for TeXing text within eps files
%\usepackage{psfrag}
% need this for including graphics (\includegraphics)
%\usepackage{graphicx}
% for neatly defining theorems and propositions
\usepackage{amsthm}
% making logically defined graphics
%%%\usepackage{xypic}

% there are many more packages, add them here as you need them

% define commands here
\theoremstyle{theorem}
\newtheorem*{thm}{Theorem}
\newtheorem*{lemma}{Lemma}
\newtheorem*{conj}{Conjecture}
\newtheorem*{cor}{Corollary}
\theoremstyle{definition}
\newtheorem*{defn}{Definition}
\begin{document}
Let $G \subset {\mathbb{C}}^N$ be an open set.

\begin{defn}
A set $V \subset G$ is said to be {\em locally analytic}
if for every point $p \in V$ there exists a neighbourhood $U$ of $p$ in $G$
and holomorphic functions $f_1,\cdots,f_m$ defined in $U$ such that
$U \cap V = \{ z : f_k(z) = 0 \text{for all} 1\leq k \leq m \}.$
\end{defn}

This basically says that around each point of $V,$ the set $V$ is analytic.
A stronger definition is required.

\begin{defn}
A set $V \subset G$ is said to be an {\em analytic variety} in $G$
(or {\em analytic set} in $G$)
if for every point $p \in G$ there exists a neighbourhood $U$ of $p$ in $G$
and holomorphic functions $f_1,\cdots,f_m$ defined in $U$ such that
$U \cap V = \{ z : f_k(z) = 0 \text{ for all } 1\leq k \leq m \}.$
\end{defn}

Note the change, now $V$ is analytic around each point of $G.$  Since the
zero sets of holomorphic functions are closed, this for example implies that
$V$ is relatively closed in $G,$ while a local variety need not be closed.
Sometimes an analytic variety is called an {\em analytic set}.

At most points an analytic variety $V$ will in fact be a complex
analytic manifold.  So

\begin{defn}
A point $p \in V$ is called a {\em regular point} if there is a neighbourhood
$U$ of $p$ such that $U \cap V$ is a complex analytic manifold.  Any other
point is called a {\em singular point}.
\end{defn}

The set of regular points of $V$ is denoted by $V^-$ or sometimes $V^*.$

For any regular point $p \in V$ we can define the dimension as
\begin{equation*}
\operatorname{dim}_p(V) = 
\operatorname{dim}_{\mathbb{C}}(U \cap V)
\end{equation*}
where $U$ is as above and thus $U \cap V$ is a manifold with a well defined
dimension.  Here we of course take the complex dimension of these manifolds.

\begin{defn}
Let $V$ be an analytic variety,
we define the dimension of $V$ by
\begin{equation*}
\operatorname{dim}(V)
=
\sup \{ \operatorname{dim}_p(V) : p \text{ a regular point of } V \} .
\end{equation*}
\end{defn}

\begin{defn}
The regular point $p \in V$ such that $\dim_p(V) = \dim(V)$ is called a {\em top
\PMlinkescapetext{simple} point} of $V$.
\end{defn}

Similarly as for manifolds we can also talk about subvarieties.  In this case we modify definition a little bit.

\begin{defn}
A set $W \subset V$ where $V \subset G$ is a local variety is said to be
a {\em subvariety} of $V$
if for every point $p \in V$ there exists a neighbourhood $U$ of $p$ in $G$
and holomorphic functions $f_1,\cdots,f_m$ defined in $U$ such that
$U \cap W = \{ z : f_k(z) = 0 \text{ for all } 1\leq k \leq m \}$.
\end{defn}

That is, a subset $W$ is a subvariety if it is definined by the vanishing of analytic functions near all points of $V$.

\begin{thebibliography}{9}
\bibitem{Chirka:CAS}
E.\@ M.\@ Chirka.
{\em \PMlinkescapetext{Complex Analytic Sets}}.
Kluwer Academic Publishers, Dordrecht, The Netherlands, 1989.
\bibitem{Whitney:varieties}
Hassler Whitney.
{\em \PMlinkescapetext{Complex Analytic Varieties}}.
Addison-Wesley, Philippines, 1972.
\end{thebibliography}
%%%%%
%%%%%
\end{document}
