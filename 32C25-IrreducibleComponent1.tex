\documentclass[12pt]{article}
\usepackage{pmmeta}
\pmcanonicalname{IrreducibleComponent1}
\pmcreated{2013-03-22 15:04:58}
\pmmodified{2013-03-22 15:04:58}
\pmowner{jirka}{4157}
\pmmodifier{jirka}{4157}
\pmtitle{irreducible component}
\pmrecord{5}{36806}
\pmprivacy{1}
\pmauthor{jirka}{4157}
\pmtype{Definition}
\pmcomment{trigger rebuild}
\pmclassification{msc}{32C25}
\pmclassification{msc}{32A60}
\pmsynonym{ircomp}{IrreducibleComponent1}
\pmrelated{AnalyticSet}
\pmdefines{irreducible analytic variety}
\pmdefines{irreducible locally analytic set}
\pmdefines{irreducible analytic variety}
\pmdefines{reducible locally analytic set}
\pmdefines{reducible analytic variety}

% this is the default PlanetMath preamble.  as your knowledge
% of TeX increases, you will probably want to edit this, but
% it should be fine as is for beginners.

% almost certainly you want these
\usepackage{amssymb}
\usepackage{amsmath}
\usepackage{amsfonts}

% used for TeXing text within eps files
%\usepackage{psfrag}
% need this for including graphics (\includegraphics)
%\usepackage{graphicx}
% for neatly defining theorems and propositions
\usepackage{amsthm}
% making logically defined graphics
%%%\usepackage{xypic}

% there are many more packages, add them here as you need them

% define commands here

\theoremstyle{definition}
\newtheorem*{defn}{Definition}
\theoremstyle{theorem}
\newtheorem*{thm}{Theorem}
\begin{document}
Let $G \subset {\mathbb{C}}^N$ be an open set.

\begin{defn}
A locally analytic set (or an analytic variety) $V \subset G$ is said to be {\em irreducible} if whenever we have two locally analytic sets $V_1$ and $V_2$ such that $V = V_1 \cup V_2$, then either $V = V_1$ or $V = V_2$.  Otherwise $V$ is
said to be {\em \PMlinkescapetext{reducible}}.  A maximal irreducible subvariety of $V$ is said to be an {\em irreducible component} of $V$.  Sometimes irreducible components are
called {\em ircomps}.
\end{defn}

Note that if $V$ is an analytic variety in $G$, then a subvariety $W$ is an irreducible component of $V$ if and only if $W^*$ (the set of regular points of $W$) is a connected complex analytic manifold.  This means that the irreducible components of $V$ are the closures of the connected components of $V^*$.

\begin{thebibliography}{9}
\bibitem{Whitney:varieties}
Hassler Whitney.
{\em \PMlinkescapetext{Complex Analytic Varieties}}.
Addison-Wesley, Philippines, 1972.
\end{thebibliography}
%%%%%
%%%%%
\end{document}
