\documentclass[12pt]{article}
\usepackage{pmmeta}
\pmcanonicalname{HolomorphicallyConvex}
\pmcreated{2013-03-22 15:04:33}
\pmmodified{2013-03-22 15:04:33}
\pmowner{jirka}{4157}
\pmmodifier{jirka}{4157}
\pmtitle{holomorphically convex}
\pmrecord{8}{36798}
\pmprivacy{1}
\pmauthor{jirka}{4157}
\pmtype{Definition}
\pmcomment{trigger rebuild}
\pmclassification{msc}{32E05}
\pmsynonym{holomorph-convex}{HolomorphicallyConvex}
\pmrelated{PolynomiallyConvexHull}
\pmrelated{SteinManifold}
\pmdefines{holomorphically convex hull}

\endmetadata

% this is the default PlanetMath preamble.  as your knowledge
% of TeX increases, you will probably want to edit this, but
% it should be fine as is for beginners.

% almost certainly you want these
\usepackage{amssymb}
\usepackage{amsmath}
\usepackage{amsfonts}

% used for TeXing text within eps files
%\usepackage{psfrag}
% need this for including graphics (\includegraphics)
%\usepackage{graphicx}
% for neatly defining theorems and propositions
\usepackage{amsthm}
% making logically defined graphics
%%%\usepackage{xypic}

% there are many more packages, add them here as you need them

% define commands here
\theoremstyle{theorem}
\newtheorem*{thm}{Theorem}
\newtheorem*{lemma}{Lemma}
\newtheorem*{conj}{Conjecture}
\newtheorem*{cor}{Corollary}
\newtheorem*{example}{Example}
\theoremstyle{definition}
\newtheorem*{defn}{Definition}
\begin{document}
Let $G \subset {\mathbb{C}}^n$ be a domain, or alternatively for a more general definition let $G$ be an $n$ dimensional complex analytic manifold.
Further let ${\mathcal{O}}(G)$ stand for the set of holomorphic functions on $G$.

\begin{defn}
Let $K \subset G$ be a compact set.
We define the {\em holomorphically \PMlinkescapetext{convex hull}} of $K$ as
\begin{equation*}
\hat{K}_G := \{ z \in G \mid \lvert f(z) \rvert \leq \sup_{w \in K} \lvert f(w) \rvert \text{ for all } f \in {\mathcal{O}}(G) \} .
\end{equation*}
The domain $G$ is called {\em holomorphically convex} if for every $K \subset G$ compact in $G$, $\hat{K}_G$ is also compact in $G$.  Sometimes this is just abbreviated as {\em holomorph-convex}.
\end{defn}

Note that when $n=1$, any domain $G$ is holomorphically convex since when $n=1$ $\hat{K}_G = K$ for all compact $K \subset G$.  Also note that this is the same as being a domain of holomorphy.

\begin{thebibliography}{9}
\bibitem{Hormander:several}
Lars H\"ormander.
{\em \PMlinkescapetext{An Introduction to Complex Analysis in Several
Variables}},
North-Holland Publishing Company, New York, New York, 1973.
\bibitem{Krantz:several}
Steven~G.\@ Krantz.
{\em \PMlinkescapetext{Function Theory of Several Complex Variables}},
AMS Chelsea Publishing, Providence, Rhode Island, 1992.
\end{thebibliography}
%%%%%
%%%%%
\end{document}
