\documentclass[12pt]{article}
\usepackage{pmmeta}
\pmcanonicalname{SteinManifold}
\pmcreated{2013-03-22 15:04:37}
\pmmodified{2013-03-22 15:04:37}
\pmowner{jirka}{4157}
\pmmodifier{jirka}{4157}
\pmtitle{Stein manifold}
\pmrecord{7}{36799}
\pmprivacy{1}
\pmauthor{jirka}{4157}
\pmtype{Definition}
\pmcomment{trigger rebuild}
\pmclassification{msc}{32E10}
\pmrelated{HolomorphicallyConvex}
\pmrelated{DomainOfHolomorphy}
\pmdefines{holomorphically separable}
\pmdefines{holomorphically spreadable}

% this is the default PlanetMath preamble.  as your knowledge
% of TeX increases, you will probably want to edit this, but
% it should be fine as is for beginners.

% almost certainly you want these
\usepackage{amssymb}
\usepackage{amsmath}
\usepackage{amsfonts}

% used for TeXing text within eps files
%\usepackage{psfrag}
% need this for including graphics (\includegraphics)
%\usepackage{graphicx}
% for neatly defining theorems and propositions
\usepackage{amsthm}
% making logically defined graphics
%%%\usepackage{xypic}

% there are many more packages, add them here as you need them

% define commands here
\theoremstyle{theorem}
\newtheorem*{thm}{Theorem}
\newtheorem*{lemma}{Lemma}
\newtheorem*{conj}{Conjecture}
\newtheorem*{cor}{Corollary}
\newtheorem*{example}{Example}
\newtheorem*{prop}{Proposition}
\theoremstyle{definition}
\newtheorem*{defn}{Definition}
\theoremstyle{remark}
\newtheorem*{rmk}{Remark}
\begin{document}
\begin{defn}
A complex manifold $M$ of complex dimension $n$ is a {\it Stein manifold} if it satisfies the following properties
\begin{enumerate}
\item $M$ is holomorphically convex,
\item if $z,w \in M$ and $z \not= w$ then $f(z) \not= f(w)$
for some function $f$ holomorphic on $M$ (i.e. $M$ is {\it holomorphically separable}),
\item for every $z \in M$ there are holomorphic functions $f_1,\ldots,f_n$
which form a coordinate system at $z$ (i.e. $M$ is {\it holomorphically spreadable}).
\end{enumerate}
\end{defn}

Stein manifold is a generalization of the concept of the domain of holomorphy to manifolds.  Furthermore, Stein manifolds are the generalizations of Riemann surfaces in higher dimensions.  Every noncompact Riemann surface is a Stein manifold
by a theorem of Behnke and Stein.
Note that every domain of holomorphy in ${\mathbb{C}}^n$ is a Stein manifold.
It is not hard to see that every closed complex submanifold of a Stein manifold is Stein.

\begin{thm}[Remmert, Narasimhan, Bishop]
If $M$ is a Stein manifold of dimension $n$.  There exists a \PMlinkname{proper}{ProperMap} holomorphic embedding of $M$ into ${\mathbb{C}}^{2n+1}$.
\end{thm}

Note that no compact complex manifold can be Stein since compact complex manifolds have no holomorphic functions.  On the other hand, every compact complex manifold is holomorphically convex.

\begin{thebibliography}{9}
\bibitem{Hormander:several}
Lars H\"ormander.
{\em \PMlinkescapetext{An Introduction to Complex Analysis in Several
Variables}},
North-Holland Publishing Company, New York, New York, 1973.
\bibitem{Krantz:several}
Steven~G.\@ Krantz.
{\em \PMlinkescapetext{Function Theory of Several Complex Variables}},
AMS Chelsea Publishing, Providence, Rhode Island, 1992.
\end{thebibliography}
%%%%%
%%%%%
\end{document}
