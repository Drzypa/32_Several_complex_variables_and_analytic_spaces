\documentclass[12pt]{article}
\usepackage{pmmeta}
\pmcanonicalname{BergmanMetric}
\pmcreated{2013-03-22 15:04:49}
\pmmodified{2013-03-22 15:04:49}
\pmowner{jirka}{4157}
\pmmodifier{jirka}{4157}
\pmtitle{Bergman metric}
\pmrecord{6}{36803}
\pmprivacy{1}
\pmauthor{jirka}{4157}
\pmtype{Definition}
\pmcomment{trigger rebuild}
\pmclassification{msc}{32F45}
\pmrelated{BergmanKernel}
\pmdefines{Bergman distance}

% this is the default PlanetMath preamble.  as your knowledge
% of TeX increases, you will probably want to edit this, but
% it should be fine as is for beginners.

% almost certainly you want these
\usepackage{amssymb}
\usepackage{amsmath}
\usepackage{amsfonts}

% used for TeXing text within eps files
%\usepackage{psfrag}
% need this for including graphics (\includegraphics)
%\usepackage{graphicx}
% for neatly defining theorems and propositions
\usepackage{amsthm}
% making logically defined graphics
%%%\usepackage{xypic}

% there are many more packages, add them here as you need them

% define commands here
\theoremstyle{theorem}
\newtheorem*{thm}{Theorem}
\newtheorem*{lemma}{Lemma}
\newtheorem*{conj}{Conjecture}
\newtheorem*{cor}{Corollary}
\theoremstyle{definition}
\newtheorem*{defn}{Definition}
\begin{document}
\begin{defn}
Let $G \subset {\mathbb{C}}^n$ be a domain and let $K(z,w)$ be the Bergman kernel
on $G$.  We define a Hermitian metric on the tangent bundle $T_z {\mathbb{C}}^n$ by
\begin{equation*}
g_{ij} (z)
:=
\frac{\partial^2}{\partial z_i \partial \bar{z}_j}
\log K(z,z) ,
\end{equation*}
for $z \in G$.  Then the length of a tangent vector $\xi \in T_z{\mathbb{C}}^n$ is then
given by
\begin{equation*}
\lvert \xi \rvert_{B,z}
:=
\sqrt{\sum_{i,j=1}^n g_{ij}(z) \xi_i \bar{\xi}_j }.
\end{equation*} 
This metric is called the {\em Bergman metric} on $G$.
\end{defn}

The length of a (piecewise) $C^1$ curve $\gamma \colon [0,1] \to {\mathbb{C}}^n$ is
then computed as
\begin{equation*}
\ell (\gamma) =
\int_0^1 \big\lvert \frac{\partial \gamma}{\partial t}(t) \big\rvert_{B,\gamma(t)} dt .
\end{equation*}
The distance $d_G(p,q)$ of two points $p,q \in G$ is then defined as
\begin{equation*}
d_G(p,q):=
\inf \{ \ell (\gamma) \mid \text{ all piecewise $C^1$ curves $\gamma$ such that $\gamma(0)=p$ and $\gamma(1)=q$} \} .
\end{equation*}
The distance $d_G$ is called the {\em Bergman distance}.

The Bergman metric is in fact a positive definite matrix at each point if $G$ is a bounded domain.  More importantly, the distance $d_G$ is invariant under
biholomorphic mappings of $G$ to another domain $G'$.  That is if $f$
is a biholomorphism of $G$ and $G'$, then $d_G(p,q) = d_{G'}(f(p),f(q))$.

\begin{thebibliography}{9}
\bibitem{Krantz:several}
Steven~G.\@ Krantz.
{\em \PMlinkescapetext{Function Theory of Several Complex Variables}},
AMS Chelsea Publishing, Providence, Rhode Island, 1992.
\end{thebibliography}
%%%%%
%%%%%
\end{document}
