\documentclass[12pt]{article}
\usepackage{pmmeta}
\pmcanonicalname{BiholomorphicallyEquivalent}
\pmcreated{2013-03-22 14:29:47}
\pmmodified{2013-03-22 14:29:47}
\pmowner{jirka}{4157}
\pmmodifier{jirka}{4157}
\pmtitle{biholomorphically equivalent}
\pmrecord{7}{36032}
\pmprivacy{1}
\pmauthor{jirka}{4157}
\pmtype{Definition}
\pmcomment{trigger rebuild}
\pmclassification{msc}{32H02}
\pmsynonym{biholomorphic}{BiholomorphicallyEquivalent}
\pmsynonym{biholomorphic equivalence}{BiholomorphicallyEquivalent}
\pmdefines{biholomorphic mapping}

% this is the default PlanetMath preamble.  as your knowledge
% of TeX increases, you will probably want to edit this, but
% it should be fine as is for beginners.

% almost certainly you want these
\usepackage{amssymb}
\usepackage{amsmath}
\usepackage{amsfonts}

% used for TeXing text within eps files
%\usepackage{psfrag}
% need this for including graphics (\includegraphics)
%\usepackage{graphicx}
% for neatly defining theorems and propositions
\usepackage{amsthm}
% making logically defined graphics
%%%\usepackage{xypic}

% there are many more packages, add them here as you need them

% define commands here
\theoremstyle{theorem}
\newtheorem*{thm}{Theorem}
\newtheorem*{lemma}{Lemma}
\newtheorem*{conj}{Conjecture}
\newtheorem*{cor}{Corollary}
\theoremstyle{definition}
\newtheorem*{defn}{Definition}
\begin{document}
\begin{defn}
Let $U,V \subset {\mathbb{C}}^n$.  If there exists a one-to-one and onto
holomorphic mapping $\phi \colon U \to V$ such that the inverse $\phi^{-1}$
exists and is also holomorphic, then we say that
$U$ and $V$ are {\em biholomorphically equivalent} or that they are
{\em biholomorphic}.  The mapping $\phi$ is called a {\em biholomorphic mapping}.
\end{defn}

It is not an obvious fact, but if the source and target dimension are the same then every one-to-one holomorphic mapping is biholomorphic as a one-to-one holomorphic map has a nonvanishing jacobian.

When $n=1$ biholomorphic equivalence is often called \PMlinkname{conformal equivalence}{ConformallyEquivalent}, since in one complex
dimension, the one-to-one holomorphic mappings are conformal mappings.

Further if $n=1$ then there are plenty of conformal (biholomorhic) equivalences,
since for example every simply connected \PMlinkname{domain}{Domain2} other than the whole complex plane is conformally
equivalent to the unit disc.  On the other hand,
when $n > 1$ then the open unit ball and open unit polydisc
are not biholomorphically equivalent.  In fact there does not exist
a \PMlinkname{proper}{ProperMap} holomorphic mapping from one to the other.

\begin{thebibliography}{9}
\bibitem{Krantz:several}
Steven~G.\@ Krantz.
{\em \PMlinkescapetext{Function Theory of Several Complex Variables}},
AMS Chelsea Publishing, Providence, Rhode Island, 1992.
\end{thebibliography}
%%%%%
%%%%%
\end{document}
