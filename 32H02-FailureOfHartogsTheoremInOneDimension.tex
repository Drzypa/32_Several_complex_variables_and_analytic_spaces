\documentclass[12pt]{article}
\usepackage{pmmeta}
\pmcanonicalname{FailureOfHartogsTheoremInOneDimension}
\pmcreated{2013-03-22 17:46:57}
\pmmodified{2013-03-22 17:46:57}
\pmowner{jirka}{4157}
\pmmodifier{jirka}{4157}
\pmtitle{failure of Hartogs' theorem in one dimension}
\pmrecord{4}{40242}
\pmprivacy{1}
\pmauthor{jirka}{4157}
\pmtype{Example}
\pmcomment{trigger rebuild}
\pmclassification{msc}{32H02}

\endmetadata

% this is the default PlanetMath preamble.  as your knowledge
% of TeX increases, you will probably want to edit this, but
% it should be fine as is for beginners.

% almost certainly you want these
\usepackage{amssymb}
\usepackage{amsmath}
\usepackage{amsfonts}

% used for TeXing text within eps files
%\usepackage{psfrag}
% need this for including graphics (\includegraphics)
%\usepackage{graphicx}
% for neatly defining theorems and propositions
\usepackage{amsthm}
% making logically defined graphics
%%%\usepackage{xypic}

% there are many more packages, add them here as you need them

% define commands here
\theoremstyle{theorem}
\newtheorem*{thm}{Theorem}
\newtheorem*{lemma}{Lemma}
\newtheorem*{conj}{Conjecture}
\newtheorem*{cor}{Corollary}
\newtheorem*{example}{Example}
\newtheorem*{prop}{Proposition}
\theoremstyle{definition}
\newtheorem*{defn}{Definition}
\theoremstyle{remark}
\newtheorem*{rmk}{Remark}

\begin{document}
It is instructive to see an example where Hartogs' theorem fails in one dimension.
Take $U = {\mathbb C}$ and let $K = \{0\}.$
The function $\frac{1}{z}$ is holomorphic in $U \setminus K,$ but cannot be extended to $U.$

To understand the example and failure of the theorem it is important to understand \PMlinkname{the proof}{ProofOfHartogsTheorem}.  In the proof, the way we construct an extension is that we start with a function holomorphic in $U \setminus K,$
modify it in a neighbourhood of $K$ to be zero, hence extending as a smooth function through $K.$  Then we solve the
\PMlinkname{$\bar{\partial}$ operator}{BarPartialOperator} inhomogeneous equation $\bar{\partial}\psi = g$ to ``correct'' our extension to be holomorphic.
The key point is that $g$ has compact support allowing us to solve the equation and find a $\psi$
with compact support.  This fails in dimension 1.  While we always get a solution $\psi,$ the solution can never have compact support.  Hence, if we tried the proof with $\frac{1}{z},$ the new function we obtain in the proof does not agree with $\frac{1}{z}$ on any open set and hence is not an extension.
%%%%%
%%%%%
\end{document}
