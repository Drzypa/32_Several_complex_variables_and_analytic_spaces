\documentclass[12pt]{article}
\usepackage{pmmeta}
\pmcanonicalname{LandausConstant}
\pmcreated{2013-03-22 15:58:07}
\pmmodified{2013-03-22 15:58:07}
\pmowner{alozano}{2414}
\pmmodifier{alozano}{2414}
\pmtitle{Landau's constant}
\pmrecord{4}{37984}
\pmprivacy{1}
\pmauthor{alozano}{2414}
\pmtype{Definition}
\pmcomment{trigger rebuild}
\pmclassification{msc}{32H02}
\pmrelated{BlochsConstant}

% this is the default PlanetMath preamble.  as your knowledge
% of TeX increases, you will probably want to edit this, but
% it should be fine as is for beginners.

% almost certainly you want these
\usepackage{amssymb}
\usepackage{amsmath}
\usepackage{amsthm}
\usepackage{amsfonts}

% used for TeXing text within eps files
%\usepackage{psfrag}
% need this for including graphics (\includegraphics)
%\usepackage{graphicx}
% for neatly defining theorems and propositions
%\usepackage{amsthm}
% making logically defined graphics
%%%\usepackage{xypic}

% there are many more packages, add them here as you need them

% define commands here

\newtheorem{thm}{Theorem}
\newtheorem*{defn}{Definition}
\newtheorem{prop}{Proposition}
\newtheorem{lemma}{Lemma}
\newtheorem{cor}{Corollary}

\theoremstyle{definition}
\newtheorem{exa}{Example}

% Some sets
\newcommand{\Nats}{\mathbb{N}}
\newcommand{\Ints}{\mathbb{Z}}
\newcommand{\Reals}{\mathbb{R}}
\newcommand{\Complex}{\mathbb{C}}
\newcommand{\Rats}{\mathbb{Q}}
\newcommand{\Gal}{\operatorname{Gal}}
\newcommand{\Cl}{\operatorname{Cl}}
\begin{document}
We suggest that the reader reads first the entry on Bloch's constant. Let $\mathcal{F}$ be the set of all functions $f$ holomorphic on a region containing the
closure of the disk $D=\{z\in\mathbb{C}:|z|<1\}$ and satisfying
$f(0)=0$ and $f'(0)=1$. For each $f\in\mathcal{F}$ let $\lambda(f)$
be the supremum of all numbers $r$ such that there is a disk
$S\subset D$ such that $f(S)$ contains a disk of radius $r$ (notice that here we don't require $f$ to be injective on $S$).

\begin{defn}
Landau's constant $L$ is defined by 
$$L=\inf \{ \lambda(f) : f\in \mathcal{F}\}.$$
\end{defn}

Let $B$ be Bloch's constant. Then, clearly, $L\geq B$. The exact value of $L$ (as that of $B$) is not known but it has been shown that $0.5 \leq L \leq 0.56$. In particular, it is known that $L$ is strictly greater than $B$.

\begin{thebibliography}{00}

\bibitem{conway} John B. Conway, {\em Functions of One Complex
Variable I}, Second Edition, 1978, Springer-Verlag, New York.

\end{thebibliography}
%%%%%
%%%%%
\end{document}
