\documentclass[12pt]{article}
\usepackage{pmmeta}
\pmcanonicalname{ProofOfHartogsTheorem}
\pmcreated{2013-03-22 17:46:48}
\pmmodified{2013-03-22 17:46:48}
\pmowner{jirka}{4157}
\pmmodifier{jirka}{4157}
\pmtitle{proof of Hartogs' theorem}
\pmrecord{5}{40238}
\pmprivacy{1}
\pmauthor{jirka}{4157}
\pmtype{Proof}
\pmcomment{trigger rebuild}
\pmclassification{msc}{32H02}

% this is the default PlanetMath preamble.  as your knowledge
% of TeX increases, you will probably want to edit this, but
% it should be fine as is for beginners.

% almost certainly you want these
\usepackage{amssymb}
\usepackage{amsmath}
\usepackage{amsfonts}

% used for TeXing text within eps files
%\usepackage{psfrag}
% need this for including graphics (\includegraphics)
%\usepackage{graphicx}
% for neatly defining theorems and propositions
\usepackage{amsthm}
% making logically defined graphics
%%%\usepackage{xypic}

% there are many more packages, add them here as you need them

% define commands here
\theoremstyle{theorem}
\newtheorem*{thm}{Theorem}
\newtheorem*{lemma}{Lemma}
\newtheorem*{conj}{Conjecture}
\newtheorem*{cor}{Corollary}
\newtheorem*{example}{Example}
\newtheorem*{prop}{Proposition}
\theoremstyle{definition}
\newtheorem*{defn}{Definition}
\theoremstyle{remark}
\newtheorem*{rmk}{Remark}

\begin{document}
\begin{lemma} \label{lemma1}
Suppose that $g$ is a smooth differential $(0,1)$-form with compact support
in ${\mathbb C}^n .$  Then there exists a smooth function $\psi$ such that
\begin{equation} \label{dbareq}
\bar{\partial} \psi = g ,
\end{equation}
and $\psi$ has compact support if $n \geq 2.$
\end{lemma}

Let $z = (z_1,\ldots,z_n) \in {\mathbb C}^n$ be our coordinates.
We note that $(0,1)$-form means a differential form given by
\begin{equation*}
\sum_{k=1}^n g_k(z) d \bar{z}_k .
\end{equation*}
The operator $\bar{\partial}$ is the
so-called
d-bar operator and we are looking for a smooth function $\phi$ solving the
equation inhomogeneous $\bar{\partial}$ equation.
It is important that $g$ has compact support, otherwise solutions
to \eqref{dbareq} are much harder to obtain.

\begin{proof}
Written out in detail we can think of $g$ as $n$ different functions
$g_1,\ldots,g_n,$
where are date $g_k$ satisfy the compatibility condition
\begin{equation*}
\frac{\partial g_k}{\partial \bar{z}_l} =
\frac{\partial g_l}{\partial \bar{z}_k} \text{ for all $k$, $l$}.
\end{equation*}
Then we write equation \eqref{dbareq} as
\begin{equation*}
\frac{\partial \psi}{\partial \bar{z}_k} = g_k \text{ for all $k$.}
\end{equation*}
We assume also that $g_k$ have compact support.

This system of equations has a solution (many equations in fact).  We can
obtain an explicit solution as follows.
\begin{equation*}
\psi(z)
=
\frac{1}{2\pi i}
\int_{\mathbb C}
\frac{
 g_1(\zeta,z_2,\ldots,z_n)
}{\zeta - z_1}
d\zeta \wedge d\bar{\zeta} .
\end{equation*}
$\psi$ is smooth by differentiating under the integral.
When $n \geq 2,$
this solution will also have compact support since $g_1$
has compact support and as $z$
tends to infinity $(\zeta,z_2,\ldots,z_n)$ also tends to infinity
no matter what $\zeta$ is.  The reader should notice that there is one
direction which does not work.  But if $\psi$ has bounded support
except for the line defined by $z_2 = z_3 = \cdots = z_n = 0,$ then 
the support must be compact by continuity of $\psi$.
It
should also be clear why $\psi$ does not have compact support
if $n=1.$

One might be wondering why
we picked $z_1$ and $g_1$ in the construction of $\psi.$  It does not matter,
we will get different solutions we we use $z_k$ and $g_k,$ but it will
still have compact support.
Further one might wonder why we only
use one part of the data, and still get an actual solution.
The answer here is that the compatibility condition relates all the data,
so we only need to look at one.


We still must
check that this really is a solution.
We apply the compatibility condition.  Let $k \geq 2$.
\begin{equation*}
\begin{split}
\frac{\partial \psi}{\partial \bar{z}_k}(z_1,z_2,\ldots,z_n)
& =
\frac{1}{2\pi i}
\int_{\mathbb C}
\frac{
 \frac{\partial g_1}{\partial \bar{z}_k}(\zeta,z_2,\ldots,z_n)
}{\zeta - z_1}
d\zeta \wedge d\bar{\zeta}
\\
& =
\frac{1}{2\pi i}
\int_{\mathbb C}
\frac{
 \frac{\partial g_k}{\partial \bar{z}_1}(\zeta,z_2,\ldots,z_n)
}{\zeta - z_1}
d\zeta \wedge d\bar{\zeta} .
\end{split}
\end{equation*}
Note that the integral can be taken over a large ball $B$
that contains
the support of $g_k$.
We apply the generalized Cauchy formula, where the
boundary part of the integral is obviously zero since it is over
a set where $g_k$ is zero.
\begin{equation*}
\begin{split}
\frac{1}{2\pi i}
\int_{\mathbb C}
\frac{
 \frac{\partial g_k}{\partial \bar{z}_1}(\zeta,z_2,\ldots,z_n)
}{\zeta - z_1}
d\zeta \wedge d\bar{\zeta}
& =
\frac{1}{2\pi i}
\int_B
\frac{
 \frac{\partial g_k}{\partial \bar{z}_1}(\zeta,z_2,\ldots,z_n)
}{\zeta - z_1}
d\zeta \wedge d\bar{\zeta}
\\
& =
g_k(z_1,\ldots,z_n) .
\end{split}
\end{equation*}
Hence $\frac{\partial \psi}{\partial \bar{z}_k} = g_k$.

When $k=1$, change coordinates to see that
\begin{equation*}
\frac{1}{2\pi i}
\int_{\mathbb C}
\frac{
 g_1(\zeta,z_2,\ldots,z_n)
}{\zeta - z_1}
d\zeta \wedge d\bar{\zeta}
=
\frac{1}{2\pi i}
\int_{\mathbb C}
\frac{
 g_1(\zeta+z_1,z_2,\ldots,z_n)
}{\zeta}
d\zeta \wedge d\bar{\zeta} .
\end{equation*}
Next differentiate in $\bar{z}_k$ and change coordinates back and apply
the generalized Cauchy formula as before to get that
$\frac{\partial \psi}{\partial \bar{z}_1} = g_1 .$
\end{proof}

\begin{proof}[Proof of Theorem]
Let $U \subset {\mathbb C}^n$, $K$ a compact subset of $U$ and $f$ be a
holomorphic function defined on $U \setminus K$ and $U \setminus K$ to
be connected.
By the smooth version of Urysohn's lemma we can
find a smooth function $\varphi$ which is 1 in a neighbourhood of
$K$ and is compactly supported in $U.$  Let
$f_0 := (1-\varphi)f,$ which is identically zero on $K$ and holomorphic
near the boundary of $U$ (since there $\varphi$ is 0).
We let $g = \bar{\partial} f_0$, that is $g_k = \frac{\partial
f_0}{\partial \bar{z}_k}$.  Let us see why $g_k$ is compactly supported.  The
only place to check is on $U \setminus K$ as elsewhere we have 0
automatically,
\begin{equation*}
\frac{\partial f_0}{\partial \bar{z}_k}
=
\frac{\partial }{\partial \bar{z}_k}
((1-\varphi)f)
=
- f \frac{\partial \varphi}{\partial \bar{z}_k} .
\end{equation*}
By Lemma \ref{lemma1} we find a compactly supported solution $\psi$
to $\bar{\partial}\psi = g$.

Set $\tilde{f} := f_0 - \psi$.  Let us check that this is the desired
extention.  Firstly let us check it is holomorphic,
\begin{equation*}
\frac{\partial \tilde{f}}{\partial \bar{z}_k}
=
\frac{\partial f_0}{\partial \bar{z}_k}
-
\frac{\partial \psi}{\partial \bar{z}_k}
=
g_k
-
g_k
= 0 .
\end{equation*}

It is not hard to see that $\psi$ is compactly supported in $U.$  This follows
by the fact that $U \setminus K$ is connected and the fact
that $\psi$
is holomorphic on the set where $g$ is identically zero.  By unique continuation
of holomorphic functions, support of $\psi$ is no larger than that of $g.$
\end{proof}

%%%%%
%%%%%
\end{document}
