\documentclass[12pt]{article}
\usepackage{pmmeta}
\pmcanonicalname{PicardsTheorem}
\pmcreated{2013-03-22 13:15:23}
\pmmodified{2013-03-22 13:15:23}
\pmowner{Koro}{127}
\pmmodifier{Koro}{127}
\pmtitle{Picard's theorem}
\pmrecord{9}{33735}
\pmprivacy{1}
\pmauthor{Koro}{127}
\pmtype{Theorem}
\pmcomment{trigger rebuild}
\pmclassification{msc}{32H25}
\pmsynonym{great Picard theorem}{PicardsTheorem}
\pmrelated{EssentialSingularity}
\pmrelated{CasoratiWeierstrassTheorem}
\pmrelated{ProofOfCasoratiWeierstrassTheorem}

% this is the default PlanetMath preamble.  as your knowledge
% of TeX increases, you will probably want to edit this, but
% it should be fine as is for beginners.

% almost certainly you want these
\usepackage{amssymb}
\usepackage{amsmath}
\usepackage{amsfonts}

% used for TeXing text within eps files
%\usepackage{psfrag}
% need this for including graphics (\includegraphics)
%\usepackage{graphicx}
% for neatly defining theorems and propositions
%\usepackage{amsthm}
% making logically defined graphics
%%%\usepackage{xypic}

% there are many more packages, add them here as you need them

% define commands here
\begin{document}
Let $f$ be an holomorphic function with an essential singularity at $w\in \mathbb{C}$. Then there is a number $z_0\in \mathbb{C}$ such that the image of any neighborhood of $w$ by $f$ contains $\mathbb{C}-\{z_0\}$. In other words, $f$ assumes every complex value, with the possible exception of $z_0$, in any neighborhood of $w$.

\emph{Remark.} Little Picard theorem follows as a corollary:
Given a nonconstant entire function $f$, if it is a polynomial, it assumes every value in $\mathbb{C}$ as a consequence of the fundamental theorem of algebra. If $f$ is not a polynomial, then $g(z)=f(1/z)$ has an essential singularity at $0$; Picard's theorem implies that $g$ (and thus $f$) assumes every complex value, with one possible exception.
%%%%%
%%%%%
\end{document}
