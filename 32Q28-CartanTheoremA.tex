\documentclass[12pt]{article}
\usepackage{pmmeta}
\pmcanonicalname{CartanTheoremA}
\pmcreated{2013-03-22 17:39:10}
\pmmodified{2013-03-22 17:39:10}
\pmowner{jirka}{4157}
\pmmodifier{jirka}{4157}
\pmtitle{Cartan theorem A}
\pmrecord{6}{40086}
\pmprivacy{1}
\pmauthor{jirka}{4157}
\pmtype{Theorem}
\pmcomment{trigger rebuild}
\pmclassification{msc}{32Q28}
\pmclassification{msc}{32C35}
\pmsynonym{Cartan's theorem A}{CartanTheoremA}
\pmrelated{CartanTheoremB}

% this is the default PlanetMath preamble.  as your knowledge
% of TeX increases, you will probably want to edit this, but
% it should be fine as is for beginners.

% almost certainly you want these
\usepackage{amssymb}
\usepackage{amsmath}
\usepackage{amsfonts}

% used for TeXing text within eps files
%\usepackage{psfrag}
% need this for including graphics (\includegraphics)
%\usepackage{graphicx}
% for neatly defining theorems and propositions
\usepackage{amsthm}
% making logically defined graphics
%%%\usepackage{xypic}

% there are many more packages, add them here as you need them

% define commands here
\theoremstyle{theorem}
\newtheorem*{thm}{Theorem}
\newtheorem*{lemma}{Lemma}
\newtheorem*{conj}{Conjecture}
\newtheorem*{cor}{Corollary}
\newtheorem*{example}{Example}
\newtheorem*{prop}{Proposition}
\theoremstyle{definition}
\newtheorem*{defn}{Definition}
\theoremstyle{remark}
\newtheorem*{rmk}{Remark}

\begin{document}
Let $\mathcal{O}_z$ denote the ring of germs of
holomorphic functions at $z$

\begin{thm}[Cartan]
Suppose $\mathcal{F}$ is a coherent analytic sheaf on a Stein
manifold $M$.
For every $z \in M$, the the stalk ${\mathcal{F}}_z$
is generated as an ${\mathcal{O}}_z$ module by the germs at $z$
of the \PMlinkname{sections}{Sheaf} $\Gamma(M,\mathcal{F})$.
\end{thm}

Philosophically, this theorem says that there is good supply of \PMlinkescapetext{global
sections} of a coherent analytic sheaf on a Stein manifold.


\begin{thebibliography}{9}
\bibitem{Hormander:several}
Lars H\"ormander.
{\em \PMlinkescapetext{An Introduction to Complex Analysis in Several
Variables}},
North-Holland Publishing Company, New York, New York, 1973.
\bibitem{Krantz:several}
Steven~G.\@ Krantz.
{\em \PMlinkescapetext{Function Theory of Several Complex Variables}},
AMS Chelsea Publishing, Providence, Rhode Island, 1992.
\end{thebibliography}
%%%%%
%%%%%
\end{document}
