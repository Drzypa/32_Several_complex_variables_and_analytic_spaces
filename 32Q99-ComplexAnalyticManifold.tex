\documentclass[12pt]{article}
\usepackage{pmmeta}
\pmcanonicalname{ComplexAnalyticManifold}
\pmcreated{2013-03-22 15:04:40}
\pmmodified{2013-03-22 15:04:40}
\pmowner{jirka}{4157}
\pmmodifier{jirka}{4157}
\pmtitle{complex analytic manifold}
\pmrecord{6}{36800}
\pmprivacy{1}
\pmauthor{jirka}{4157}
\pmtype{Definition}
\pmcomment{trigger rebuild}
\pmclassification{msc}{32Q99}
\pmsynonym{complex manifold}{ComplexAnalyticManifold}
\pmdefines{complex analytic submanifold}
\pmdefines{complex submanifold}
\pmdefines{analytic structure}
\pmdefines{holomorphic structure}

\endmetadata

% this is the default PlanetMath preamble.  as your knowledge
% of TeX increases, you will probably want to edit this, but
% it should be fine as is for beginners.

% almost certainly you want these
\usepackage{amssymb}
\usepackage{amsmath}
\usepackage{amsfonts}

% used for TeXing text within eps files
%\usepackage{psfrag}
% need this for including graphics (\includegraphics)
%\usepackage{graphicx}
% for neatly defining theorems and propositions
\usepackage{amsthm}
% making logically defined graphics
%%%\usepackage{xypic}

% there are many more packages, add them here as you need them

% define commands here
\theoremstyle{theorem}
\newtheorem*{thm}{Theorem}
\newtheorem*{lemma}{Lemma}
\newtheorem*{conj}{Conjecture}
\newtheorem*{cor}{Corollary}
\theoremstyle{definition}
\newtheorem*{defn}{Definition}
\begin{document}
\begin{defn}
A manifold $M$ is called a {\em complex analytic manifold} (or sometimes just
a {\em complex manifold}) if the transition functions are holomorphic.
\end{defn}

\begin{defn}
A subset $N \subset M$ is called a {\em complex analytic submanifold} of $M$
if $N$ is closed in $M$ and if for every point $z \in N$ there is a coordinate neighbourhood $U$ in $M$ with coordinates $z_1,\ldots,z_n$ such that
$U \cap N = \{ p \in U \mid z_{d+1}(p) = \ldots = z_n(p) \}$ for some integer $d \leq n$. 
\end{defn}

Obviously $N$ is now also a complex analytic manifold itself.

For a complex analytic manifold, dimension always means the complex dimension,
not the real dimension.  That is $M$ is of dimension $n$ when there are neighbourhoods of every point homeomorphic to ${\mathbb{C}}^n$.  Such a manifold is of real dimension $2n$ if we identify ${\mathbb{C}}^n$ with
${\mathbb{R}}^{2n}$.
Of course the tangent bundle is now also a complex vector space.

A function $f$ is said to be holomorphic on $M$ if it is a holomorphic function when considered as a function of the local coordinates.

Examples of complex analytic manifolds are for example the Stein manifolds or the Riemann surfaces.  Of course also any open set in ${\mathbb{C}}^n$ is also a complex analytic manifold.  Another example may be the set of regular points of an analytic set.

Complex analytic manifolds can also be considered as a special case of CR manifolds where the CR dimension is maximal.

Complex manifolds are sometimes described as manifolds carrying an {\em \PMlinkescapetext{analytic structure}} or {\em \PMlinkescapetext{holomorphic structure}}.  This refers to the atlas and transition functions defined on the manifold.

\begin{thebibliography}{9}
\bibitem{Hormander:several}
Lars H\"ormander.
{\em \PMlinkescapetext{An Introduction to Complex Analysis in Several
Variables}},
North-Holland Publishing Company, New York, New York, 1973.
\bibitem{Krantz:several}
Steven~G.\@ Krantz.
{\em \PMlinkescapetext{Function Theory of Several Complex Variables}},
AMS Chelsea Publishing, Providence, Rhode Island, 1992.
\end{thebibliography}
%%%%%
%%%%%
\end{document}
