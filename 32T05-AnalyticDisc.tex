\documentclass[12pt]{article}
\usepackage{pmmeta}
\pmcanonicalname{AnalyticDisc}
\pmcreated{2013-03-22 14:30:49}
\pmmodified{2013-03-22 14:30:49}
\pmowner{jirka}{4157}
\pmmodifier{jirka}{4157}
\pmtitle{analytic disc}
\pmrecord{8}{36052}
\pmprivacy{1}
\pmauthor{jirka}{4157}
\pmtype{Definition}
\pmcomment{trigger rebuild}
\pmclassification{msc}{32T05}
\pmdefines{closed analytic disc}
\pmdefines{boundary of a closed analytic disc}
\pmdefines{attached analytic disc}

% this is the default PlanetMath preamble.  as your knowledge
% of TeX increases, you will probably want to edit this, but
% it should be fine as is for beginners.

% almost certainly you want these
\usepackage{amssymb}
\usepackage{amsmath}
\usepackage{amsfonts}

% used for TeXing text within eps files
%\usepackage{psfrag}
% need this for including graphics (\includegraphics)
%\usepackage{graphicx}
% for neatly defining theorems and propositions
\usepackage{amsthm}
% making logically defined graphics
%%%\usepackage{xypic}

% there are many more packages, add them here as you need them

% define commands here
\theoremstyle{theorem}
\newtheorem*{thm}{Theorem}
\newtheorem*{lemma}{Lemma}
\newtheorem*{conj}{Conjecture}
\newtheorem*{cor}{Corollary}
\theoremstyle{definition}
\newtheorem*{defn}{Definition}
\begin{document}
\begin{defn}
Let $D := \{ z \in {\mathbb{C}} \mid \lvert z \rvert < 1 \}$ be the
open unit disc.  A non-constant holomorphic mapping $\varphi \colon D \to
{\mathbb{C}}^n$ is called an {\em analytic disc} in ${\mathbb{C}}^n$.  The
\PMlinkescapetext{term} really refers to both the embedding and the image.
If the mapping $\varphi$ extends continuously to the closed unit disc
$\bar{D}$, then $\varphi(\bar{D})$ is called a {\em closed analytic disc}
and $\varphi(\partial D)$ is called the {\em boundary of a closed analytic
disc}.
\end{defn}

Analytic discs play in some sense a role of line segments in ${\mathbb{C}}^n$.
For example they give another way to see that a domain
$G \subset {\mathbb{C}}^n$ is pseudoconvex.  See the Hartogs Kontinuitatssatz
theorem.

Another use of analytic discs are as a technique for extending CR functions on generic manifolds \cite{ber:submanifold}.  The idea here is that you can always extend a function from the boundary of a disc to the inside of the disc by solving the Dirichlet problem.

\begin{defn}
A closed analytic disc $\varphi$ is said to be {\em attached}
to a set $M \subset
{\mathbb{C}}^n$ if $\varphi(\partial D) \subset M$, that is if $\varphi$
maps the boundary of the unit disc to $M$.
\end{defn}

Analytic discs are also used for defining the Kobayashi metric and thus plays a role in the study of invariant metrics.

\begin{thebibliography}{9}
\bibitem{ber:submanifold}
M.\@ Salah Baouendi,
Peter Ebenfelt,
Linda Preiss Rothschild.
{\em \PMlinkescapetext{Real Submanifolds in Complex Space and Their Mappings}},
Princeton University Press,
Princeton, New Jersey, 1999.
\bibitem{Krantz:several}
Steven~G.\@ Krantz.
{\em \PMlinkescapetext{Function Theory of Several Complex Variables}},
AMS Chelsea Publishing, Providence, Rhode Island, 1992.
\end{thebibliography}
%%%%%
%%%%%
\end{document}
