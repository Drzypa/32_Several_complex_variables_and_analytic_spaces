\documentclass[12pt]{article}
\usepackage{pmmeta}
\pmcanonicalname{AnalyticPolyhedron}
\pmcreated{2013-03-22 14:32:39}
\pmmodified{2013-03-22 14:32:39}
\pmowner{jirka}{4157}
\pmmodifier{jirka}{4157}
\pmtitle{analytic polyhedron}
\pmrecord{7}{36091}
\pmprivacy{1}
\pmauthor{jirka}{4157}
\pmtype{Definition}
\pmcomment{trigger rebuild}
\pmclassification{msc}{32T05}
\pmclassification{msc}{32A07}
\pmsynonym{analytic polyhedra}{AnalyticPolyhedron}
\pmdefines{frame of an analytic polyhedron}

\endmetadata

% this is the default PlanetMath preamble.  as your knowledge
% of TeX increases, you will probably want to edit this, but
% it should be fine as is for beginners.

% almost certainly you want these
\usepackage{amssymb}
\usepackage{amsmath}
\usepackage{amsfonts}

% used for TeXing text within eps files
%\usepackage{psfrag}
% need this for including graphics (\includegraphics)
%\usepackage{graphicx}
% for neatly defining theorems and propositions
\usepackage{amsthm}
% making logically defined graphics
%%%\usepackage{xypic}

% there are many more packages, add them here as you need them

% define commands here
\theoremstyle{theorem}
\newtheorem*{thm}{Theorem}
\newtheorem*{lemma}{Lemma}
\newtheorem*{conj}{Conjecture}
\newtheorem*{cor}{Corollary}
\theoremstyle{definition}
\newtheorem*{defn}{Definition}
\begin{document}
\begin{defn}
Suppose $G \subset {\mathbb{C}}^n$ is a domain and let $W \subset G$
be an open set.  Let $f_1,\ldots,f_k \colon W \to {\mathbb{C}}$ be holomorphic 
functions.  Then if the set
\begin{equation*}
\Omega := \{ z \in W \mid \lvert f_j (z) \rvert < 1 , j=1,\ldots,k \}
\end{equation*}
is relatively compact in $W$, we say that $\Omega$ is an
{\em analytic polyhedron} in $G$.  Sometimes it is denoted
$\Omega(f_1,\ldots,f_k)$.  Further $(W,f_1,\ldots,f_k)$ is called the
{\em \PMlinkescapetext{frame} of the analytic polyhedron}.
\end{defn}

An analytic polyhedron is automatically a domain of holomorphy by using
the functions that define it as $g(z) := \frac{1}{e^{i\theta}-f_j(z)}$ to
show that $g$ cannot be extended beyond a point where $f_j(z) = e^{i\theta}$.
Every boundary point of $\Omega$ is of that form for some $f_j$.

Furthermore every domain of holomorphy can be exhausted by analytic polyhedra (that is, every compact subset is contained in an analytic polyhedron)
and in fact only domains of holomorphy can be exhausted by analytic
polyhedra, see the Behnke-Stein theorem.

Note that sometimes $W$ is required to be homeomorphic to the unit ball.

\begin{thebibliography}{9}
\bibitem{Hormander:several}
Lars H\"ormander.
{\em \PMlinkescapetext{An Introduction to Complex Analysis in Several
Variables}},
North-Holland Publishing Company, New York, New York, 1973.
\bibitem{Krantz:several}
Steven~G.\@ Krantz.
{\em \PMlinkescapetext{Function Theory of Several Complex Variables}},
AMS Chelsea Publishing, Providence, Rhode Island, 1992.
\end{thebibliography}
%%%%%
%%%%%
\end{document}
