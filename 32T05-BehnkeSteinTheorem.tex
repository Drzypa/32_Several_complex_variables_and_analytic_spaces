\documentclass[12pt]{article}
\usepackage{pmmeta}
\pmcanonicalname{BehnkeSteinTheorem}
\pmcreated{2013-03-22 14:31:13}
\pmmodified{2013-03-22 14:31:13}
\pmowner{jirka}{4157}
\pmmodifier{jirka}{4157}
\pmtitle{Behnke-Stein theorem}
\pmrecord{5}{36061}
\pmprivacy{1}
\pmauthor{jirka}{4157}
\pmtype{Theorem}
\pmcomment{trigger rebuild}
\pmclassification{msc}{32T05}
\pmsynonym{increasing union of domains of holomorphy is a domain of holomorphy}{BehnkeSteinTheorem}

\endmetadata

% this is the default PlanetMath preamble.  as your knowledge
% of TeX increases, you will probably want to edit this, but
% it should be fine as is for beginners.

% almost certainly you want these
\usepackage{amssymb}
\usepackage{amsmath}
\usepackage{amsfonts}

% used for TeXing text within eps files
%\usepackage{psfrag}
% need this for including graphics (\includegraphics)
%\usepackage{graphicx}
% for neatly defining theorems and propositions
\usepackage{amsthm}
% making logically defined graphics
%%%\usepackage{xypic}

% there are many more packages, add them here as you need them

% define commands here
\theoremstyle{theorem}
\newtheorem*{thm}{Theorem}
\newtheorem*{lemma}{Lemma}
\newtheorem*{conj}{Conjecture}
\newtheorem*{cor}{Corollary}
\theoremstyle{definition}
\newtheorem*{defn}{Definition}
\begin{document}
\begin{thm}[Behnke-Stein]
Suppose that $G_k \subset {\mathbb{C}}^n$ are domains of holomorphy
such that $G_k \subset G_j$ whenever $k < j$.  Then
$\bigcup_{k=1}^\infty G_k$ is a domain of holomorphy.
\end{thm}

This is related to the fact that an increasing union of pseudoconvex domains is pseudoconvex and so it can be proven using that fact and the solution of the
Levi problem.  Though historically this theorem was in fact used to solve
the Levi problem.

\begin{thebibliography}{9}
\bibitem{Krantz:several}
Steven~G.\@ Krantz.
{\em \PMlinkescapetext{Function Theory of Several Complex Variables}},
AMS Chelsea Publishing, Providence, Rhode Island, 1992.
\end{thebibliography}
%%%%%
%%%%%
\end{document}
