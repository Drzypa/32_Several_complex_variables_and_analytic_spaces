\documentclass[12pt]{article}
\usepackage{pmmeta}
\pmcanonicalname{DomainOfHolomorphy}
\pmcreated{2013-03-22 14:29:29}
\pmmodified{2013-03-22 14:29:29}
\pmowner{jirka}{4157}
\pmmodifier{jirka}{4157}
\pmtitle{domain of holomorphy}
\pmrecord{7}{36026}
\pmprivacy{1}
\pmauthor{jirka}{4157}
\pmtype{Definition}
\pmcomment{trigger rebuild}
\pmclassification{msc}{32T05}
\pmclassification{msc}{32A10}
\pmrelated{LeviPseudoconvex}
\pmrelated{SolutionOfTheLeviProblem}
\pmrelated{SteinManifold}

% this is the default PlanetMath preamble.  as your knowledge
% of TeX increases, you will probably want to edit this, but
% it should be fine as is for beginners.

% almost certainly you want these
\usepackage{amssymb}
\usepackage{amsmath}
\usepackage{amsfonts}

% used for TeXing text within eps files
%\usepackage{psfrag}
% need this for including graphics (\includegraphics)
%\usepackage{graphicx}
% for neatly defining theorems and propositions
\usepackage{amsthm}
% making logically defined graphics
%%%\usepackage{xypic}

% there are many more packages, add them here as you need them

% define commands here
\theoremstyle{theorem}
\newtheorem*{thm}{Theorem}
\newtheorem*{lemma}{Lemma}
\newtheorem*{conj}{Conjecture}
\newtheorem*{cor}{Corollary}
\newtheorem*{example}{Example}
\theoremstyle{definition}
\newtheorem*{defn}{Definition}
\begin{document}
\begin{defn}
An open set $\Omega \subset {\mathbb{C}}^n$ is called a {\em domain of holomorphy}
if there do not exist non-empty open sets $U \subset \Omega$ and $V \subset {\mathbb{C}}^n$ where $V$ is connected, $V \not\subset \Omega$ and $U \subset \Omega \cap V$ such that for every holomorphic function $f$ on $\Omega$ there exists
a holomorphic function $g$ on $V$ such that $f = g$ on $U.$
\end{defn}

When $n=1$, then every open set is a domain of holomorphy.  For an example, assume that the boundary of $\Omega \subset {\mathbb{C}}$ is a Jordan curve for simplicity.  We can define a holomorphic function which has zeros which accumulate on the boundary of the domain and thus the function cannot be continued past any point in the boundary.
If you could extend the function, it would be identically zero.

Alternatively given any open set $\Omega \subset \mathbb{C}$ and any point $p \in \partial \Omega,$ the function $z \mapsto \frac{1}{z-p}$ is holomorphic in $\Omega$, but cannot be continued past $p$.

For $n \geq 2$ many domains are not domains of holomorphy.  For example if you take ${\mathbb{C}}^2 \setminus \{0\},$
this is no longer a domain of holomorphy by \PMlinkname{Hartogs's theorem}{HartogsTheorem}.  It turns out that a domain is a domain of holomorphy if and only if the boundary is pseudoconvex.  In particular, every convex (in the classical sense)
domain is a domain of holomorphy.  \PMlinkescapetext{Simple} examples of domains of holomorphy are ${\mathbb{C}}^n,$ an open ball, or a polydisc.

\begin{thebibliography}{9}
\bibitem{Hormander:several}
Lars H\"ormander.
{\em \PMlinkescapetext{An Introduction to Complex Analysis in Several
Variables}},
North-Holland Publishing Company, New York, New York, 1973.
\bibitem{Krantz:several}
Steven~G.\@ Krantz.
{\em \PMlinkescapetext{Function Theory of Several Complex Variables}},
AMS Chelsea Publishing, Providence, Rhode Island, 1992.
\end{thebibliography}
%%%%%
%%%%%
\end{document}
