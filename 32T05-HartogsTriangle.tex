\documentclass[12pt]{article}
\usepackage{pmmeta}
\pmcanonicalname{HartogsTriangle}
\pmcreated{2013-03-22 14:31:08}
\pmmodified{2013-03-22 14:31:08}
\pmowner{jirka}{4157}
\pmmodifier{jirka}{4157}
\pmtitle{Hartogs triangle}
\pmrecord{5}{36059}
\pmprivacy{1}
\pmauthor{jirka}{4157}
\pmtype{Example}
\pmcomment{trigger rebuild}
\pmclassification{msc}{32T05}

% this is the default PlanetMath preamble.  as your knowledge
% of TeX increases, you will probably want to edit this, but
% it should be fine as is for beginners.

% almost certainly you want these
\usepackage{amssymb}
\usepackage{amsmath}
\usepackage{amsfonts}

% used for TeXing text within eps files
%\usepackage{psfrag}
% need this for including graphics (\includegraphics)
\usepackage{graphicx}
% for neatly defining theorems and propositions
%\usepackage{amsthm}
% making logically defined graphics
%%%\usepackage{xypic}

% there are many more packages, add them here as you need them

% define commands here
\begin{document}
A non-trivial example of domain of holomorphy that has some
interesting 
non-obvious properties is the {\em Hartogs triangle} which is the set
\begin{equation*}
\{ (z,w) \in {\mathbb{C}}^2 \mid \lvert z \rvert < \lvert w \rvert < 1 \} .
\end{equation*}
Since it is a Reinhardt domain it can be represented
by plotting it on the plane $\lvert z \rvert \times \lvert w \rvert$
as follows.

\begin{center}
\includegraphics[scale=1.0]{hartogstriangle.eps}
\vspace*{0.1in}

{\tiny Figure 1: Hartogs triangle}
\end{center}

It is obvious then where the name comes from.  To see that this is a domain of holomorphy, then given a boundary point we wish to exhibit a holomorphic function on the whole Hartogs triangle which does not extend beyond that point.  First note that on the top boundary $z$ is anything and $w = e^{i\theta}$ for some $\theta$, so
$f(z,w) = \frac{1}{w-e^{i\theta}}$ will not extend beyond $(z,e^{i\theta})$.
Now for the diagonal boundary this is where $\lvert z \rvert = \lvert w \rvert$,
that is $z = e^{i\theta} w$ for some $\theta$, so 
$f(z,w) = \frac{1}{z-e^{i\theta}w}$ will do not extend beyond $(e^{i\theta}w,w)$.

One of the many properties of this domain is that if $U$ is the Hartogs
triangle, then it is a domain of holomorphy, but if we take a sufficently
small neighbourhood $V$ of $\bar{U}$ (the closure of $U$),
then any function holomorphic on $V$ is holomorphic on the polydisc
$D^2(0,1)$ (just fill in everything below the triangle in Figure 1).  So if $V$ does not include all of $D^2(0,1)$ then it is not a domain of
holomorphy.  This is because a Reinhardt domain that contains zero (the point
$(0,0)$) is a domain
of holomorphy if and only if it is a logarithmically convex set and any neighbourhood of $\bar{U}$ does contain zero while $U$ itself does not.

\begin{thebibliography}{9}
\bibitem{Krantz:several}
Steven~G.\@ Krantz.
{\em \PMlinkescapetext{Function Theory of Several Complex Variables}},
AMS Chelsea Publishing, Providence, Rhode Island, 1992.
\end{thebibliography}
%%%%%
%%%%%
\end{document}
