\documentclass[12pt]{article}
\usepackage{pmmeta}
\pmcanonicalname{Kontinuitatssatz}
\pmcreated{2013-03-22 15:49:15}
\pmmodified{2013-03-22 15:49:15}
\pmowner{jirka}{4157}
\pmmodifier{jirka}{4157}
\pmtitle{Kontinuit\"atssatz}
\pmrecord{5}{37787}
\pmprivacy{1}
\pmauthor{jirka}{4157}
\pmtype{Theorem}
\pmcomment{trigger rebuild}
\pmclassification{msc}{32T05}
\pmsynonym{Hartogs Kontinuit\"atssatz}{Kontinuitatssatz}
\pmsynonym{Kontinuitatssatz}{Kontinuitatssatz}

% this is the default PlanetMath preamble.  as your knowledge
% of TeX increases, you will probably want to edit this, but
% it should be fine as is for beginners.

% almost certainly you want these
\usepackage{amssymb}
\usepackage{amsmath}
\usepackage{amsfonts}

% used for TeXing text within eps files
%\usepackage{psfrag}
% need this for including graphics (\includegraphics)
%\usepackage{graphicx}
% for neatly defining theorems and propositions
\usepackage{amsthm}
% making logically defined graphics
%%%\usepackage{xypic}

% there are many more packages, add them here as you need them

% define commands here
\theoremstyle{theorem}
\newtheorem*{thm}{Theorem}
\newtheorem*{lemma}{Lemma}
\newtheorem*{conj}{Conjecture}
\newtheorem*{cor}{Corollary}
\newtheorem*{example}{Example}
\newtheorem*{prop}{Proposition}
\theoremstyle{definition}
\newtheorem*{defn}{Definition}
\theoremstyle{remark}
\newtheorem*{rmk}{Remark}
\begin{document}
\begin{thm}
$G \subset {\mathbb{C}}^n$ is pseudoconvex if and only if
for any family of closed analytic discs $\{ d_{\alpha} \}_{\alpha \in I}$
in $G$ with $\cup_{\alpha \in I} \partial d_\alpha$ being
a relatively compact set in $G$
then $\cup_{\alpha \in I} d_\alpha$ is also
a relatively compact set in $G$.
\end{thm}

This is the analogue of one of the definitions of a convex set. Just replace
pseudoconvex with convex and closed analytic discs with closed line segments.

\begin{thebibliography}{9}
\bibitem{ber:submanifold}
M.\@ Salah Baouendi,
Peter Ebenfelt,
Linda Preiss Rothschild.
{\em \PMlinkescapetext{Real Submanifolds in Complex Space and Their Mappings}},
Princeton University Press,
Princeton, New Jersey, 1999.
\bibitem{Krantz:several}
Steven~G.\@ Krantz.
{\em \PMlinkescapetext{Function Theory of Several Complex Variables}},
AMS Chelsea Publishing, Providence, Rhode Island, 1992.
\end{thebibliography}
%%%%%
%%%%%
\end{document}
