\documentclass[12pt]{article}
\usepackage{pmmeta}
\pmcanonicalname{SolutionOfTheLeviProblem}
\pmcreated{2013-03-22 14:31:11}
\pmmodified{2013-03-22 14:31:11}
\pmowner{jirka}{4157}
\pmmodifier{jirka}{4157}
\pmtitle{solution of the Levi problem}
\pmrecord{6}{36060}
\pmprivacy{1}
\pmauthor{jirka}{4157}
\pmtype{Theorem}
\pmcomment{trigger rebuild}
\pmclassification{msc}{32T05}
\pmclassification{msc}{32E40}
\pmrelated{Pseudoconvex}
\pmrelated{DomainOfHolomorphy}
\pmdefines{Levi problem}

\endmetadata

% this is the default PlanetMath preamble.  as your knowledge
% of TeX increases, you will probably want to edit this, but
% it should be fine as is for beginners.

% almost certainly you want these
\usepackage{amssymb}
\usepackage{amsmath}
\usepackage{amsfonts}

% used for TeXing text within eps files
%\usepackage{psfrag}
% need this for including graphics (\includegraphics)
%\usepackage{graphicx}
% for neatly defining theorems and propositions
\usepackage{amsthm}
% making logically defined graphics
%%%\usepackage{xypic}

% there are many more packages, add them here as you need them

% define commands here
\theoremstyle{theorem}
\newtheorem*{thm}{Theorem}
\newtheorem*{lemma}{Lemma}
\newtheorem*{conj}{Conjecture}
\newtheorem*{cor}{Corollary}
\newtheorem*{example}{Example}
\newtheorem*{prop}{Proposition}
\theoremstyle{definition}
\newtheorem*{defn}{Definition}
\begin{document}
The {\em Levi problem} is the problem of characterizing domains of
holomorphy by a local condition on the boundary that does not involve
holomorphic functions themselves.  This condition turned out to
be pseudoconvexity.

\begin{thm}
An open set $G \subset {\mathbb{C}}^n$ is a domain of holomorphy if
and only if $G$ is pseudoconvex.
\end{thm}

The forward direction (domain of holomorphy implies pseudoconvexity) is
not hard to prove and was known for a long time.  The opposite direction
is really what's called the solution to the Levi problem.

\begin{thebibliography}{9}
\bibitem{Krantz:several}
Steven~G.\@ Krantz.
{\em \PMlinkescapetext{Function Theory of Several Complex Variables}},
AMS Chelsea Publishing, Providence, Rhode Island, 1992.
\end{thebibliography}
%%%%%
%%%%%
\end{document}
