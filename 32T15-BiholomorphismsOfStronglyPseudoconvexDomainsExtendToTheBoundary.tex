\documentclass[12pt]{article}
\usepackage{pmmeta}
\pmcanonicalname{BiholomorphismsOfStronglyPseudoconvexDomainsExtendToTheBoundary}
\pmcreated{2013-03-22 16:44:30}
\pmmodified{2013-03-22 16:44:30}
\pmowner{jirka}{4157}
\pmmodifier{jirka}{4157}
\pmtitle{biholomorphisms of strongly pseudoconvex domains extend to the boundary}
\pmrecord{4}{38965}
\pmprivacy{1}
\pmauthor{jirka}{4157}
\pmtype{Theorem}
\pmcomment{trigger rebuild}
\pmclassification{msc}{32T15}
\pmrelated{LeviPseudoconvex}

\endmetadata

% this is the default PlanetMath preamble.  as your knowledge
% of TeX increases, you will probably want to edit this, but
% it should be fine as is for beginners.

% almost certainly you want these
\usepackage{amssymb}
\usepackage{amsmath}
\usepackage{amsfonts}

% used for TeXing text within eps files
%\usepackage{psfrag}
% need this for including graphics (\includegraphics)
%\usepackage{graphicx}
% for neatly defining theorems and propositions
\usepackage{amsthm}
% making logically defined graphics
%%%\usepackage{xypic}

% there are many more packages, add them here as you need them

% define commands here
\theoremstyle{theorem}
\newtheorem*{thm}{Theorem}
\newtheorem*{lemma}{Lemma}
\newtheorem*{conj}{Conjecture}
\newtheorem*{cor}{Corollary}
\newtheorem*{example}{Example}
\newtheorem*{prop}{Proposition}
\theoremstyle{definition}
\newtheorem*{defn}{Definition}
\theoremstyle{remark}
\newtheorem*{rmk}{Remark}

\begin{document}
It is a basic question in complex analysis to ask when does a biholomorphic mapping of two domains extend to the boundary.  The following is a celebrated theorem of Fefferman for strongly pseudoconvex domains.

\begin{thm}[Fefferman]
Let $U, V \subset {\mathbb{C}}^n$ ($n \geq 2$) be two strongly pseudoconvex domains with smooth ($C^\infty$) boundaries (the boundaries are smooth submanifolds).  Let $f \colon U \to V$ be a biholomorphism.
Then $f$ extends to a smooth diffeomorphism of $\bar{U}$ to $\bar{V}$.
\end{thm}


\begin{thebibliography}{9}
\bibitem{Fefferman:bihol}
Fefferman, Charles.
{\em \PMlinkescapetext{The Bergman kernel and biholomorphic mappings of pseudoconvex domains}}.
Invent. Math. {\bf 26} (1974), 1--65. 
\bibitem{Krantz:several}
Steven~G.\@ Krantz.
{\em \PMlinkescapetext{Function Theory of Several Complex Variables}},
AMS Chelsea Publishing, Providence, Rhode Island, 1992.
\end{thebibliography}
%%%%%
%%%%%
\end{document}
