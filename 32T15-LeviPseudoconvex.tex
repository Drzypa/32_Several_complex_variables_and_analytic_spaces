\documentclass[12pt]{article}
\usepackage{pmmeta}
\pmcanonicalname{LeviPseudoconvex}
\pmcreated{2013-03-22 14:30:37}
\pmmodified{2013-03-22 14:30:37}
\pmowner{jirka}{4157}
\pmmodifier{jirka}{4157}
\pmtitle{Levi pseudoconvex}
\pmrecord{10}{36048}
\pmprivacy{1}
\pmauthor{jirka}{4157}
\pmtype{Definition}
\pmcomment{trigger rebuild}
\pmclassification{msc}{32T15}
\pmclassification{msc}{32T05}
\pmrelated{DomainOfHolomorphy}
\pmrelated{Pseudoconvex}
\pmrelated{BiholomorphismsOfStronglyPseudoconvexDomainsExtendToTheBoundary}
\pmdefines{Levi form}
\pmdefines{strongly Levi pseudoconvex}
\pmdefines{strongly pseudoconvex}
\pmdefines{strictly pseudoconvex}
\pmdefines{weakly pseudoconvex}
\pmdefines{weakly Levi pseudoconvex}
\pmdefines{holomorphic tangent vector}

\endmetadata

% this is the default PlanetMath preamble.  as your knowledge
% of TeX increases, you will probably want to edit this, but
% it should be fine as is for beginners.

% almost certainly you want these
\usepackage{amssymb}
\usepackage{amsmath}
\usepackage{amsfonts}

% used for TeXing text within eps files
%\usepackage{psfrag}
% need this for including graphics (\includegraphics)
%\usepackage{graphicx}
% for neatly defining theorems and propositions
\usepackage{amsthm}
% making logically defined graphics
%%%\usepackage{xypic}

% there are many more packages, add them here as you need them

% define commands here
\theoremstyle{theorem}
\newtheorem*{thm}{Theorem}
\newtheorem*{lemma}{Lemma}
\newtheorem*{conj}{Conjecture}
\newtheorem*{cor}{Corollary}
\newtheorem*{example}{Example}
\theoremstyle{definition}
\newtheorem*{defn}{Definition}
\begin{document}
Let $G \subset {\mathbb{C}}^n$ be a \PMlinkname{domain}{Domain2} (open connected subset) with $C^2$
boundary, that is the boundary is locally the graph of a twice continuously
differentiable function.  Let $\rho \colon {\mathbb{C}}^n \to {\mathbb{R}}$
be a defining function of $G$,
that is $\rho$ is a twice continuously differentiable function such that
$\operatorname{grad} \rho (z) \not= 0$ for $z \in \partial G$ and $G =
\{ z \in {\mathbb{C}}^n \mid \rho(z) < 0 \}$ (such a function always exists).

\begin{defn}
Let $p \in \partial G$ (boundary of $G$).
We call the space of vectors $w = (w_1,\ldots,w_n) \in {\mathbb{C}}^n$
such that
\begin{equation*}
\sum_{k=1}^n \frac{\partial \rho}{\partial z_k} (p) w_k = 0 ,
\end{equation*}
the space of {\em holomorphic tangent vectors}
at $p$ and denote it
$T^{1,0}_p(\partial G)$.
\end{defn}

$T^{1,0}_p(\partial G)$ is an $n-1$ dimensional complex vector space
and is a subspace of the complexified \PMlinkname{real tangent space}{TangentSpace}, that is ${\mathbb{C}} \otimes_{\mathbb{R}} T_p(\partial G)$.

Note that when $n=1$ then the complex
tangent space contains just the zero vector.

\begin{defn}
The point $p \in \partial G$
is called {\em Levi pseudoconvex} (or just {\em pseudoconvex})
if
\begin{equation*}
\sum_{j,k=1}^n
\frac{\partial^2 \rho}{\partial z_j \partial \bar{z}_k} (p) w_j \bar{w}_k
\geq 0 ,
\end{equation*}
for all $w \in T^{1,0}_p(\partial G)$.  The point is
called {\em strongly Levi pseudoconvex} (or just {\em strongly pseudoconvex} or also {\em strictly pseudoconvex})
if the inequality above is strict.  The expression on the left is
called the {\em Levi form}.
\end{defn}

Note that if a point is not strongly Levi pseudoconvex then it is sometimes called a {\em weakly Levi pseudoconvex} point.

The Levi form really acts on an $n-1$ dimensional space, so the expression above may be confusing as it only acts on $T^{1,0}_p(\partial G)$ and not on all
of ${\mathbb{C}}^n$.

\begin{defn}
The domain $G$
is called Levi pseudoconvex
if every boundary point is
Levi pseudoconvex.  Similarly
$G$ is called strongly Levi pseudoconvex
if every boundary point is
strongly Levi pseudoconvex.
\end{defn}

Note that in particular all convex domains are pseudoconvex.

It turns out that $G$ with $C^2$ boundary is a domain of holomorphy
if and only if
$G$ is Levi pseudoconvex.

\begin{thebibliography}{9}
\bibitem{ber:submanifold}
M.\@ Salah Baouendi,
Peter Ebenfelt,
Linda Preiss Rothschild.
{\em \PMlinkescapetext{Real Submanifolds in Complex Space and Their Mappings}},
Princeton University Press,
Princeton, New Jersey, 1999.
\bibitem{Krantz:several}
Steven~G.\@ Krantz.
{\em \PMlinkescapetext{Function Theory of Several Complex Variables}},
AMS Chelsea Publishing, Providence, Rhode Island, 1992.
\end{thebibliography}
%%%%%
%%%%%
\end{document}
