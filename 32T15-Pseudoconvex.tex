\documentclass[12pt]{article}
\usepackage{pmmeta}
\pmcanonicalname{Pseudoconvex}
\pmcreated{2013-03-22 14:30:58}
\pmmodified{2013-03-22 14:30:58}
\pmowner{jirka}{4157}
\pmmodifier{jirka}{4157}
\pmtitle{pseudoconvex}
\pmrecord{5}{36056}
\pmprivacy{1}
\pmauthor{jirka}{4157}
\pmtype{Definition}
\pmcomment{trigger rebuild}
\pmclassification{msc}{32T15}
\pmclassification{msc}{32T05}
\pmsynonym{Hartogs pseudoconvex}{Pseudoconvex}
\pmrelated{LeviPseudoconvex}
\pmrelated{SolutionOfTheLeviProblem}
\pmrelated{ExhaustionFunction}

\endmetadata

% this is the default PlanetMath preamble.  as your knowledge
% of TeX increases, you will probably want to edit this, but
% it should be fine as is for beginners.

% almost certainly you want these
\usepackage{amssymb}
\usepackage{amsmath}
\usepackage{amsfonts}

% used for TeXing text within eps files
%\usepackage{psfrag}
% need this for including graphics (\includegraphics)
%\usepackage{graphicx}
% for neatly defining theorems and propositions
\usepackage{amsthm}
% making logically defined graphics
%%%\usepackage{xypic}

% there are many more packages, add them here as you need them

% define commands here
\theoremstyle{theorem}
\newtheorem*{thm}{Theorem}
\newtheorem*{lemma}{Lemma}
\newtheorem*{conj}{Conjecture}
\newtheorem*{cor}{Corollary}
\newtheorem*{example}{Example}
\newtheorem*{prop}{Proposition}
\theoremstyle{definition}
\newtheorem*{defn}{Definition}
\begin{document}
\begin{defn}
Let $G \subset {\mathbb{C}}^n$ be a domain (open connected subset).
We say $G$ is {\em pseudoconvex} (or {\em Hartogs pseudoconvex}) if there exists a continuous plurisubharmonic function $\varphi$ on $G$ such that
the sets $\{ z \in G \mid \varphi(z) < x \}$ are relatively compact
subsets of $G$ for all $x \in {\mathbb{R}}$.  That is we say that
$G$ has a continuous plurisubharmonic exhaustion function.
\end{defn}

When $G$ has a $C^2$ (twice continuously differentiable) boundary then this
notion is the same as \PMlinkname{Levi pseudoconvexity}{LeviPseudoconvex}, which
is easier to work with if you have such nice boundaries.  If you don't have
nice boundaries then the following approximation result can come in useful.

\begin{prop}
If $G \subset {\mathbb{C}}^n$ is pseudoconvex then there exist bounded,
strongly Levi pseudoconvex domains $G_k \subset G$ with $C^\infty$ (smooth)
boundary which are relatively compact
in $G$, such that $G = \bigcup_{k=1}^\infty G_k$.
\end{prop}

This is because once we have a $\varphi$ as in the definition we can actually find a $C^\infty$ exhaustion function.

The reason for the definition of pseudoconvexity is that it classifies domains of holomorphy.  One thing to note then is that every open domain in one complex
dimension (in the complex plane ${\mathbb{C}}$) is then pseudoconvex.

\begin{thebibliography}{9}
\bibitem{Krantz:several}
Steven~G.\@ Krantz.
{\em \PMlinkescapetext{Function Theory of Several Complex Variables}},
AMS Chelsea Publishing, Providence, Rhode Island, 1992.
\end{thebibliography}
%%%%%
%%%%%
\end{document}
