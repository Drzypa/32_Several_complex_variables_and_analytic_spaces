\documentclass[12pt]{article}
\usepackage{pmmeta}
\pmcanonicalname{PluripolarSet}
\pmcreated{2013-03-22 14:29:15}
\pmmodified{2013-03-22 14:29:15}
\pmowner{jirka}{4157}
\pmmodifier{jirka}{4157}
\pmtitle{pluripolar set}
\pmrecord{5}{36021}
\pmprivacy{1}
\pmauthor{jirka}{4157}
\pmtype{Definition}
\pmcomment{trigger rebuild}
\pmclassification{msc}{32U05}
\pmclassification{msc}{31C10}

% this is the default PlanetMath preamble.  as your knowledge
% of TeX increases, you will probably want to edit this, but
% it should be fine as is for beginners.

% almost certainly you want these
\usepackage{amssymb}
\usepackage{amsmath}
\usepackage{amsfonts}

% used for TeXing text within eps files
%\usepackage{psfrag}
% need this for including graphics (\includegraphics)
%\usepackage{graphicx}
% for neatly defining theorems and propositions
\usepackage{amsthm}
% making logically defined graphics
%%%\usepackage{xypic}

% there are many more packages, add them here as you need them

% define commands here
\theoremstyle{theorem}
\newtheorem*{thm}{Theorem}
\newtheorem*{lemma}{Lemma}
\newtheorem*{conj}{Conjecture}
\newtheorem*{cor}{Corollary}
\newtheorem*{example}{Example}
\newtheorem*{prop}{Proposition}
\theoremstyle{definition}
\newtheorem*{defn}{Definition}
\begin{document}
\begin{defn}
Let $G \subset {\mathbb{C}}^n$ and let
$f \colon G \to {\mathbb{R}} \cup \{ - \infty \}$ be a plurisubharmonic
function which is not identically $-\infty$.
The set ${\mathcal{P}} := \{ z \in G \mid f(z) = - \infty \}$ is
called a {\em pluripolar set}.
\end{defn}

If $f$ is a holomorphic function then $\log \lvert f \rvert$ is a plurisubharmonic function.  The zero set of $f$ is then
a pluripolar set.

\begin{thebibliography}{9}
\bibitem{Krantz:several}
Steven~G.\@ Krantz.
{\em \PMlinkescapetext{Function Theory of Several Complex Variables}},
AMS Chelsea Publishing, Providence, Rhode Island, 1992.
\end{thebibliography}
%%%%%
%%%%%
\end{document}
