\documentclass[12pt]{article}
\usepackage{pmmeta}
\pmcanonicalname{ExhaustionFunction}
\pmcreated{2013-03-22 14:32:41}
\pmmodified{2013-03-22 14:32:41}
\pmowner{jirka}{4157}
\pmmodifier{jirka}{4157}
\pmtitle{exhaustion function}
\pmrecord{5}{36092}
\pmprivacy{1}
\pmauthor{jirka}{4157}
\pmtype{Definition}
\pmcomment{trigger rebuild}
\pmclassification{msc}{32U10}
\pmclassification{msc}{32T35}
\pmrelated{Pseudoconvex}
\pmdefines{bounded exhaustion function}
\pmdefines{hyperconvex}

\endmetadata

% this is the default PlanetMath preamble.  as your knowledge
% of TeX increases, you will probably want to edit this, but
% it should be fine as is for beginners.

% almost certainly you want these
\usepackage{amssymb}
\usepackage{amsmath}
\usepackage{amsfonts}

% used for TeXing text within eps files
%\usepackage{psfrag}
% need this for including graphics (\includegraphics)
%\usepackage{graphicx}
% for neatly defining theorems and propositions
\usepackage{amsthm}
% making logically defined graphics
%%%\usepackage{xypic}

% there are many more packages, add them here as you need them

% define commands here
\theoremstyle{theorem}
\newtheorem*{thm}{Theorem}
\newtheorem*{lemma}{Lemma}
\newtheorem*{conj}{Conjecture}
\newtheorem*{cor}{Corollary}
\newtheorem*{example}{Example}
\theoremstyle{definition}
\newtheorem*{defn}{Definition}
\begin{document}
\begin{defn}
Let $G \subset {\mathbb{C}}^n$ be a domain and let $f \colon G \to
{\mathbb{R}}$ is called an {\em exhaustion function} whenever
\begin{equation*}
\{ z \in G \mid f(z) < r \}
\end{equation*}
is relatively compact in $G$ for all $r \in {\mathbb{R}}$.
\end{defn}

For example $G$ is pseudoconvex if and only if $G$ has a continuous
plurisubharmonic exhaustion function.

We can also define a bounded version.

\begin{defn}
Let $G \subset {\mathbb{C}}^n$ be a domain and let $f \colon G \to
(-\infty,c]$ for some $c \in {\mathbb{R}}$,
is called a {\em bounded exhaustion function} whenever
\begin{equation*}
\{ z \in G \mid f(z) < r \}
\end{equation*}
is relatively compact in $G$ for all $r < c$.
\end{defn}

A domain which has a bounded plurisubharmonic exhaustion function is usually
referred to as a {\em hyperconvex} domain.  Note that not all pseudoconvex
domains have a bounded plurisubharmonic exhaustion function.  For example
the Hartogs's triangle does not, though it does have an unbounded one.

\begin{thebibliography}{9}
\bibitem{Krantz:several}
Steven~G.\@ Krantz.
{\em \PMlinkescapetext{Function Theory of Several Complex Variables}},
AMS Chelsea Publishing, Providence, Rhode Island, 1992.
\end{thebibliography}
%%%%%
%%%%%
\end{document}
