\documentclass[12pt]{article}
\usepackage{pmmeta}
\pmcanonicalname{CRSubmanifold}
\pmcreated{2013-03-22 14:49:04}
\pmmodified{2013-03-22 14:49:04}
\pmowner{jirka}{4157}
\pmmodifier{jirka}{4157}
\pmtitle{CR submanifold}
\pmrecord{9}{36479}
\pmprivacy{1}
\pmauthor{jirka}{4157}
\pmtype{Definition}
\pmcomment{trigger rebuild}
\pmclassification{msc}{32V05}
\pmsynonym{CR manifold}{CRSubmanifold}
\pmsynonym{Cauchy-Riemann submanifold}{CRSubmanifold}
\pmrelated{GenericManifold}
\pmrelated{TotallyRealSubmanifold}
\pmrelated{TangentialCauchyRiemannComplexOfCinftySmoothForms}
\pmrelated{ACRcomplex}
\pmdefines{CR bundle}
\pmdefines{CR vector field}
\pmdefines{complex tangent space}
\pmdefines{complex bundle}
\pmdefines{space of antiholomorphic vectors}
\pmdefines{antiholomorphic vector}
\pmdefines{CR dimension}

\endmetadata

% this is the default PlanetMath preamble.  as your knowledge
% of TeX increases, you will probably want to edit this, but
% it should be fine as is for beginners.

% almost certainly you want these
\usepackage{amssymb}
\usepackage{amsmath}
\usepackage{amsfonts}

% used for TeXing text within eps files
%\usepackage{psfrag}
% need this for including graphics (\includegraphics)
%\usepackage{graphicx}
% for neatly defining theorems and propositions
\usepackage{amsthm}
% making logically defined graphics
%%%\usepackage{xypic}

% there are many more packages, add them here as you need them

% define commands here
\theoremstyle{theorem}
\newtheorem*{thm}{Theorem}
\newtheorem*{lemma}{Lemma}
\newtheorem*{conj}{Conjecture}
\newtheorem*{cor}{Corollary}
\newtheorem*{example}{Example}
\theoremstyle{definition}
\newtheorem*{defn}{Definition}
\begin{document}
Suppose that $M \subset {\mathbb{C}}^N$ is a real submanifold of real dimension $n.$  Take $p \in M,$ then let $T_p({\mathbb{C}}^N)$ be the tangent vectors of ${\mathbb{C}}^N$ at the point $p.$  If we identify ${\mathbb{C}}^N$ with
${\mathbb{R}}^{2N}$ by $z_j = x_j + i y_j,$ we can take the following vectors as our basis
\begin{equation*}
\frac{\partial}{\partial x_1} \Bigg\rvert_p,
\frac{\partial}{\partial y_1} \Bigg\rvert_p,
\ldots,
\frac{\partial}{\partial x_N} \Bigg\rvert_p,
\frac{\partial}{\partial y_N} \Bigg\rvert_p .
\end{equation*} 

We define a real linear mapping $J\colon T_p({\mathbb{C}}^N) \to T_p({\mathbb{C}}^N)$ such that for any $1 \leq j \leq N$ we have
\begin{equation*}
J \left(
\frac{\partial}{\partial x_1} \Bigg\rvert_p
\right)
= \frac{\partial}{\partial y_1} \Bigg\rvert_p
\qquad \text{ and }
J \left(
\frac{\partial}{\partial y_1} \Bigg\rvert_p
\right)
= - \frac{\partial}{\partial x_1} \Bigg\rvert_p .
\end{equation*}
Where $J$ is referred to as the complex structure on $T_p({\mathbb{C}}^N).$  Note that $J^2 = -I,$ that is applying $J$ twice we just negate the vector.

Let $T_p(M)$ be the tangent space of $M$ at the point $p$ (that is, those vectors of $T_p({\mathbb{C}}^N)$ which are tangent to $M$).

\begin{defn}
The subspace $T_p^c(M)$ defined as
\begin{equation*}
T_p^c(M) := \{ X \in T_p(M) \mid J(X) \in T_p(M) \}
\end{equation*}
is called the {\em complex tangent space} of $M$ at the point $p,$ and if the dimension of $T_p(M)$ is constant for all
$p \in M$ then the
corresponding vector bundle $T^c(M) := \bigcup_{p\in M} T_p^c(M)$ is called the {\em complex bundle} of $M$.
\end{defn}

Do note that the complex tangent space is a real (not complex) vector space, despite its rather unfortunate name.

Let ${\mathbb{C}} T_p(M)$ and ${\mathbb{C}} T_p({\mathbb{C}}^N)$ be the complexified vector spaces, by just allowing the coefficents of the vectors to
be complex numbers.  That is for 
$X = \sum a_j \frac{\partial}{\partial x_1} \Big\rvert_p + b_j \frac{\partial}{\partial x_1} \Big\rvert_p$ we allow $a_j$ and $b_j$ to be complex numbers.  Next we can extend the mapping $J$ to be ${\mathbb{C}}$-linear on these new vector spaces and still get that $J^2 = -I$ as before.  We notice
that the operator $J$ has two eigenvalues, $i$ and $-i$.

\begin{defn}
Let ${\mathcal{V}}_p$ be the eigenspace of ${\mathbb{C}} T_p(M)$ corresponding to the eigenvalue $-i.$  That is
\begin{equation*}
{\mathcal{V}}_p := \{ X \in {\mathbb{C}} T_p(M) \mid J(X) = -iX \} .
\end{equation*}
If the dimension of ${\mathcal{V}}_p$ is constant for all $p \in M,$ then 
we get a corresponding vector bundle ${\mathcal{V}}$ which we call the
{\em CR bundle} of $M.$  A smooth section of the CR bundle is then called
a {\em CR vector field}.
\end{defn}

\begin{defn}
The submanifold $M$ is called a {\em CR submanifold} (or just {\em CR manifold}) if the dimension of ${\mathcal{V}}_p$ is constant for all $p \in M.$
The complex dimension of ${\mathcal{V}}_p$ will then be called the
{\em CR dimension} of $M.$
\end{defn}

An example of a CR submanifold is for example a hyperplane defined by
$\operatorname{Im} z_N = 0$ where the CR dimension is $N-1.$  Another less trivial example is the Lewy hypersurface.

Note that sometimes ${\mathcal{V}}_p$ is written as $T_p^{0,1} (M)$ and referred to as the {\em space of antiholomorphic vectors}, where an {\em antiholomorphic vector} is a tangent vector which can be written in terms of the basis
\begin{equation*}
\frac{\partial}{\partial \bar{z}_j} \Bigg\rvert_p := 
\frac{1}{2}
\left(
\frac{\partial}{\partial x_j} \Bigg\rvert_p + i
\frac{\partial}{\partial y_j} \Bigg\rvert_p
\right) .
\end{equation*}

The CR in the name refers to Cauchy-Riemann and that is because the vector space
${\mathcal{V}}_p$ corresponds to differentiating with respect to $\bar{z}_j.$

\begin{thebibliography}{9}
\bibitem{ber:submanifold}
M.\@ Salah Baouendi,
Peter Ebenfelt,
Linda Preiss Rothschild.
{\em \PMlinkescapetext{Real Submanifolds in Complex Space and Their Mappings}},
Princeton University Press,
Princeton, New Jersey, 1999.
\bibitem{boggess}
Albert Boggess.
{\em \PMlinkescapetext{CR Manifolds and the Tangential Cauchy Riemann Complex}},
CRC, 1991.
\end{thebibliography}
%%%%%
%%%%%
\end{document}
