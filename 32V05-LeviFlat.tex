\documentclass[12pt]{article}
\usepackage{pmmeta}
\pmcanonicalname{LeviFlat}
\pmcreated{2013-03-22 17:39:41}
\pmmodified{2013-03-22 17:39:41}
\pmowner{jirka}{4157}
\pmmodifier{jirka}{4157}
\pmtitle{Levi flat}
\pmrecord{4}{40095}
\pmprivacy{1}
\pmauthor{jirka}{4157}
\pmtype{Definition}
\pmcomment{trigger rebuild}
\pmclassification{msc}{32V05}
\pmsynonym{Levi-flat}{LeviFlat}

\endmetadata

% this is the default PlanetMath preamble.  as your knowledge
% of TeX increases, you will probably want to edit this, but
% it should be fine as is for beginners.

% almost certainly you want these
\usepackage{amssymb}
\usepackage{amsmath}
\usepackage{amsfonts}

% used for TeXing text within eps files
%\usepackage{psfrag}
% need this for including graphics (\includegraphics)
%\usepackage{graphicx}
% for neatly defining theorems and propositions
\usepackage{amsthm}
% making logically defined graphics
%%%\usepackage{xypic}

% there are many more packages, add them here as you need them

% define commands here
\theoremstyle{theorem}
\newtheorem*{thm}{Theorem}
\newtheorem*{lemma}{Lemma}
\newtheorem*{conj}{Conjecture}
\newtheorem*{cor}{Corollary}
\newtheorem*{example}{Example}
\newtheorem*{prop}{Proposition}
\theoremstyle{definition}
\newtheorem*{defn}{Definition}
\theoremstyle{remark}
\newtheorem*{rmk}{Remark}

\begin{document}
Suppose $M \subset {\mathbb{C}}^n$ is at least a $C^2$ hypersurface.

\begin{defn}
$M$ is \emph{Levi-flat} if it is pseudoconvex from both sides, or equivalently if and only if the Levi form of $M$ vanishes identically.
\end{defn}

Suppose $M$ is locally defined by $\rho = 0$.
The vanishing of the Levi form is equivalent to the complex
Hessian of $\rho$ vanishing on all holomorphic vectors tangent to the hypersurface.
Hence $M$ 
is Levi-flat if and only if the complex bordered Hessian of $\rho$
is of rank two on the hypersurface.  In other words, it is not hard to see that
$M$ is Levi-flat if and only if
\begin{equation*}
\operatorname{rank}
\left[
\begin{matrix}
\rho & \rho_z \\
\rho_{\bar{z}} & \rho_{z\bar{z}}
\end{matrix}
\right]
= 2
\ \ \ \text{ for all points on $\{\rho = 0\}$. }
\end{equation*}
Here $\rho_z$ is the row vector
$\left[ \frac{\partial \rho}{\partial z_1} ,\ldots,
\frac{\partial \rho}{\partial z_n} \right] ,$ 
$\rho_{\bar{z}}$ is the column vector
$\left[ \frac{\partial \rho}{\partial z_1} ,\ldots,
\frac{\partial \rho}{\partial z_n} \right]^T ,$
and $\rho_{z\bar{z}}$ is the complex Hessian
$\left[
\frac{\partial^2 \rho}{\partial z_i \partial \bar{z}_j}
\right]_{ij}.$

Let $T^cM$ be the complex tangent space of $M,$ that is at each point $p \in M,$
define 
$T_p^cM = J(T_pM) \cap T_pM,$
where $J$ is the complex structure.
Since $M$ is a hypersurface the dimension
of $T_p^cM$ is always $2n-2,$ and so $T^cM$ is a subbundle of $TM.$  $M$ is Levi-flat
if and only if $T^cM$ is involutive.  Since the leaves are graphs of functions that satisfy
the Cauchy-Riemann equations, the leaves are complex analytic.  Hence, $M$ is Levi-flat, if and only if it is foliated by complex hypersurfaces.

The cannonical example of a Levi-flat hypersurface is the hypersurface defined in ${\mathbb{C}}^n$ by
the equation $\operatorname{Im} z_1 = 0$.  In fact, locally, all real analytic Levi-flat hypersurfaces
are biholomorphic to this example.

\begin{thebibliography}{9}
\bibitem{ber:submanifold}
M.\@ Salah Baouendi,
Peter Ebenfelt,
Linda Preiss Rothschild.
{\em \PMlinkescapetext{Real Submanifolds in Complex Space and Their Mappings}},
Princeton University Press,
Princeton, New Jersey, 1999.
Lars H\"ormander.
{\em \PMlinkescapetext{An Introduction to Complex Analysis in Several
Variables}},
North-Holland Publishing Company, New York, New York, 1973.
\bibitem{Krantz:several}
Steven~G.\@ Krantz.
{\em \PMlinkescapetext{Function Theory of Several Complex Variables}},
AMS Chelsea Publishing, Providence, Rhode Island, 1992.
\end{thebibliography}
%%%%%
%%%%%
\end{document}
