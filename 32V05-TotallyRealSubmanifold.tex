\documentclass[12pt]{article}
\usepackage{pmmeta}
\pmcanonicalname{TotallyRealSubmanifold}
\pmcreated{2013-03-22 14:56:05}
\pmmodified{2013-03-22 14:56:05}
\pmowner{jirka}{4157}
\pmmodifier{jirka}{4157}
\pmtitle{totally real submanifold}
\pmrecord{5}{36624}
\pmprivacy{1}
\pmauthor{jirka}{4157}
\pmtype{Definition}
\pmcomment{trigger rebuild}
\pmclassification{msc}{32V05}
\pmsynonym{totally real manifold}{TotallyRealSubmanifold}
\pmrelated{CRSubmanifold}
\pmrelated{GenericManifold}
\pmdefines{maximally totally real submanifold}
\pmdefines{maximally totally real manifold}
\pmdefines{maximally real manifold}

\endmetadata

% this is the default PlanetMath preamble.  as your knowledge
% of TeX increases, you will probably want to edit this, but
% it should be fine as is for beginners.

% almost certainly you want these
\usepackage{amssymb}
\usepackage{amsmath}
\usepackage{amsfonts}

% used for TeXing text within eps files
%\usepackage{psfrag}
% need this for including graphics (\includegraphics)
%\usepackage{graphicx}
% for neatly defining theorems and propositions
\usepackage{amsthm}
% making logically defined graphics
%%%\usepackage{xypic}

% there are many more packages, add them here as you need them

% define commands here
\theoremstyle{theorem}
\newtheorem*{thm}{Theorem}
\newtheorem*{lemma}{Lemma}
\newtheorem*{conj}{Conjecture}
\newtheorem*{cor}{Corollary}
\newtheorem*{example}{Example}
\newtheorem*{prop}{Proposition}
\theoremstyle{definition}
\newtheorem*{defn}{Definition}
\theoremstyle{remark}
\newtheorem*{rmk}{Remark}
\begin{document}
\begin{defn}
Suppose that $M \subset {\mathbb{C}}^N$ is a CR submanifold.  If the CR dimension of $M$ is 0, we say that $M$ is {\em totally real}.  If in addition
$M$ is \PMlinkname{generic}{GenericManifold}, then $M$ is said to be {\em maximally totally real} (or sometimes just {\em maximally real}).
\end{defn}

Note that if $M$ is maximally totally real, then the real dimension is automatically $N$, this is because $T_x^c(M) = T_x(M) \cap JT_x(M)$
(the complex tangent space) is of dimension 0, and thus $T_x(M)$ must be of real dimension $N$ if $M$ is to be a generic manifold.

\begin{thebibliography}{9}
\bibitem{ber:submanifold}
M.\@ Salah Baouendi,
Peter Ebenfelt,
Linda Preiss Rothschild.
{\em \PMlinkescapetext{Real Submanifolds in Complex Space and Their Mappings}},
Princeton University Press,
Princeton, New Jersey, 1999.
\end{thebibliography}
%%%%%
%%%%%
\end{document}
