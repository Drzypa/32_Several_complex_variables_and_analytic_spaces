\documentclass[12pt]{article}
\usepackage{pmmeta}
\pmcanonicalname{CRFunction}
\pmcreated{2013-03-22 14:57:10}
\pmmodified{2013-03-22 14:57:10}
\pmowner{jirka}{4157}
\pmmodifier{jirka}{4157}
\pmtitle{CR function}
\pmrecord{5}{36647}
\pmprivacy{1}
\pmauthor{jirka}{4157}
\pmtype{Definition}
\pmcomment{trigger rebuild}
\pmclassification{msc}{32V10}
\pmdefines{CR distribution}

\endmetadata

% this is the default PlanetMath preamble.  as your knowledge
% of TeX increases, you will probably want to edit this, but
% it should be fine as is for beginners.

% almost certainly you want these
\usepackage{amssymb}
\usepackage{amsmath}
\usepackage{amsfonts}

% used for TeXing text within eps files
%\usepackage{psfrag}
% need this for including graphics (\includegraphics)
%\usepackage{graphicx}
% for neatly defining theorems and propositions
\usepackage{amsthm}
% making logically defined graphics
%%%\usepackage{xypic}

% there are many more packages, add them here as you need them

% define commands here
\theoremstyle{theorem}
\newtheorem*{thm}{Theorem}
\newtheorem*{lemma}{Lemma}
\newtheorem*{conj}{Conjecture}
\newtheorem*{cor}{Corollary}
\newtheorem*{example}{Example}
\theoremstyle{definition}
\newtheorem*{defn}{Definition}
\begin{document}
\begin{defn}
Let $M \subset {\mathbb{C}}^N$ be a CR submanifold and let $f$ be a 
$C^k(M)$ ($k$ times continuously differentiable) function to ${\mathbb{C}}$
where $k \geq 1$.  Then $f$ is a {\em CR function} if for every CR vector
field $L$ on $M$ we have $Lf \equiv 0$.  A
\PMlinkname{distribution}{Distribution4}
$f$ on $M$ is called a
{\em CR distribution} if similarly every CR vector field annihilates $f$.
\end{defn}

For example restrictions of holomorphic functions in ${\mathbb{C}}^N$ to
$M$ are CR functions.  The converse is not always true and is not easy to
see.  For example the following basic theorem is very useful when you have
real analytic submanifolds.

\begin{thm}
Let $M \subset {\mathbb{C}}^N$ be a generic submanifold which is real
analytic (the defining function is real analytic).  And let $f \colon M \to
{\mathbb{C}}$ be a real analytic function.  Then $f$ is a CR function if
and only if $f$ is a restriction to $M$ of a holomorphic function
defined in an open neighbourhood of $M$ in ${\mathbb{C}}^N$.
\end{thm}

\begin{thebibliography}{9}
\bibitem{ber:submanifold}
M.\@ Salah Baouendi,
Peter Ebenfelt,
Linda Preiss Rothschild.
{\em \PMlinkescapetext{Real Submanifolds in Complex Space and Their Mappings}},
Princeton University Press,
Princeton, New Jersey, 1999.
\end{thebibliography}
%%%%%
%%%%%
\end{document}
