\documentclass[12pt]{article}
\usepackage{pmmeta}
\pmcanonicalname{BaouendiTrevesApproximationTheorem}
\pmcreated{2015-05-07 16:14:43}
\pmmodified{2015-05-07 16:14:43}
\pmowner{jirka}{4157}
\pmmodifier{jirka}{4157}
\pmtitle{Baouendi-Treves approximation theorem}
\pmrecord{7}{40093}
\pmprivacy{1}
\pmauthor{jirka}{4157}
\pmtype{Theorem}
\pmcomment{trigger rebuild}
\pmclassification{msc}{32V25}

\endmetadata

% this is the default PlanetMath preamble.  as your knowledge
% of TeX increases, you will probably want to edit this, but
% it should be fine as is for beginners.

% almost certainly you want these
\usepackage{amssymb}
\usepackage{amsmath}
\usepackage{amsfonts}

% used for TeXing text within eps files
%\usepackage{psfrag}
% need this for including graphics (\includegraphics)
%\usepackage{graphicx}
% for neatly defining theorems and propositions
\usepackage{amsthm}
% making logically defined graphics
%%%\usepackage{xypic}

% there are many more packages, add them here as you need them

% define commands here
\theoremstyle{theorem}
\newtheorem*{thm}{Theorem}
\newtheorem*{lemma}{Lemma}
\newtheorem*{conj}{Conjecture}
\newtheorem*{cor}{Corollary}
\newtheorem*{example}{Example}
\newtheorem*{prop}{Proposition}
\theoremstyle{definition}
\newtheorem*{defn}{Definition}
\theoremstyle{remark}
\newtheorem*{rmk}{Remark}

\begin{document}
Suppose $M$ is a real smooth manifold.  Let $\mathcal{V}$ be a subbundle of the complexified tangent space $\mathbb{C} TM$
(that is $\mathbb{C} \otimes TM$).  Let $n = \dim_{\mathbb{C}} \mathcal{V}$ and $d = \dim_{\mathbb{R}} M .$  We will say that $\mathcal{V}$ is \emph{integrable}, if it is integrable in the following sense.   Suppose that for any
point $p \in M,$
there exist $m = d-n$ smooth complex valued functions
$z_1,\ldots,z_m$ defined in a neighbourhood of $p$, such that the differentials $dz_1,\ldots,dz_m$ are $\mathbb{C}$-linearly independent and for all sections $L \in \Gamma(M,\mathcal{V})$ we have $Lz_k = 0$ for
$k = 1,\ldots,m.$  We say  $z=(z_1,\ldots,z_m)$ are \emph{\PMlinkescapetext{basic solutions}} near $p.$

We say $f$ is a \emph{\PMlinkescapetext{continuous solution}} if $Lf = 0$ for every $L \in \Gamma(M,\mathcal{V})$ in the sense
of distributions (or classically if $f$ is in fact smooth).

\begin{thm}[Baouendi-Treves]
Suppose $M$ is a smooth manifold of real dimension $d$ and $\mathcal{V}$ an integrable subbundle as above.
Let $p \in M$ be fixed and let $z=(z_1,\ldots,z_m)$ be basic solutions near $p$.  Then there exists a compact
neighbourhood $K$ of $p$, such that for any continuous solution $f \colon M \to \mathbb{C},$
there exists a sequence $p_j$ of polynomials in $m$ variables with complex coefficients such that
\begin{equation*}
p_j(z_1,\ldots,z_m) \to f
\text{ ~~~~ uniformly in $K.$}
\end{equation*}
\end{thm}

In particular we have the following corollary for CR submanifolds.  A real smooth CR submanifold
that is embedded in ${\mathbb{C}}^N$ has the CR vector fields as the integrable subbundle $\mathcal{V}$.
Also the coordinate functions $z_1,\ldots,z_N$ can be taken as the basic solutions.  We will require that
$M$ be a generic submanifold
rather than just any CR submanifold to make sure that ${\mathbb{C}}^N$ is of the minimal dimension.

\begin{cor}
Let $M \subset {\mathbb{C}}^N$ be an embedded real smooth generic submanifold and $p \in M$.  Then there exists a
compact neighbourhood $K \subset M$ of $p$ such that any continuous CR function $f$ is uniformly approximated on $K$ by polynomials
in $N$ variables.
\end{cor}

This result can be used to extend CR functions from CR submanifolds.  For example, if we can fill a certain set
with analytic discs attached to $M$, we can approximate $f$ on $K \subset M$ and by the maximum principle we will
be able to use the fact that uniform limits of holomorphic functions (in this case polynomials) are holomorphic.
A key point is that while $K$ is not arbitrary, it does not depend on $f$, it only depends on $M$ and $p$.

\begin{example}
Suppose $M \subset {\mathbb{C}}^2$ is given in coordinates $(z,w)$ by $\operatorname{Im} w = \lvert z \rvert^2 .$
Note that for
some $t > 0,$
the map $\xi \mapsto (t \xi, t)$ is an attached analytic disc.  By taking different $t > 0,$
we can fill the set $\{ (z,w) \mid \operatorname{Im} w \geq \lvert z \rvert^2 \}$ by analytic discs attached to $M.$
If $f$
is a continuous CR function on $M$, then there exists some compact neighbourhood $K$ of $(0,0)$ such that $f$
is uniformly approximated on $K$ by holomorphic polynomials.  By maximum principle we get that this sequence
of holomorphic polynomials converges uniformly on all the discs for $t < \epsilon$ for some $\epsilon > 0$ (such that the boundary of the disc lies in $K$).
Hence $f$ extends to a holomorphic function on $\epsilon > \operatorname{Im} w > \lvert z \rvert^2$, and which is
continuous on $\epsilon > \operatorname{Im} w \geq \lvert z \rvert^2$.
\end{example}

Using methods of the example it is possible (among many other results) to prove the following.

\begin{cor}
Suppose $M \subset {\mathbb{C}}^N$ be a smooth strongly pseudoconvex hypersurface and $f$ a continuous
CR function on $M.$  Then $f$ extends to a small neighbourhood on the pseudoconvex side of $M$ as a
holomorphic function.
\end{cor}

Using the above corollary we can prove the Hartogs phenomenon for hypersurfaces by reducing to the standard
Hartogs phenomenon (although the theorem also holds without pseudoconvexity with a different proof).

\begin{cor}
Let $U \subset {\mathbb{C}}^N$ be a domain with smooth strongly pseudoconvex boundary.
Suppose $f$ is a continuous CR function on $\partial U$.  Then there exists a function $f$
holomorphic in $U$ and continuous on $\bar{U},$ such that $F|_{\partial U} = f .$
\end{cor}

\begin{thebibliography}{9}
\bibitem{ber:submanifold}
M.\@ Salah Baouendi,
Peter Ebenfelt,
Linda Preiss Rothschild.
{\em \PMlinkescapetext{Real Submanifolds in Complex Space and Their Mappings}},
Princeton University Press,
Princeton, New Jersey, 1999.
\bibitem{boggess}
Albert Boggess.
{\em \PMlinkescapetext{CR Manifolds and the Tangential Cauchy Riemann Complex}},
CRC, 1991.
\end{thebibliography}
%%%%%
%%%%%
\end{document}
