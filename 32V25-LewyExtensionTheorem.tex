\documentclass[12pt]{article}
\usepackage{pmmeta}
\pmcanonicalname{LewyExtensionTheorem}
\pmcreated{2013-03-22 17:39:44}
\pmmodified{2013-03-22 17:39:44}
\pmowner{jirka}{4157}
\pmmodifier{jirka}{4157}
\pmtitle{Lewy extension theorem}
\pmrecord{4}{40096}
\pmprivacy{1}
\pmauthor{jirka}{4157}
\pmtype{Theorem}
\pmcomment{trigger rebuild}
\pmclassification{msc}{32V25}
\pmsynonym{Lewy extension}{LewyExtensionTheorem}

% this is the default PlanetMath preamble.  as your knowledge
% of TeX increases, you will probably want to edit this, but
% it should be fine as is for beginners.

% almost certainly you want these
\usepackage{amssymb}
\usepackage{amsmath}
\usepackage{amsfonts}

% used for TeXing text within eps files
%\usepackage{psfrag}
% need this for including graphics (\includegraphics)
%\usepackage{graphicx}
% for neatly defining theorems and propositions
\usepackage{amsthm}
% making logically defined graphics
%%%\usepackage{xypic}

% there are many more packages, add them here as you need them

% define commands here
\theoremstyle{theorem}
\newtheorem*{thm}{Theorem}
\newtheorem*{lemma}{Lemma}
\newtheorem*{conj}{Conjecture}
\newtheorem*{cor}{Corollary}
\newtheorem*{example}{Example}
\newtheorem*{prop}{Proposition}
\theoremstyle{definition}
\newtheorem*{defn}{Definition}
\theoremstyle{remark}
\newtheorem*{rmk}{Remark}

\begin{document}
Let $M \subset {\mathbf{C}}^n$ be a smooth real hypersurface.
Let $\rho$ be a defining function for $M$ near $p.$  That is, for some neighbourhood
of $p,$ the submanifold $M$ is defined by $\rho = 0 .$
For a neighbourhood $U \subset {\mathbb{C}}^n,$ define the set
$U_+$ to be the set $U \cap \{ \rho > 0 \} .$  We will say that 
$M$ has at least one negative eigenvalue if the Levi form defined by $\rho$ has at least one negative
eigenvalue.  That is, if
\begin{equation*}
\sum_{j,k=1}^n \frac{\partial^2 \rho (p)}{\partial z_j \partial \bar{z}_k} w_j \bar{w}_k < 0
~\text{ for some }~
w \in {\mathbb{C}}^n
~\text{ such that }~
\sum_{j=1}^n w_j \frac{\partial \rho(p)}{\partial z_j} = 0 .
\end{equation*}

\begin{thm}
Let $f$ be a smooth CR function on $M.$  Suppose that
near $p \in M$ the Levi form of $M$ has at least one positive eigenvalue at $p.$  Then there exists a
neighbourhood $U$ of $p,$ such that for every smooth CR function $f$ on $M,$ there exists a
function $F$ holomorphic in $U_+$ and $C^1$ up to $M,$ such that $F|_{U \cap M} = f|_{U \cap M}.$
\end{thm}

By considering $-\rho$ instead of $\rho$ as a defining function, we get the corresponding result for
at least one negative eigenvalue. 
If the Levi form of $M$ has both positive and negative eigenvalues at a point, then $f$ extends to both sides
of $M$ and is then a restriction of a holomorphic function.

A \PMlinkescapetext{key} point is the fact that $U$ is fixed and does not depend on $f.$  To see why this is necessary, imagine a Levi flat example.  Let $M$ be defined in ${\mathbb{C}}^2$ in coordinates $(z,w)$ by $\operatorname{Im} w = 0.$  The domains
$U_\epsilon := \{ \lvert \operatorname{Im} w \rvert < \epsilon \},$ for $\epsilon > 0,$ are pseudoconvex and hence
there exist functions
holomorphic on $\Omega_\epsilon$ (and hence CR on $M$) that do not extend past any point of the boundary.  No neighbourhood of a point on $M$ fits in all $U_\epsilon .$  So at least one nonzero eigenvalue of the Levi form
is needed.

The statement of this theorem is not exactly the theorem that Lewy formulated\cite{lewy}, but this is generally called the Lewy extension.  There have been many results
in this direction since Lewy's original paper, but this is the most \PMlinkescapetext{basic} result.

\begin{thebibliography}{9}
\bibitem{ber:submanifold}
M.\@ Salah Baouendi,
Peter Ebenfelt,
Linda Preiss Rothschild.
{\em \PMlinkescapetext{Real Submanifolds in Complex Space and Their Mappings}},
Princeton University Press,
Princeton, New Jersey, 1999.
\bibitem{boggess}
Albert Boggess.
{\em \PMlinkescapetext{CR Manifolds and the Tangential Cauchy Riemann Complex}},
CRC, 1991.
\bibitem{Hormander:several}
Lars H\"ormander.
{\em \PMlinkescapetext{An Introduction to Complex Analysis in Several
Variables}},
North-Holland Publishing Company, New York, New York, 1973.
\bibitem{lewy}
Hans Lewy.
{\em \PMlinkescapetext{On the local character of the solutions of an atypical linear differential equation in three variables and a related theorem for regular functions of two complex variables.}}
{\it Ann. of Math.} (2) {\bf 64} (1956), 514--522. 
\end{thebibliography}

%%%%%
%%%%%
\end{document}
