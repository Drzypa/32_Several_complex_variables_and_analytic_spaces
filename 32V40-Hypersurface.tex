\documentclass[12pt]{article}
\usepackage{pmmeta}
\pmcanonicalname{Hypersurface}
\pmcreated{2013-03-22 14:32:56}
\pmmodified{2013-03-22 14:32:56}
\pmowner{jirka}{4157}
\pmmodifier{jirka}{4157}
\pmtitle{hypersurface}
\pmrecord{8}{36099}
\pmprivacy{1}
\pmauthor{jirka}{4157}
\pmtype{Definition}
\pmcomment{trigger rebuild}
\pmclassification{msc}{32V40}
\pmclassification{msc}{14J70}
\pmrelated{Submanifold}
\pmdefines{smooth hypersurface}
\pmdefines{real analytic hypersurface}
\pmdefines{real hypersurface}
\pmdefines{local defining function}
\pmdefines{singular hypersurface}
\pmdefines{non-singular hypersurface}
\pmdefines{hypervariety}

% this is the default PlanetMath preamble.  as your knowledge
% of TeX increases, you will probably want to edit this, but
% it should be fine as is for beginners.

% almost certainly you want these
\usepackage{amssymb}
\usepackage{amsmath}
\usepackage{amsfonts}

% used for TeXing text within eps files
%\usepackage{psfrag}
% need this for including graphics (\includegraphics)
%\usepackage{graphicx}
% for neatly defining theorems and propositions
\usepackage{amsthm}
% making logically defined graphics
%%%\usepackage{xypic}

% there are many more packages, add them here as you need them

% define commands here
\theoremstyle{theorem}
\newtheorem*{thm}{Theorem}
\newtheorem*{lemma}{Lemma}
\newtheorem*{conj}{Conjecture}
\newtheorem*{cor}{Corollary}
\newtheorem*{example}{Example}
\theoremstyle{definition}
\newtheorem*{defn}{Definition}
\begin{document}
\begin{defn}
Let $M$ be a subset of ${\mathbb{R}}^n$ such that for every point
$p \in M$ there exists a neighbourhood $U_p$ of $p$ in ${\mathbb{R}}^n$
and a continuously differentiable function $\rho \colon U \to {\mathbb{R}}$ with
$\operatorname{grad} \rho \not= 0$ on $U$,
such that
\begin{equation*}
M \cap U = \{ x \in U \mid \rho(x) = 0 \} .
\end{equation*}
Then $M$ is called a {\em hypersurface}.
\end{defn}

If $\rho$ is in fact smooth then $M$ is a {\em smooth hypersurface} and
similarly if $\rho$ is real analytic then $M$ is a {\em real analytic
hypersurface}.  If
we identify ${\mathbb{R}}^{2n}$ with ${\mathbb{C}}^n$ and we have a
hypersurface there it is called a {\em real hypersurface} in
${\mathbb{C}}^n$.  $\rho$ is usually called the {\em local defining function}.
Hypersurface is really special name for a submanifold of codimension 1.  In fact if $M$ is just a topological manifold of codimension 1, then it is often also called a hypersurface.

A \PMlinkname{real}{RealAnalyticSubvariety} or complex analytic subvariety of codimension 1 (the zero set of a real or complex analytic function) is called a 
{\em singular hypersurface}.  That is the definition is the same as above, but
we do not require $\operatorname{grad} \rho \not= 0$.  Note that some authors leave out the word {\em singular} and then use {\em non-singular hypersurface} for a hypersurface which is also a manifold.  Some authors use the word {\em hypervariety} to describe a singular hypersurface.

An example of a hypersurface is the hypersphere (of radius 1 for simplicity) which has the defining equation
\begin{equation*}
x_1^2 + x_2^2 + \ldots + x_n^2 = 1 .
\end{equation*}

Another example of a hypersurface would be the boundary of a domain in
${\mathbb{C}}^n$ with smooth boundary.

An example of a singular hypersurface in ${\mathbb{R}}^2$ is for example the zero set
of $\rho(x_1,x_2) = x_1 x_2$ which is really just the two axis.  Note that this
hypersurface fails to be a manifold at the origin.


%FIXME: terrible reference (too specific)

\begin{thebibliography}{9}
\bibitem{ber:submanifold}
M.\@ Salah Baouendi,
Peter Ebenfelt,
Linda Preiss Rothschild.
{\em \PMlinkescapetext{Real Submanifolds in Complex Space and Their Mappings}},
Princeton University Press,
Princeton, New Jersey, 1999.
\end{thebibliography}
%%%%%
%%%%%
\end{document}
