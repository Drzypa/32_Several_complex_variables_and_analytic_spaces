\documentclass[12pt]{article}
\usepackage{pmmeta}
\pmcanonicalname{Submanifold}
\pmcreated{2013-03-22 14:47:20}
\pmmodified{2013-03-22 14:47:20}
\pmowner{jirka}{4157}
\pmmodifier{jirka}{4157}
\pmtitle{submanifold}
\pmrecord{8}{36440}
\pmprivacy{1}
\pmauthor{jirka}{4157}
\pmtype{Definition}
\pmcomment{trigger rebuild}
\pmclassification{msc}{32V40}
\pmclassification{msc}{53C40}
\pmclassification{msc}{53B25}
\pmclassification{msc}{57N99}
\pmrelated{Manifold}
\pmrelated{Hypersurface}
\pmdefines{real submanifold}
\pmdefines{codimension of a manifold}
\pmdefines{local defining functions}
\pmdefines{real submanifold}
\pmdefines{smooth submanifold}
\pmdefines{real analytic submanifold}
\pmdefines{regular submanifold}
\pmdefines{imbedded submanifold}
\pmdefines{embedded submanifold}
\pmdefines{germ of a submanifold}
\pmdefines{open submanifold}

% this is the default PlanetMath preamble.  as your knowledge
% of TeX increases, you will probably want to edit this, but
% it should be fine as is for beginners.

% almost certainly you want these
\usepackage{amssymb}
\usepackage{amsmath}
\usepackage{amsfonts}

% used for TeXing text within eps files
%\usepackage{psfrag}
% need this for including graphics (\includegraphics)
%\usepackage{graphicx}
% for neatly defining theorems and propositions
\usepackage{amsthm}
% making logically defined graphics
%%%\usepackage{xypic}

% there are many more packages, add them here as you need them

% define commands here
\theoremstyle{theorem}
\newtheorem*{thm}{Theorem}
\newtheorem*{lemma}{Lemma}
\newtheorem*{conj}{Conjecture}
\newtheorem*{cor}{Corollary}
\newtheorem*{example}{Example}
\newtheorem*{prop}{Proposition}
\theoremstyle{definition}
\newtheorem*{defn}{Definition}
\theoremstyle{remark}
\newtheorem*{rmk}{Remark}
\begin{document}
There are several conflicting definitions of what a submanifold is, depending on which author you are reading.  All that agrees is that a submanifold is a subset of a manifold which is itself a manifold, however how structure is inherited from the ambient space is not generally agreed upon.
So let's start with differentiable submanifolds of ${\mathbb{R}}^n$ as that's the most useful case.

\begin{defn}
Let $M$ be a subset of ${\mathbb{R}}^n$ such that for every point
$p \in M$ there exists a neighbourhood $U_p$ of $p$ in ${\mathbb{R}}^n$
and $m$ continuously differentiable functions $\rho_k \colon U \to {\mathbb{R}}$ where the differentials of $\rho_k$ are linearly independent,
such that
\begin{equation*}
M \cap U = \{ x \in U \mid \rho_k(x) = 0 , 1 \leq k \leq m \} .
\end{equation*}
Then $M$ is called a {\em submanifold} of ${\mathbb{R}}^n$ of dimension $m$
and of codimension $n-m$.
\end{defn}

If $\rho_k$ are in fact smooth then $M$ is a {\em smooth submanifold} and
similarly if $\rho$ is real analytic then $M$ is a {\em real analytic
submanifold}. If
we identify ${\mathbb{R}}^{2n}$ with ${\mathbb{C}}^n$ and we have a
submanifold there it is called a {\em real submanifold} in
${\mathbb{C}}^n$.  $\rho_k$ are usually called the {\em local defining functions}.

Let's now look at a more general definition.  Let $M$ be a manifold of dimension $m$.  A subset $N \subset M$ is said to have the {\em submanifold property} if there exists an integer $n \leq m$, such that for
each
$p \in N$ there is a coordinate neighbourhood $U$ and a coordinate function $\varphi \colon U \to {\mathbb{R}}^m$ of $M$ such that $\varphi(p) = (0,0,0,\ldots,0)$,
$\varphi(U \cap N) = \{ x \in \varphi(U) \mid x_{n+1} = x_{n+2} = \ldots = x_m = 0 \}$ if $n < m$ or $N \cap U = U$ if $n=m$.

\begin{defn}
Let $M$ be a manifold of dimension $m$.
A subset $N \subset M$ with the submanifold property
for some $n \leq m$ is called a {\em submanifold} of $M$ of dimension $n$ and of codimension $m-n$.
\end{defn}

The ambiguity arises about what topology we require $N$ to have.  Some authors require $N$ to have the relative topology inherited from $M$, others don't.

One could also mean that a subset is a submanifold if it is a disjoint
union of submanifolds of different dimensions.  It is not hard to see that
if $N$ is connected this is not an issue (whatever the topology on $N$ is).

In case of differentiable manifolds,
if we take $N$ to be a subspace of $M$ (the topology on $N$ is the relative topology inherited from $M$) and the differentiable structure of $N$ to
be the one determined by
the coordinate neighbourhoods above then we call $N$ a {\em regular submanifold}.

If $N$ is a submanifold and the inclusion map $i \colon N \to M$ is an imbedding, then we
say that $N$ is an {\em imbedded} (or {\em embedded}) {\em submanifold} of $M$.

\begin{defn}
Let $p \in M$ where $M$ is a manifold.  Then the equivalence class of all
submanifolds $N \subset M$ such that $p \in N$ where we say $N_1$ is
equivalent to $N_2$ if there is some open neighbourhood $U$ of $p$ such
that $N_1 \cap U = N_2 \cap U$ is called the {\em germ of a submanifold} through the point $p$.
\end{defn}

If $N \subset M$ is an open subset of $M$, then $N$ is called the {\em open submanifold} of $M$.  This is the easiest class of examples of submanifolds.

Example of a submanifold (a \PMlinkescapetext{regular and smooth submanifold} in fact) is the unit sphere in ${\mathbb{R}}^n$.  This is in fact a hypersurface
as it is of codimension 1.

\begin{thebibliography}{9}
\bibitem{boothby}
William M.\@ Boothby.
{\em \PMlinkescapetext{An Introduction to Differentiable Manifolds and
Riemannian Geometry}},
Academic Press, San Diego, California, 2003.
\bibitem{ber:submanifold}
M.\@ Salah Baouendi,
Peter Ebenfelt,
Linda Preiss Rothschild.
{\em \PMlinkescapetext{Real Submanifolds in Complex Space and Their Mappings}},
Princeton University Press,
Princeton, New Jersey, 1999.
\end{thebibliography}
%%%%%
%%%%%
\end{document}
