\documentclass[12pt]{article}
\usepackage{pmmeta}
\pmcanonicalname{LewyHypersurface}
\pmcreated{2013-03-22 14:49:01}
\pmmodified{2013-03-22 14:49:01}
\pmowner{jirka}{4157}
\pmmodifier{jirka}{4157}
\pmtitle{Lewy hypersurface}
\pmrecord{4}{36478}
\pmprivacy{1}
\pmauthor{jirka}{4157}
\pmtype{Example}
\pmcomment{trigger rebuild}
\pmclassification{msc}{32V99}

\endmetadata

% this is the default PlanetMath preamble.  as your knowledge
% of TeX increases, you will probably want to edit this, but
% it should be fine as is for beginners.

% almost certainly you want these
\usepackage{amssymb}
\usepackage{amsmath}
\usepackage{amsfonts}

% used for TeXing text within eps files
%\usepackage{psfrag}
% need this for including graphics (\includegraphics)
%\usepackage{graphicx}
% for neatly defining theorems and propositions
\usepackage{amsthm}
% making logically defined graphics
%%%\usepackage{xypic}

% there are many more packages, add them here as you need them

% define commands here
\begin{document}
The real hypersurface in $(z_1,\ldots,z_n) \in {\mathbb{C}}^n$ given by
\begin{equation*}
\operatorname{Im} z_n = \sum_{j=1}^{n-1} \lvert z_j \rvert^2
\end{equation*}
is called the {\em Lewy hypersurface}.  Note that this is a real hypersurface of real dimension $2n-1$.  This is an example of a non-trivial real hypersurface in complex space.  For example it is not biholomorphically equivalent to the hyperplane defined by $\operatorname{Im} z_n = 0$, but it is locally (not globally) biholomorphically equivalent to a unit sphere.


\begin{thebibliography}{9}
\bibitem{ber:submanifold}
M.\@ Salah Baouendi,
Peter Ebenfelt,
Linda Preiss Rothschild.
{\em \PMlinkescapetext{Real Submanifolds in Complex Space and Their Mappings}},
Princeton University Press,
Princeton, New Jersey, 1999.
\end{thebibliography}
%%%%%
%%%%%
\end{document}
